\documentclass[11pt, oneside]{article}   	% use "amsart" instead of "article" for AMSLaTeX format
%\geometry{landscape}                		% Activate for for rotated page geometry
%\usepackage[parfill]{parskip}    		% Activate to begin paragraphs with an empty line rather than an indent
\usepackage{graphicx}				% Use pdf, png, jpg, or eps§ with pdflatex; use eps in DVI mode
								% TeX will automatically convert eps --> pdf in pdflatex		
\usepackage{url}
\usepackage{amssymb}

\title{Structure overlaps}
\author{Ian Thompson}
\date{}							% Activate to display a given date or no date
\newcommand{\beq}{\begin{equation}}
\newcommand{\eeq}{\end{equation}}
\begin{document}
\maketitle
%\section{}
%\subsection{}


\section{Cluster overlaps}

Reactions involving the transfer of a cluster can either be treated as the
movement of a single `particle', or more microscopically as a combination of
simultaneous and sequential transfers.

The bound states of clusters treated as one particle are commonly found in a
Saxon-Woods potential with the same geometry parameters as are
found for low-energy scattering of the same nuclei. The number of
radial nodes is chosen by means of a shell-model counting of oscillator quanta,
to accommodate the Pauli exclusion principle, as follows.

Suppose that the valence nucleon shell has ${\cal N}_i =2 n_i + \ell_i$ quanta, that
is ${\cal N}=1$ for the $p$-shell, 2 for the $sd$-shell, etc. If the transferred valence cluster is taken
as composed of $v$ such nucleons, then the total energy of the cluster nucleons
will be $v {\cal N}_i \hbar\omega$. If the {\em internal} cluster configuration has
${\cal N}_{\rm int} = 2 n_{\rm int} + \ell_{\rm int}$ quanta (0 for $s$-shell clusters), then the
cluster-core motion in partial wave $L$ has $N$ nodes where \index{radial nodes}
\beq \label{cluster-q}
   2N + L + {\cal N}_{\rm int} = v {\cal N}_i .
\eeq
Usually $L$ is fixed by the spin of the composite state, so $N$ may be
determined.  Note that $N=0,1,\dots$ is the number of nodes {\em excluding} the
origin.%
\footnote{The {\tt NN} parameter for {\sc Fresco} bound states includes the
origin, so ${\tt NN} = N + 1 \geq 1$. \index{Fresco!radial nodes}}%
\index{overlap function|)}

~ \\
Section 5.3.3 from `Nuclear Reactions for Astrophysics',\\
\url{http://www.fresco.org.uk/book/reactions.htm}


\end{document}  