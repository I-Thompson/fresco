\documentclass[11pt]{article}
\usepackage{times}
\usepackage{mathptm}
\oddsidemargin -1.0cm \evensidemargin -1.0cm
\textheight 22.5cm \textwidth 17.00cm
\topmargin -1cm
%
\title{\Huge Fresco}
\date{Version FRES 3.1, December 2012\\  LLNL-SM-485670}
\author{I.J. Thompson\\
%
Nuclear Theory and Modelling, Livermore National Laboratory, Livermore CA 94551, USA\\
and 
\\Department of Physics, University of Surrey, Guildford GU2 7XH, England.
\\ %
\\Email:  I-Thompson@llnl.gov  or Ian@kernz.org}

\parskip 0pt
\parindent 0pt

\newcommand{\beqn}{\begin{eqnarray*}}
\newcommand{\eeqn}{\end{eqnarray*}}
%
\begin{document}
\maketitle
%\centerline{\large Fresco}
\section{Introduction}

{\large
\parskip 5pt
\parindent 10pt

FRESCO is a Coupled-Channels program that can have finite-range
transfer interactions among any number of mass partitions, and any
number of nuclear excitations in each partition.

This is accomplished by calculating and storing the kernels of the
non-local interactions, and then solving the coupled-channels set
iteratively.
At each iteration the stored kernels are used to integrate the wave
functions to generate the source terms for the next iteration.
Pade approximants can be used if the iterations diverge because of
strong couplings.

The program can also generate local couplings for the rotational
or single-particle excitations of either the projectile or the
target in any partition,
the local form factors for zero-range or local-energy-approximation
can also be used.
These various local interactions may also be included iteratively
(to give multistep DWBA), or alternatively the first few coupled
channels may be blocked together and solved by exact coupled-channels
methods.

Special treatment is provided for any long ranged Coulomb multipoles,
using James Christley's coupled-Coulomb functions CRCWFN,
and interpolation in partial waves is also possible.

The nonlocal kernels for single-particle transfers are calculated first
at a much smaller number of interpolation points,
and then expanded when necessary to calculate the source terms by
integrating
\beqn
S(R _ f ) = \int _ 0 ^ {R _ {match}}
                K _ {fi} (R _ f , R _ i ) u(R _ i ) dR _ i
\eeqn
where RMATCH and HCM, the step size, are given on card 1.
Since the kernel function $K_{fi}(R_f,R_i)$ is usually rapidly varying with
$D_{fi} = R_{f} - R_{i}$ (especially with heavy-ion reactions), and only slowly
varying with $R_{f}$ (if $D_{fi}$ is constant),  FRESCO calculates and stores
the function $K_{fi}'(R_{f},D_{fi})$ at intervals of RINTP (card 1) in $R_{f}$,
and intervals of HNL in $D_{fi}$. The $D_{fi}$ range considered is
CENTRE-RNL/2 to CENTRE+RNL/2, i.e. range of RNL centred at CENTRE,
and FRESCO later suggests improved values for RNL \& CENTRE.
The HNL reflects to physical variation of $K_{fi}'$ with $D_{fi}$, and can be
a fraction of HCM (for heavy ion reactions) or a multiple of HCM
(for light ion reactions especially with `prior' interactions).
If HNL is a submultiple of HCM, the program anticipates the interpolation
of $u(R_{i})$ in the equation above, and only stores an effective kernel
function at intervals of HCM.

The new variable MTMIN in the FRT version controls the method used to
calculate the form factors of the transfer kernels. If the L-transfer value
(estimated by $\ell_f + \ell_i$)
is larger than or equal to MTMIN, then the m-dependent expressions for the
spherical harmonics are used explicitly.
Otherwise, the
angular dependencies of the initial and final transfer states
are transformed into those of the channel variables $R_{i}$ \& $R_{f}$ using
Moshinsky's solid-harmonic transformation, as given for example
in
%\footnote{
Austern et al. Phys. Rev. {\bf 133} pp B 3 - 16.
%}
%
By careful control of the numerical approximations, the second method can
be extended to find heavy-ion transfer form factors, but the maximum
transferred angular momentum L is still limited by numerical cancellation
errors arising from the finite roundoff errors in the computer.
With 8-byte word lengths, the unit roundoff is approximately $10^{-15}$,
and this limits the transferred L to 6 or less, so the default value
of MTMIN is set to 6. The Moshinsky method is much faster when it is
accurate, so the default value of MTMIN has been set as high as practical.

A general description of the formalisms used is given in
%\footnote{
I.J. Thompson,
{\em Coupled Reaction Channels Calculations in Nuclear Physics},
Computer Physics Reports, {\bf 7} (1988) pp 167 - 212.
%}

}
% .
\newpage
\section{Input Cards}
\subsection{Definition Cards}
%
\subsubsection*{Card 0}
\begin{verbatim}
10A8 HEADNG(10)
\end{verbatim}
\smallskip

Heading to describe the nature of this run, for user's information only.
\bigskip

%\hangindent 0ex\hangafter 1
\subsubsection*{Card 1}
\begin{verbatim}
10F8.4  HCM,RMATCH,RINTP,HNL,RNL,CENTRE
\end{verbatim}
%

%\leftskip=7ex
Wave functions calculated at intervals of HCM up to abs(RMATCH).

If RMATCH $<$ 0, then read Card 1a for  coupled Coulomb wave functions.

\hangindent 5ex\hangafter 1
Non-local kernels $K_{fi}'(R_{f},D_{fi})$ calculated at $R_f$ intervals of RINTP,
and for a non-local ($D_{fi}$) range of RNL centred at CENTRE in steps of HNL.
RMATCH and RINTP are rounded to multiples of HCM, and HNL is rounded either to
a multiple or a sub-multiple of HCM.


%\leftskip=0em
%
\subsubsection*{Card 1a}
\begin{verbatim}
2F8.6, 2F8.2    RASYM,ACCUR,SWITCH,AJSWITCH
\end{verbatim}


\hangindent 7ex\hangafter 1
Use coupled Coulomb wave functions from CRCWFN out to asymptotic radius RASYM from
inner radius abs(RMATCH) for those partitions in which PWF is TRUE.
% Otherwise, use uncoupled Coulomb wavefunctions at radius abs(RMATCH).
\\
If RASYM $<$ 0, then determine the outer radius in order that classical
Coulomb trajectories reach an angle abs(RASYM) degrees.

\hangindent 7ex\hangafter 1
ACCUR is an accuracy parameter controlling the piecewise step length.
Default is 0.01: smaller values give greater accuracy.

\hangindent 7ex\hangafter 1
SWITCH is the radius at which to switch from Airy functions
to sines and cosines in piecewise method.  Default is 1000 fm.

\hangindent 7ex\hangafter 1
AJSWITCH - default is 0.0
\\
Normally the Coupled Coulomb wfns are matched to zero
and the Numerov integration is omitted,
when the Coulomb distance of closest approach is more than 4.5 fm outside
abs(RMATCH) (or  the  --CUTR distance if   CUTR negative).
This is only allowed when $J_{total} \geq $AJSWITCH.
\bigskip

\subsubsection*{Card 2}
\begin{verbatim}
2F4.0,      F8.4,  L2, I2,  1X,A1,   I2     4(I4,F4.0)
JTMIN,JTMAX,ABSEND,DRY,JSETS,  RELA, NEARFA,(JUMP(i),JBORD(i),i=2,5)
\end{verbatim}



\hangindent 5ex\hangafter 1
Calculate coupled-channels sets with total J in the interval
max(0,JTMIN) $<$  J  $<$ JTMAX,
stopping sooner if the absorbtion from the elastic channel
is less than ABSEND millibarns of three successive J/parity sets.
(If ABSEND $<$ 0, this takes the full J interval.)
\\
If JTMIN $<$ 0, then for J $<$ abs(JTMIN) include only the incoming channel
in the calculations, ignoring tranfers and excited states.
This is needed if the elastic scattering cross sections are to be given
correctly.

\hangindent 5ex\hangafter 1
DRY is normally F (false), but if T (true) the code does a `dry run'
to check that all arrays are of sufficient size.
All coupled channel sets are generated, but only the elastic
channels should be non-zero.

\hangindent 5ex\hangafter 1
JSETS = number of coupled-channels sets to be solved, for each energy
on Card 19.  JSETS = 0, blank or F, for no special limit.\\
If JSETS = 'P', include only positive parity coupled channels sets,\\
If JSETS = 'M' or 'N', include only negative parity coupled channels sets.

\hangindent 5ex\hangafter 1
RELA = `T' for relativistic kinematics for the incident projectile.
{\em Not yet implemented}.

\hangindent 8ex\hangafter 1  NEARFA
\\  = 0 or 1 for the usual cross sections,
\\  = 2 or -2 for printing `far side' cross sections too,
\\  = 3 or -3 for printing far and near side cross sections too.
\\  $>$ 0 for printing far \& near-sides for elastic channel only.
\\  $<$ 0 for printing far \& near-sides for all channels.



\hangindent 5ex\hangafter 1
JUMP(i),JBORD(i) : Calculate coupled-channels sets not for every J
value, but at intervals of JUMP(i) for J $\geq$ JBORD(i), for i=1,5.
(The program sets JUMP(1)=1 \& JBORD(1)=JTMIN, to give no J jumping initially).
The omitted J values are provided by interpolation on the scattering
amplitudes $A(m'M':mM; L)$ prior to calculating cross sections.
%.if SYSOUT = PRINT \newpage
% %.if SYSOUT = PAGE .or SYSOUT = PS \newpage

\subsubsection*{Card 3}
\begin{verbatim}
2I1,F6.4,F8.4,F7.4,I1,3F8.4  KQMAX,PP,THMIN,THMAX,THINC,KOORDS,CUTL,CUTR,CUTC
\end{verbatim}


\hangindent 5ex\hangafter 1
Give cross sections (and tensor analysing powers up to rank K = KQMAX)
for centre-of-mass scattering angle from THMIN
to abs(THMAX) in steps of THINC.\\
Elastic channels normally output the ratio to Rutherford, unless THMAX $< 0$.\\
Calculate analysing powers/polarisations for projectile (PP=0 or blank),
target (PP=1), ejectile (PP=2) or residual nucleus (PP=3).
PP=4 gives projectile (PP=0) analysing powers, along with Kyy results.

\hangindent 5ex\hangafter 1
KOORDS determines the coordinate systems used for the analysing powers:

\hangindent 8ex\hangafter 1  KOORDS
\\  = 0 : Madison coordinates (default)
\\  = 1 : Madison + Transverse
\\  = 2 : Madison + Transverse + Recoil
\\  = 3 : Madison + Transverse + Recoil + Hooton-Johnson



\hangindent 5ex\hangafter 1
CUTL = radial points per L (angular momentum of partial wave)
of lower radial cutoff when integrating the radial equations. Default = -1.6\\
When CUTL$>$0, use $\ell=J_{total}$ (total angular momentum of CC set),\\
When CUTL$<$0, use $\ell=L_{in}$ (orbital angular momentum of incoming partial wave).\\
Using  CUTL$<$0 gives more accurate analysing powers.

\hangindent 5ex\hangafter 1
CUTR = lower radial cutoff (fm).  Use max(CUTL*$\ell$*HCM,CUTR).\\
If CUTR $<$ 0, put cutoff at point-Coulomb turning point $- |$CUTR$|$.

\hangindent 5ex\hangafter 1
CUTC = lower radial cutoff (in fm) for off-diagonal couplings.
\newpage

\subsubsection*{Card 4}
\begin{verbatim}
F6.4,I2, I4,  I2,    I2,  A1,  I3, 2I4        I2,    F6.4
IPS, IT0,ITER,IBLOCK,PADE,ISO, NNU,MAXL,MINL, MTMIN, EPC,

2F8.4     2I2,      2e8.1,              4x,i4
ERANGE,DK,INH,plane,smallchan,smallcoup,   initwf
\end{verbatim}


\hangindent 5ex\hangafter 1
Solve the coupled channels equations by at least IT0 iterations, and
up to ITER iterations.
Stop sooner if the absolute differences between successive S-matrix
elements (scaled by (2J+1)/(2.JTMAX+1)) are less than IPS percent.
(Excited state pairs with IGNORE set on Card 7
are not counted against IPS).

\hangindent 5ex\hangafter 1
Putting IT0=ITER zero solves only the elastic channel (along with the
IBLOCK channels: see below).
Putting IT0=ITER = 1 or 2 etc. gives 1 or 2-step DWBA.
\\
Normally, a run is terminated if more than ITER steps are required for
convergence. Setting ITER $<$ 0 allows continuation even after
convergence has failed after abs(ITER) iterations.
\\
Iterations are normally also stopped if the successive differences are smaller
than the errors estimated for the numerical integration of the coupled
equations. Setting IPS $< 0$ uses abs(IPS), without this extra check.

\hangindent 5ex\hangafter 1
IBLOCK is the number of pairs of excitation levels
(starting from partition 1, excitation 1) that are coupled exactly by
blocking together.\\
If IBLOCK$<$1, then read Card 4.5 (see below) for R-matrix solution of the
coupled equations.
%\\

\hangindent 8ex\hangafter 1
PADE
\\ = 0 for no Pade acceleration,
\\ = 1 for Pade acceleration by the epsilon algorithm.



\hangindent 8ex\hangafter 1
ISO (not implemented in this version)

\hangindent 5ex\hangafter 1
NNU is the number of Gaussian integration points
in the angular
integration used for the non-local transfer kernels. NNU should
be a multiple of 6; NNU = 18 is the minimum, and 24 or 36 give
acceptable accuracy for all the reactions tried so far.


\hangindent 5ex\hangafter 1
MAXL,MINL are the maximum and minimum L values
for the non-local
kernels. If zero, MAXL has the default value JTMAX+6, and
if MINL $<$ 0 it takes the default value $|$JTMIN$|$-6.


\hangindent 5ex\hangafter 1
MTMIN is the lowest L-transfer for calculating transfer form factors
using the m-dependent expressions for spherical harmonics.
Putting MTMIN = 0 gives default value MTMIN = 6 (use MTMIN $<$ 0 to avoid
invoking default, if all transfers are to use this method).


\hangindent 5ex\hangafter 1
EPC = percentage cutoff accuracy in the
NNU angular integration.
If zero, the default is (30/NNU)$^2$\%.

\hangindent 5ex\hangafter 1
ERANGE, DK   {\em -- ignored}

\hangindent 8ex\hangafter 1  INH
\\ = 0 : zero-range transfer forms in intervals of HCM exactly
\\ = 1 : stored in steps of HCM * (proj. core)/(proj. composite mass)
\\ = 2 : stored in steps of HCM * (targ. core)/(targ. composite mass)\\
    So INH=2 corrects for longitudinal recoil during transfers with zero-range projectiles.



\hangindent 9ex\hangafter 1
PLANE = 1,3: zero Coulomb potential for elastic channel\\
      = 2,3: zero Coulomb potentials for all nonelastic channels.


\hangindent 5ex\hangafter 1
SMALLCHAN = fraction of unitarity to define a 'small channel'.\\
A channel that is 'small' for NSMALL=2 times is dropped permanently.

\hangindent 5ex\hangafter 1
SMALLCOUP = if all nonelastic channels are weaker than the fraction
SMALLCOUP of unitarity, then permanently change from coupled-channels to DWBA.

\hangindent 5ex\hangafter 1
INITWF = file number from which to read fixed channel  wave functions during all iterations. The channels whose wfs are changed must have the same
IT  index, that counts cards 7 as excited state pairs.
This file as the same format as the fort.17 produced when WDISK is nonzero. 
The INITWF $>$ 0 for formatted wf file, and $<$ 0 for unformatted (the same sign convention as for WDISK).

\subsubsection*{Card 4.5}
If IBLOCK$<$0, then read:
\begin{verbatim}
2i4,           I1,    L1,     I2,  i2,   f6.4, 3f8.4,
NRBASES,NRBMIN,BUTTLE,PRALPHA,PCON,MEIGS,RMATR,EBETA(2),WEAK
\end{verbatim}


\hangindent 5ex\hangafter 1
NRBASES = target number of radial basis states in each channel.
(Use 2*NRBASES for the elastic and first-inelastic channel for more accuracy).

\hangindent 5ex\hangafter 1
NRBMIN = minimum number of radial basis states.

\hangindent 5ex\hangafter 1
BUTTLE = 4 for none, 0,2 for complex, 1,3 for real
(2,3 without energy shift) Buttle correction (default 0)

\hangindent 5ex\hangafter 1
PRALPHA = print basis-state eigenvalues to files fort.60,61,62,63

\hangindent 5ex\hangafter 1
PCON = trace variable for calculation of radial basis states (same
meaning as IPC on Cards 13 for single-particle bound states).

\hangindent 5ex\hangafter 1
MEIGS = maximum number of bound states to find, if ENLAB $<$ 0 on Card 19.

\hangindent 5ex\hangafter 1
RMATR = R-matrix matching radius (default is RMATCH on Card 1).
Warning: RMATR will be changed to an even multiple of HCM.

\hangindent 5ex\hangafter 1
EBETA(i) = energy $\hbar^2k^2/2m$, where $k= f'/f$, the logarithmic derivative
for all radial basis states at $r$=RMATR, with $k$ having the same
sign as EBETA(i). Use i=1,2 for positive,negative parity coupled channels
sets (respectively).

\hangindent 5ex\hangafter 1
If R-matrix solutions are selected (by IBLOCK$<$0) then all channels
are `blocked' together and solved in a full CRC procedure.

\hangindent 5ex\hangafter 1
If WEAK$>$0, then non-elastic columns of the R-matrix are set to zero,
  when penetrabilities $<$ WEAK.


\subsection{Trace Control Variables:}
%
\subsubsection*{Card 5}
\begin{verbatim}
14I2
CHANS,LISTCC,TRENEG,CDETR,SMATS,XSTABL,NLPL,WAVES,LAMPL,VEFF,KFUS,WDISK,
               BPM,MEL,CDCC,NFUS
\end{verbatim}
A value of 0 gives no trace,
increasing values give progressively more printed output.
\\
{\em Decremented} variables are decreased by 1 on each use.

\bigskip

\hangindent 12ex\hangafter 1
CHANS  $\geq$ 1 : Print the sets of coupled partial waves for each J,parity.
Decremented.

\hangindent 8ex\hangafter 1  LISTCC
\\  = 1 : Print coupling coefficients between these channels. Decremented.
\\  = 1,2,... Print progressively more detail of couplings.



\hangindent 8ex\hangafter 1  TRENEG
\\  $\geq$ 1 : Print all potentials as they are calculated from Cards 10
\\  $\geq$ 3 : Print all potentials as they are calculated from Cards 9



\hangindent 8ex\hangafter 1  CDETR
\\  $\geq$ 1 : Print information on the solving of the coupled equations.
(decremented).

\hangindent 8ex\hangafter 1
SMATS
\\  $\geq$ 1 : Print absorbtion \& reaction cross sections for successive
partitions and excitations.
\\  $\geq$ 2 : Print elastic S-matrix elements ($S_{el}$).
Also `punch' these elastic $S_{el}$ on output file 7,
in format (2F15.10,L,J,JTOT) for $S_{el}$ complex, L, J and JTOT.
See WDISK below for description of these quantum numbers.
\\  $\geq$ 3 : Print all S-matrix elements for the `grazing partial waves'
defined by $0.05 < Re(S_{el}) < 0.95$
\\  $\geq$ 4 : Always print all the S-matrix elements.
\\  $\geq$ 5 : Print all S-matrix elements at each iteration of the coupled
equations (or, if PADE  $>$ 0, the Pade approximant)
\\  $\geq$ 6 : Print all actual S-matrix elements at each iteration (these
may be divergent before Pade acceleration).





\hangindent 12ex\hangafter 1
XSTABL $\neq$ 0 : If XSTABL is non-zero, in file 16 punch output cross sections
for all excitation levels in all partitions.
A header card in FORMAT(5I6) gives partition IC, level pair IA,
number of tensor ranks of analysing powers 1 $<$ KQ1PR $<$ XSTABL,
number of angles NANGL, and NEARF.  NEARF=1 for total cross section,
2 for far-side component, and 3 for near-side component.
\\
Then follow NANGL print operations in
{\tt FORMAT(1P,6E12.4)}, repeating
the {\tt FORMAT} for each operation if KQ1PR is large,
of THETA, elastic xs (mb), $T_{10}$, $iT_{11}$,
$T_{20}$, $ T_{21}$, $ T_{22}$, $ iT_{30}$, $ iT_{31}$
etc.

\hangindent 12ex\hangafter 1
NLPL   $>$ 0 : print a `contour plot' of the non-local kernels
$K_{fi}(R_{f},D_{fi})$.  This is useful to determine if the parameters on card 1
are adequate.  Decremented.

\hangindent 8ex\hangafter 1  WAVES
\\  $\pm$1 or $\pm$3 : print out wave function solutions of the coupled
equations at the end of the iterations.
(If WAVES$<$0 : print out the RATIO of the w/f to its asymptotic
form $((G-iF) - S.(G+iF)).i/2$)
\\  2 or 3     : print out the source terms at
{\em each} iteration of the coupled equations.



\hangindent 8ex\hangafter 1  LAMPL
\\  $\neq$ 0: Print out (on Fortran file 36)
the coefficients $A(m'M':mM; L)$ for the Legendre
coefficients in the scattering amplitude for the partition number
abs(LAMPL),
{\em and}
print out the $f(m'M':mM; \theta)$ for each angle $\theta$.
\\ $<$0 : only print out on file 37 the amplitudes $f$, not the $A$'s,
for partition abs(LAMPL).
\\
The phase convention here is that there is no Coulomb phase shift
for $L = 0$ in the Coulomb scattering amplitude : factors such as
$\exp i(\sigma_L-\sigma_0))$  appear in the $A$'s.



\hangindent 8ex\hangafter 1
VEFF
\\  $\neq$ 0 : Calculate the `coupled channels effective potential'
found be averaging the `trivially equivalent potential' over all
the $J,\pi$ sets, with weights of the elastic wave functions
times the reaction cross section, in each set.
\\  $<$  0 : Add this effective potential to optical potential
of the elastic channel before printing.
\\  = -2 or +2 : Exclude partial waves with elastic $S$-matrix element
$S_\ell< 0.1$ from the averaging sum.
\\ The results show the real and imaginary parts for successive
values of $J-L$, for the projectile only.





\hangindent 8ex\hangafter 1
KFUS,NFUS
\\  $>$ 0 :  Calculate `core fusion' using the imaginary and scalar
parts of potential number KFUS (i.e. cards 10 with TYPE = 1 or 2,
and KP = KFUS), also for the first NFUS inelastic chamnels



\hangindent 8ex\hangafter 1  WDISK
\\  = 1 :  Print elastic wave functions on output file 17, FORMATTED
\\  = 2 :  Print   all   wave functions on output file 17, FORMATTED
\\  =-1 :  Print elastic wave functions on output file 17, UNFORMATTED
\\  =-2 :  Print   all   wave functions on output file 17, UNFORMATTED


\begin{quotation}
\noindent
The following data formats are used when WDISK $>$ 0 :
\begin{verbatim}
card A: (I4,2F8.4,F8.1,I3)
         NR,H,ENLAB,JTOTAL,PARITY,MP,MT,ZP,ZT :
           number radial points, step size, lab. energy,  J,pi,
           projectile and target masses and charges
card B: (2I4,2F6.1,I4,F6.1,2F15.10,f12.8)
         IT,L,J,JTOT,LIN,JIN,SMAT (complex), ETA
\end{verbatim}
where

\hangindent 8ex\hangafter 1
IT = index to excited state pair, counts cards 7.\\
L  = partial wave\\
J  = L + projectile spin\\
JTOT = total spin = J$_{total}$ = J + target spin\\
LIN  = incoming partial wave\\
JIN  = incoming J value.\\
SMAT = S matrix element for this partial wave.\\

\begin{verbatim}
card C: (6E12.4)   (psi(I),I=1,NR)   wave function
card C is repeated until NR complex values given
NB: the first point psi(1) = 0 always, as at r=0
\end{verbatim}
Cards B \& C are repeated for each channel, until IT $<$ 0.
\\
When WDISK $<$ 0, successive records contain the two real values
of psi(I), starting IN THIS CASE, from I=2 (i.e. $r=h$).
\end{quotation}

\hangindent 8ex\hangafter 1  BPM
\\  $\geq$ 1 : Calculate fusion cross sections in the Barrier Penetration
Model using first the bare potential, and then the bare potential + the
`weighted equivalent potential' calculated when VEFF $\neq$ 0.
\\  $\geq$ 2 : Print out L-distributions of the fusion cross section.


\hangindent 8ex\hangafter 1  MEL   {\em -- ignored}

\hangindent 8ex\hangafter 1  CDCC
\\  $\neq$ 0:
Print out the $f(m'M':mM; \theta)$ for each angle $\theta$
on file 57 for partition {\tt PEL}, after the following information:
{\small
\begin{verbatim}
card A: (F10.4,3F8.4)   ENLAB,Bproj,H2SM,e^2    lab energy,projectile binding energy
                                                hbar^2/2.m, e^2
card B: (4f8.4)         massp,masst,massc,massv masses projectile, target, core, valence
card C: (4f8.4)         Zp,Zt,Zc,Zv             charges
card D: (4A8)           namep,namet,namec,namev names
card E: (4f8.1)         Jp,Jt,Jc,Jv             g.s. spins
card F: (4i8)           Pp,Pt,Pc,Pv             g.s. parities
card G: (4I4)           NBINS,NKMAX,NEXB,NNJMAX no. CDCC bins, max NK,
                                                no. excited states, max(2*Jex+1)
card H: (I4,2f8.4)      NANGL,THMIN,THINC       (cm angular range from Card 3)
for each of the NBINS bins:
 card I:(i2,2f4.1,3f8.4,2i4)
         l,j,Emid,kmin,kmax,NK,KN,ISC
            l,j: quantum numbers  (s==Jv)
            Emid:  centre of bin with respect to continuum threshold
            kmin,kmax,NK: Min,max and number of k values in bin integral
            KN:  original KN index for bin state
            ISC:  normalisation used for bin
     for each IK=1,NK
         card J: (10f8.4) delta(IK): nuclear phase shift used in bin integral (radians)

for each excited state pair in the entrance partition: IA=1,NEXB::
    card K: (f4.1,i4,f8.4,i4) Jex,Parity,Eex,IBIN:
            Jex :          spin of this projectile excited state
            Parity:        parity of this projectile state
            Eex:           excitation energy of this state above g.s.
            IBIN:          (first) bin defined for this excited state
    for each IANG=1,NANGL: read complex numbers:
    card L: (6E12.4): ((FAM(MEX,MP,IANG,IA),MEX=1,2*Jex(IA)+1),MP=1,2*Jp+1)
\end{verbatim}
}
The phase convention again is that there is no Coulomb phase shift
for $L = 0$ in the Coulomb scattering amplitude : factors such as
$\exp i(\sigma_L-\sigma_0))$  appear in the $A$'s.


Summary of bin normalisation factors for different ISC values:

\hangindent 8ex\hangafter 1
ISC
\\ = 2:  $\exp(-i\delta(k))$
\\ = 4:  $\sin(\delta(k))\exp(-i\delta(k))$
\\ = 12:  $k\exp(-i\delta(k))$
\\ = 14:  $k\sin(\delta(k))\exp(-i\delta(k))$

 
\newpage
\subsection{Partitions and Excited States}
%
\subsubsection*{Card 6}
(for each partition IC=1,2,... repeat until a blank card 8) :
\begin{verbatim}
A8,      2F8.4,           I4,   L1,1X, A8,      2F8.4,           F8.4
NAME(1,),MASS(1,),ZZ(1,), NEX(),PWF(), NAME(2,),MASS(2,),ZZ(2,), QVAL(IC)

name-----mass-----charge-              name-----mass----charge-
----projectile----------               ---------target--------
\end{verbatim}

Abs(NEX) = number of pairs of excitations of projectile and/or target,

PWF = T, if acceleration of long-range Coulomb couplings to be
performed when card 1a present. Default is T.

QVAL = $Q$ value of this partition,
relative to the $Q$-values of the other partitions, for Ep = Et = 0.

If NEX $<$ 0, then cross sections for this partition are not printed.

\subsubsection*{Card 7}
(repeat NEX times before the next Card 6) :
\begin{verbatim}
F4.1,I2,   I2,    F8.4,2F4.1, I4,1X,A1
Jp,  COPYp,BANDp ,Ep,  Kp,Tp, KP,PWF,
-------projectile state-----

F4.1,I2,   I2,    F8.4,2F4.1,  2L2,         2I4
Jt,  COPYt,BANDt ,Et,  Kt,Tt,  EXCH,IGNORE, INFAM,OUTFAM
-----target state-----------
\end{verbatim}

\hangindent 0ex
This card is repeated NEX times, IA=1,NEX

\hangindent 7ex\hangafter 1
For both projectile (p) and target (t), the following are read :

\hangindent 7ex\hangafter 1
J     = spin of the state

\hangindent 7ex\hangafter 1
BAND  $>$ 0 for positive parity state, and $<$ 0 for negative parity.
The value of BAND defines the rotational band : all states of the
same BAND may be coupled by rotational mechanisms.

\hangindent 7ex\hangafter 1
If COPY is positive, then this level is a copy of a
previous level number (COPY $<$ IA) in the same partition.

\hangindent 7ex\hangafter 1
If COPYp $<$ 0, then this level is an exchange copy of
{\em the same IA numbered}
level in the previous partition abs(COPY).
That is, a copy with projectile and target nuclei exchanged.

\hangindent 7ex\hangafter 1
If COPYt = -IC, then the target state is an exchanged projectile.
If that state has spin zero, then only even partial waves are allowed.

\hangindent 7ex\hangafter 1
COPYs are needed to define identical states of one nucleus in different
pairings IA.   Spectroscopic amplitudes (see Cards 17) should be defined
for only the original `copied' level.

\hangindent 7ex\hangafter 1
E     = excitation energy of state relative to ground state.

\hangindent 7ex\hangafter 1
K     = K of rotation band containing this state.
(If zero, use spin of first level in this band.)

\hangindent 7ex\hangafter 1
T     = isospin of this state (if zero, use INT((mass-2*charge)/2))
(The T value is not used by this version of FRESCO).

\hangindent 7ex\hangafter 1
KP = index of optical potential for this pair of excited states.

\hangindent 7ex\hangafter 1
PWF= T to use CRCWFN  coupled Coulomb wave functions in this
partition (see Cards 1,1a)
\\
If zero, use as default KP = IC, the number of this partition.

\hangindent 10ex\hangafter 1
If EXCH = T calculate and write out all cross sections and amplitudes
for $180-\theta$.


\hangindent 10ex\hangafter 1
If IGNORE = T, ignore the convergence of this  state pair
(see IPS discussion on Card 4).

\hangindent 10ex\hangafter 1
If INFAM $\ne$ 0,  read in complex amplitudes later to add to
calculated scattering amplitudes (see cards A3).
(If INFAM $<$ 0, read in \underline{only one} amplitude (non spin-flip)
for each scattering angle.)

\hangindent 10ex\hangafter 1
If OUTFAM $\ne$ 0,  write out complex amplitudes later to add to
calculated scattering amplitudes (see cards A3).
(If OUTFAM $<$ 0, write out \underline{only one} amplitude (non spin-flip)
for each scattering angle: the average of diagonal amplitudes.)

%\bigskip
%
\subsubsection*{Card A3}

Cards A3 :  Extra scattering amplitudes --
\bigskip

\hangindent 10ex\hangafter 1
If INFAM on a Card 7 is non-zero, then read on file $|$INFAM$|$, on a new card
for each scattering angle, some complex amplitudes $f(\theta)$,
to be added to those calculated by the program before the cross
sections are calculated.


\hangindent 10ex\hangafter 1
If INFAM $<$ 0, then only
{\em one} f(external) is read, and this is used for all the diagonal
f(mM:mM), the off-diagonal values being zero.

\hangindent 10ex\hangafter 1
If INFAM $>$ 0, then all the $f(m'M':mM)$ are read in.
%
\subsubsection*{Card 8}
\begin{verbatim}
Blank card for end of reading partitions.
\end{verbatim}
%.ty Card 8 ended on page &$PN., with &$LC lines remaining.
% %.if SYSOUT = PRINT \newpage
\newpage
\subsection{Potentials}
%
\bigskip

 Cards 9,10 (to define the potentials to be indexed by KP) :


\hangindent 5ex\hangafter 1
Each potential is defined by a Card 9, defining the radii to be used
and the diagonal Coulomb potential, followed by any number of Cards 10
which cumulatively construct the components of the nuclear potential.
The first input on each card is KP, identifying the potential of which this
is to be a component.
%\bigskip

\subsubsection*{Card 9}
\begin{verbatim}
Card 9: I3,I2,3X,2F8.4,    2F8.4
        KP, 0,   A#1,A#2,  R0C,AC  (i.e. would be Card 10, but TYPE=0)
\end{verbatim}
    All the radii are multiplied by CC,
       which is updated by a TYPE=0 Card 9 to CC = A\#1$^{1/3}$ + A\#2$^{1/3}$
       and by TYPE$>$0 Cards 10, if P0 $>$ 0,  to CC = P0$^{1/3}$

R0C = radius (when multiplied by CC) of charge distribution
\\
AC  = diffuseness of charge distribution
\bigskip

\hangindent 4ex\hangafter 1
The values of A\#1 and A\#2 affect {\em only} the conversion from $r_0$ radii
to R radii in the definition of potentials in Cards 9 \& 10.
If however you adopt the convention that A\#1 is the target mass number
and A\#2 the projectile's, then FRESCO can later display the
fractional $\beta$ values corresponding to given deformation lengths.
(These are for information only, and do not affect the couplings.)
%
%\hangindent 0ex\hangafter 1
\subsubsection*{Card 10}

\begin{verbatim}
Card 10:  I3,I2,  A1,I2,     3F8.4,      3F8.4,       F8.4
          KP,TYPE,IT,SHAPE,  P1,P2,P3,   P4,P5,P6,    P0
                             ---REAL---  -IMAGINARY-  A#
                             V,Vr0,Va,   W,Wr0,Wa,    A#
\end{verbatim}
\vspace*{-10mm}
\begin{tabbing}
\hspace*{5mm} \= \hspace*{1cm} \= \hspace*{2cm} \=
\\
 for\>TYPE\\
\>1\>SHAPE\>Central potential, Volume\\
\>2\>SHAPE\>Central potential, Derivative (i.e. surface)\\
\>3\>SHAPE\>Spin-orbit for projectile\\
\>4\>SHAPE\>Spin-orbit for target\\
\>5\>SHAPE\>Tr tensor force for projectile\\
\>6\>SHAPE\>Tr tensor force for target\\
\>7\>SHAPE\>Tensor force between L and combined projectile+target spins
(e.g.~n-p tensor force)\\
\>8\>SHAPE\>Spin.spin force for target \& projectile spins\\
\>10\>SHAPE\>Deformed projectile   (matrix elements from ROTOR)\\
\>11\>SHAPE\>Deformed target       (matrix elements from ROTOR)\\
\>12\>SHAPE\>Projectile coupled by matrix elements read in.\\
\>13\>SHAPE\>Target     coupled by matrix elements read in.\\
\>30\>SHAPE\>L(L+1) central potential, Volume\\
\>0\> -\>Coulomb potential : see Card 9.
\end{tabbing}

\hangindent 8ex\hangafter 1  Where
      TYPE indicates type of spin tensor or excitation coupling,
\\  IT is `1' or `3' to make this component included only iteratively,\\
 \hspace*{8mm} `2' or `3' to NOT subtract this component in KIND=3,4 single-particle
 couplings,
\\  the SHAPE indicates the radial shape of the form factor,
\\  and P1-P3 \& P4-P6 are parameters for the real and imaginary
parts (respectively).


\medskip

\hangindent 5ex\hangafter 1
If TYPE $<$ 0, add new potential numerically into previous potential,
and use abs(TYPE) instead.   This saves space, and is necessary if adding
together components to be deformed by a following card 10 $\leq$ TYPE $\leq$ 13.

\hangindent 5ex\hangafter 1
If KP $<$ 0, this will be the last Card 10.  Use abs(KP) below,
and omit Card 12.

\bigskip

\hangindent 5ex\hangafter 3
The SHAPEs for volume potentials (TYPE=1, 8 \& 30) are,
\\
with R = P2 * CC, RH = (r - R)/P3, and E = exp(-(r-R)/P3),
\\
(replacing (P1,P2,P3) by (P4,P5,P6) throughout for the imaginary parts)\\
%
   0: Woods-Saxon  = - P1 / (1 + 1/E)
\\ 1: WS squared   = - P1 / (1 + 1/E)$^2$.
\\ 2: Gaussian     = - P1 * exp(-RH$^2$.)
\\ 3: Yukawa       = - P1 * E / r
\\ 4: Exponential  = - P1 * E
\\ 5: Reid soft core for $T$=0, central part
\\ 6: Reid soft core for $T$=1, central part
\\ 7: Read Real
\\ 8: Read Imaginary
\\ 9: Read Complex
\\--1: Fourier-Bessel = $j_0$(RH) = $\sin$(RH)/RH


\hangindent 5ex\hangafter 1
For `SHAPE's 7-9, immediately read in a comment card,
then (free format) NPOINTS, RSTEP, RFIRST,
followed by NPOINTS potential points
for the shape in steps of RSTEP starting at r=RFIRST,
reading card images all from {\em Input File 4} in free format.
\\
Rescale the real part by P1, the imaginary part by P2,
and use radius = P3 * CC for subsequent deformation calculations.
\\
If SHAPE = -7, -8, or -9, rewind file 4 before reading from it.
\bigskip

\hangindent 5ex\hangafter 1
The SHAPEs for surface potentials (TYPE=2), with R, RH and E as before,
are the first derivative forms (normalised to --1 when E=1,
if reasonable):

\hangindent 8ex\hangafter 0
      0 : Woods-Saxon  = - P1 * 4 * E / (1+E)$^2$ \\
      1 : WS squared   = - P1 * 8 * E*E / (1+E)$^3$\\
      2 : Gaussian     = - P1 * 2 * exp(-RH*RH) * RH\\
      3 : Yukawa       = - P1 * E * (1 + r/P3) / (r*r)\\
      4 : Exponential  = - P1 * E\\
      5 : Reid soft core for $T$=0, spin-orbit part $\times r$,\\
      6 : Reid soft core for $T$=1, spin-orbit part $\times r$,\\
      7,8,9 : Read from cards, as above.

\bigskip

\hangindent 5ex\hangafter 1
The SHAPEs for spin-orbit potentials (TYPE=3 or 4) are the following surface
shapes, \\ where CONLS = $\hbar^2/(m_\pi^2 c^2)$ = 2.000.

\hangindent 8ex\hangafter 0
      0 : Woods-Saxon  = -- CONLS * P1 *  E / [(1+E)$^2$ * P3 * r] \\
      1 : WS squared   = -- CONLS * P1 * E*E / (1+E)$^3$  / (2*r*P3) \\
      2 : Gaussian     = -- CONLS * P1 * exp(-RH*RH) * RH  / (2*r*P3) \\
      3 : Yukawa       = -- CONLS * P1 * E * (1 + r/P3) / (4*r$^3$*P3)\\
      4 : Exponential  = -- CONLS * P1 * E / (4*r*P3)\\
      5 : Reid soft core for $T$=0, spin-orbit part\\
      6 : Reid soft core for $T$=1, spin-orbit part\\
      7,8,9 : Read from cards, as above.

To become interaction potentials,
these form factors will be multiplied by
$j(j+1) - l(l+1) - s(s+1)= 2{\bf l}.{\bf s}$, where $s$ = spin of projectile or target
in each of its ground state and excited levels. There is no option here for
state-dependent spin-orbit forces.
\bigskip
%\newpage

\hangindent 5ex\hangafter 1
The SHAPEs for tensor potentials (TYPE=5, 6 \& 7), with R, RH and E as before,
are the second derivative forms (normalised to unity when E=1,
if reasonable):

\hangindent 8ex\hangafter 0
      0 : Woods-Saxon  = P1 * 8 * E*(E-1) / (1+E)$^3$\\
      1 : WS squared   = P1 * 4 * E*E / (1+E)$^3$\\
      2 : Gaussian     = P1 * 2 * exp(-RH*RH) * RH\\
      3 : Yukawa       = P1 * E * (1/P3$^2$. + 2/r$^2$.) / r\\
      4 : Exponential  = P1 * E\\
      5 : Reid soft core, 12 * tensor part\\
      6 : Vol Gaussian = 12 * P1 * exp(-RH*RH)\\
      7,8,9 : Read from cards, as above.

\hangindent 0ex\hangafter 1
\bigskip
{\em Deformations}

\hangindent 5ex\hangafter 1
The SHAPEs for deformed potentials (TYPE $\geq$ 10) are constructed either
by reading external forms (SHAPE = 7, 8 or 9) for each P(k) non-zero
(the input k-form being multiplied by P(k)),
or (for SHAPE $\geq$ 10) by deforming the potential on the
{\em immediately preceding} input card.
This preceding card must be of TYPE=0 for Coulomb deformations,
or TYPE = 1 or 2 for nuclear deformations: deformation of non-central
forces is not yet implemented.
\bigskip

\hangindent 9ex\hangafter 1
For Nuclear deformations P(k) = DEF(k), the deformation lengths (in fm.).
(P0 = DEF(0) for {\em inelastic} monopoles.)

\hangindent 3ex\hangafter 1
For Coulomb deformations P(k) = Mn(Ek), the intrinsic reduced matrix elements
in units of $e.fm^k$  (P0 gives Mn(0) for {\em inelastic} monopoles).
These intrinsic matrix elements Mn(Ek)  use
the reduced matrix elements $\langle I ' || Ek || I \rangle$ given by the
definitions of Alder \& Winther (not by those of Brink \& Satchler),
by assuming a rotational model where $K$ is a good quantum number:
 \beqn
  Mn(Ek) = \langle I ' || Ek || I \rangle
                     / (\sqrt {2I+1}  \langle IK k0 | I ' K \rangle ) \ .
  \eeqn
Thus, for (possibly hypothetical) $0 \rightarrow k$ transitions, $ Mn(Ek)= \langle k || Ek || 0 \rangle$.\\
Mn(E2) is related to the intrinsic quadrupole moment in the rotational model by
$Mn(Ek) = Q _ 0 ~ \sqrt {5 / {16 \pi}}$.
For a uniformly charged sphere of radius $R$, to first order in $\beta_k$ we have
\beqn
  Mn(Ek) = {{3 ~ Z ~ \beta _ k ~ R^k} \over {4 \pi}}  \ .
\eeqn
The Mn(Ek) are thus of the same sign as $\beta$ and $Q_{0}$ for all rotational
bands.
\\
\bigskip
TYPES 12 \& 13, for detailed and model-independent specifications,
directly use the reduced matrix element
\beqn
  M(Ek) & = &i^{ I-I ' + |I-I ' | } ~ \langle I ' || Ek || I \rangle \\
     & =& \pm \sqrt{ (2I+1) ~ B(Ek, I \rightarrow I ' ) } \ ,
\eeqn
where the sign convention has been chosen to make M(Ek) real and symmetric under $I \leftrightarrow I'$ interchange.
The diagonal reduced matrix element gives the spectroscopic quadrupole moment
\beqn
Q_2 = \sqrt {{16 \pi} /5} ~ (2I+1)^{-1/2} ~ \langle II 20 | II \rangle ~\langle I || E2 || I \rangle \ .
\eeqn
The intrinsic and reduced matrix elements are therefore connected by
\beqn
 M(Ek) = Mn(Ek) ~ (-1)^{ [I-I ' +|I-I ' |]/2} ~ \sqrt{2I+1}  \langle IK k0 | I ' K \rangle
\eeqn
  in the rotational model where $K$ is a good quantum number,
 for states of spin $I,I'$, projection $K$, charge $Z$, and radius $R$
(DEF(k) = $\beta_k \times R$ for such states).

\bigskip
So Mn(Ek) is the square root of $B(Ek, 0 \rightarrow k)$ with some sign
within $K=0$ bands. Mn(Ek) always has the same sign as $\beta$.
\\

\hangindent 5ex\hangafter 1
{\em
IN OTHER CASES, DO {\bf NOT} ASSUME THAT ALL TRANSITIONS HAVE THE SAME M(Ek) VALUES.
THUS Mn(Ek) ARE THE SAME FOR ALL PURE ROTOR TRANSITIONS, IN CONTRAST TO
M(Ek).}
 %
In order to guide the user here,
effective $\beta$ values are calculated from the given Mn(Ek) or M(Ek), for
each possible transition.
{\em From version FRV onward, Mn(Ek) is used in place of M(Ek) for TYPE
10 \& 11 potentials.}
\bigskip

\subsubsection*{Card 11}

\hangindent 5ex\hangafter 1
For TYPEs 12--13, Cards 11 are read to determine the details of the
required couplings between each pair of excited states
and of the reorientation terms for each individual state.
\begin{verbatim}
Card 11      4X,3I4,F8.4       IB,IA,k, STR
\end{verbatim}
    to put in a coupling {\em to} state IB {\em from} state IA,
of multipolarity k and strength STR.

For each k used here, P(k) must have been non-zero,
and the required coupling has the same form factor but with relative strength
STR / P(k).   That is, STR is the reduced matrix element $<IB | M(Ek) | IA>$
= M(Ek, IA$\rightarrow$IB)
for Coulomb transitions ({\bf NOT} Mn(Ek)),
and is the `{\em reduced} deformation length'
RDEF(k) for nuclear transitions.

This `reduced deformation length' in the rotational model is
\beqn
   RDEF(k, I \rightarrow I')
         &=& DEF(k) (-1)^{ [I-I'+|I-I'|]/2}  \sqrt{2I+1} ~ \langle IK k0 | I'K\rangle\\
         &=& M(Ek) * 4\pi / [ 3 Z R^{k-1} ]
\eeqn
but may in general be varied independently of DEF(k) and M(Ek).
Both these nuclear and Coulomb interactions are thus designed to give
channel couplings which are {\em in}dependent of
the projection K values read in the Cards 7.
This of course requires making RDEF(k) and M(Ek) dependent on the
spins of the excited states, not on any simple model.

\bigskip
The SHAPEs for deformed potentials (TYPE=10 to 13) are constructed by:

\hangindent 8ex\hangafter 1
SHAPE
%\\
     = 0 to 6 : not used (as card gives deformation strengths).
\\
     = 7,8,9 : Read from cards, as above for TYPE=1.
The multipoles are then multiplied by P(k) (by 1.0 for k=0), and
should end up as scaled to replace the form factors calculated
below:
\\
= 10 : Coulomb multipoles have the usual simple form for a
deformed charged sphere to produce M(Ek) = P(k),
varying at large distances as
\beqn
      F(r) = M(Ek) ~ e^2 ~ {\sqrt{4\pi}   \over  (2k+1)}~  r^{-k-1}
\eeqn
where $e^2=1.440$, and where this form factor will later be multiplied
by the charge number of the opposing nucleus.
\\
= 10 : Nuclear multipoles are the deformation length DEF(k) = P(k)
times the derivative of the undeformed potential times $-1/\sqrt{4\pi}$:
\beqn
      F(r) = - DEF(k) ~  \frac{1}{\sqrt{4\pi}} ~ \frac{dU(r)}{dr}
\eeqn
The shape is the same for all multipoles k $>$ 0.
\\
= 11: Nuclear multipoles calculated by numerically deforming the radii
of the previous potential, and projecting by Gaussian quadrature
onto the required multipoles (those k $>$ 0 for which P(k) is non-zero).
For small deformation lengths, these SHAPEs are the same as for
SHAPE = 10.
\\
= 12: Same as SHAPE = 11, except that the monopole k = 0 part of
the potential is also recalculated by quadrature. A first order
correction is made to the radius to improve volume conservation.
\\
= 13: Same as SHAPE = 12, except that {\bf no} first order
correction is made to the radius to improve volume conservation.

\bigskip
\subsubsection*{Card 12}

\begin{verbatim}
Blank card for end of reading potential parameters.
            (Unless the last KP on card 10 was negative)
\end{verbatim}

\newpage
\subsection{One- and Two-Particle Form Factors}
%
\bigskip


\hangindent 5ex\hangafter 1
Read in a card for each desired particle-nucleus form factor,
then a blank card (13).

\hangindent 5ex\hangafter 1
Each form factor is indexed by a number KN, which may refer to
a single form, or for two-nucleon form factors will refer to a range
of forms KN1 - KN2
(one for each distance and angular momentum between the two nucleons).

\hangindent 5ex\hangafter 1
Each form factor is the binding of one or two particles to a
specified nucleus, where the composite system is another specified
nucleus.

\hangindent 5ex\hangafter 1
To specify the core and composite nuclei, their partition numbers
IC1 and IC2 are required (either order), with $|$IN$|$=1 for projectile
and $|$IN$|$=2 for target nuclei.

\hangindent 5ex\hangafter 1
The mass of the particle is the strict difference of the masses of
the core and composite nuclei, except that if IN $<$ 0, then a
relativistic correction is made for effect of the relative Q-values
on the mass of the composite nucleus, when extracting by differences
the mass of the bound fragment.

\subsubsection*{Card 13}
\begin{verbatim}
 2I3,     3I2,        I2,  1X,A1,  3I2,      F4.1,I2,F4.1,I2,
 KN1,KN2, IC1,IC2,IN, KIND,   CH1, NN,L,LMAX,SN,IA,JN,IB

   2I3,         F8.4, 2I3,     I3,  I3,F8.4,  I3,3F8.4
   KBPOT,KRPOT, BE,   ISC,IPC, NFL, NAM,AMPL, NK,ER,RSMIN,RSMAX,
   
   f8.3,          i4,      F8.3
   RSALPHA,NLAG,PHASE
\end{verbatim}
 %
where

\hangindent 10ex\hangafter 1
KN (=KN1) or the range KN1-KN2 indexes this form factor,

\hangindent 10ex\hangafter 1
IC1, IC2, IN define the particle and core nuclei as explained above,

\hangindent 10ex\hangafter 1
KIND is the kind of couplings:
\\
= 0 for (LN,SN) JN couplings   ~ ~ ~$\leftarrow$ use for typical transfers\\
%= 1, 2 (not used) \\
= 3 for $|(LN,SN)Jn, J_{core}; J_{com}\rangle$    (IA \& IB must be given)\\
= 6 for $|L_{nn}, (\ell,S_{12})j_{12}; J_{12}\rangle$  with isospin $(.5,.5) T$
% = 7 for $|(L_{nn},\ell)L_t, (S_{12},J_{core})S_t ;  J_{com}\rangle$ \& $|(.5,.5)T,T_{core};T_{com}\rangle$  *
= 9 for $(L_{nn},(\ell,S_{12})j_{12})J_{12}, J_{core}; J_{com}\rangle$ \&$(.5,.5)T,T_{core};T_{com}$

\hangindent 8ex\hangafter 1  where
\\  $J_{core}$ = spin of core nucleus (state Ia if given)
\\  $J_{com}$  = spin of composite nucleus (state IB if given)

\bigskip


\hangindent 9ex\hangafter 1
CH1 = single-character identifier to distinguish clusters of
nucleons of different structures that are not further described
but should not be confused with each other.
Use A-M for positive parity clusters, and N-Z for negative parities.

\hangindent 9ex\hangafter 1
NN = number of nodes (include the origin, but not infinity, so NN $>$ 0)

\hangindent 9ex\hangafter 1
L  = LN = angular momentum of bound cluster relative to the core

\hangindent 9ex\hangafter 1
LMAX = maximum value of L in states in deformed potential,

\hangindent 8ex\hangafter 1  SN
      = intrinsic spin of bound nucleon (one-particle states)
\\  = total angular momentum (L+S) of bound custer (KIND  9)
      
\hangindent 9ex\hangafter 1
IA = index (within core partition) of excited state of core,
or zero if to be specified later.

\hangindent 8ex\hangafter 1  JN
     = vector sum LN + SN

\hangindent 9ex\hangafter 1
IB = index (within composite partition) of excited state of composite,
or zero if to be specified later.

\hangindent 9ex\hangafter 1
KBPOT = index KP of potential in which to bind this state

\hangindent 9ex\hangafter 1
KRPOT = index KP$'$ of potential with which to multiply this states'
wave function for transfer interactions. If zero, use KBPOT.
If the binding potential was adjusted for a specific binding
energy, then this adjusted potential (not the original) is used
for transfers.

\hangindent 9ex\hangafter 1
BE = Binding Energy (positive for bound states, negative for
continuum bins)

\hangindent 9ex\hangafter 1
RSMIN,RSMAX = After calculation, radial wave functions are set to zero for radii below RSMIN and/or above RSMAX. No further renormalisations are performed.

\hangindent 9ex\hangafter 1
RSALPHA = After calculation, radial wave functions are rescaled by radial factor $e^{-\alpha  r }$ for $\alpha$ = RSALPHA.  No further renormalisations are performed.

\hangindent 9ex\hangafter 1
NLAG= Use Lagrange mesh to find bound states, with NLAG basis functions.

\hangindent 9ex\hangafter 1
PHASE= Find potential to give phase shift of PHASE degrees at energy $-$BE for bins. Requires NLAG$\gg$0. \\
The revised potential is saved for further bins, just as with ISC$<$0 for bound states.


\bigskip
{\bf BOUND STATES:}


\hangindent 8ex\hangafter 1  ISC
      = 0 to vary the binding energy for fixed potential,
\\  $>$ 0 to vary the TYPE = ISC component of the potential KBPOT
by a scaling factor to give binding energy BE.
\\  $<$ 0 to vary the TYPE = ISC component of the potential KBPOT,
as above, but {\bf also} to permanently rescale all the varied potential
components. This affects all later bound and scattering states using
the potential KBPOT.



\bigskip

\hangindent 0ex\hangafter 1
{\bf CONTINUUM BINS:}


\hangindent 8ex\hangafter 1  ISC   (default value 2)\\
      = -2: no weighting or normalisation,
\\  = 1,2 : weight wave functions by $\exp(-i\delta(k))$,
so they are real, before integrating over bin width.
\\  = 3,4 : weight wave functions by $T(k)$* (useful for resonances)
\\  = -1,1,3 : normalise wave functions to unity (by usual square norm).
{\bf (This option is not recommended, for physics reasons!)}
\\  $\geq$ 10: use additional $k$ factor in the weighting function, with
mod(ISC,10) for above choices. Recommended for low-energy  bins.

\hangindent 4ex\hangafter 1
The values ISC$>$0 give real-valued bins for single-channel states,
but not, in general, for coupled-channels bins. In that case, Fresco stores
{\em all} bound and continuum states as complex functions.
KIND=3 and 4 forward and reverse couplings are both calculated
explicitly, but not KIND=7 transfer couplings. For transfers, the
reverse couplings must be put in explicitly.\\
 %
{\bf Warning:} imaginary parts of bins give imaginary parts of long-range
Coulomb couplings that are {\em ignored} between abs(RMATCH) and RASYM
in the CRCWFN calculations (the values on Card 1a).\\
 %
In general, ISC=2 is recommended (or ISC=4 for resonances), since then
the coupled-channels bins will be nearly real. With ISC=--2 (no phase weighting)
there will be  different results because of different weighting within the
bin. Note that all channels within a bin have the same scalar
weighting factor.



\bigskip

%
\hangindent 0ex\hangafter 1
IPC = print control for further infomation:

\begin{tabbing}
\=IPC ~ ~ \=Print Iterations~ ~ \=Print Final Result ~ ~\=Print W/F\\
 \>0 \>no \>no \>no\\
 \>1 \>no \>yes \>yes\\
 \>2 \>no \>yes \>no\\
 \>3 \>yes \>yes \>yes\\
 \>4 \>yes \>yes \>no\\
\end{tabbing}

For bin states, read this table with `iterations' replaced by `intermediate
phase shifts'.


\hangindent 8ex\hangafter 1  
NFL   $<$ 0    : to write wave-function u(R)/R and potential overlap
V(R)u(R)/R to file number abs(NFL).
\\  $>$ 0 : to read a previously-written wave function from file number
NFL.
\\ These wave function files contain a comment card,
then (free format) NPOINTS, RSTEP, RFIRST,
followed by NPOINTS wf points in steps of RSTEP starting at r=RFIRST,
and then NPOINTS for the vertex function (potential*wf).
The file numbers NFL should be in the range 20--33 (see appendix IV).



\hangindent 5ex\hangafter 1
NAM, AMPL = if IA and IB are both non-zero, then there is enough
information to set up the spectroscopic amplitude now,
to  $\sqrt{NAM} \times AMPL$.
See Card 17, and discussion after Card 15, for further information
about these amplitudes.\\
{\bf Note:} If this is a form factor with mixed core levels,
then AMPL should be specified {\em here} rather than on a Card 17
(as Card 17 requires an IA specification, and such a state would
have multiple IA assignments).

\hangindent 5ex\hangafter 1
If NAM = -1, then use AMPL for the mass of the bound particle,
independent of the MASSes on Cards 6.
(If NAM $\geq$ 0, then the default particle mass is the difference of
the MASSes of the projectiles (IN=1) or targets (IN=2) for partitions
IC1 and IC2).

\hangindent 5ex\hangafter 1
If NAM $<$ -1, then use AMPL instead of ERANGE for the range ER of the
energies of the upper
and lower boundaries of the continuum bins, and use at least
NK=$5\times|NAM|$ integration steps over this range.
(These ER and NK override the input values.)
%.if SYSOUT = PRINT \newpage
%.if SYSOUT = PS \newpage
%\subsubsection{One-particle KINDs}

\subsubsection{One-particle KINDs}
\bigskip

\hangindent 5ex\hangafter 1
For KIND = 0,   (LN,SN) JN couplings.
\\
One form factor with LN,SN, \& JN as read in.
\\
LMAX is not used.
\\
IA \& IB are used only if NAM \& AMPL are non-zero, to specify
spectroscopic amplitudes.
\bigskip

\hangindent 5ex\hangafter 1
For KIND = 3,   sum over coupled core and (ls)j particle states :
\\
Coupled form factors (L$_n$,SN)Jn,  J$_{core}$; J$_{com}$
        (summing over L$_n$,   Jn \& J$_{core}$).
\\
NN is the required number of radial nodes for the component
wave function with core state IA and partial wave L$_n$ = L.
\\
LMAX is maximum L$_n$ in summation.
\\
SN as read in, the intrinsic spin of the bound particle.
\\
All core states are included that can be coupled to form J$_{com}$,
using a deformed binding potential.
\\
{\em Note that such a deformed potential must be TYPE = 11,
whether the projectile or target is deformed: not 10 or 12 or 13}.
\\
JN is the maximum Jn in the summation (single particle $l+s$)
\\
IB gives $J_{com}$ (fixed) : spin of composite nucleus.
\\
BE is the single-particle binding energy for core state IA.

\subsubsection{Two-particle KINDs}
\bigskip

\hangindent 5ex\hangafter 1
For KIND $\geq$ 6,   two-particle bound states are constructed out of sums
of pairs of previously-defined one-particle states,
and input parameters NN through to BE on Card 13 are given new meanings.
\bigskip

\hangindent 18ex\hangafter 1
NN    $\rightarrow$ NPAIRS, the number of pair-products to be summed

\hangindent 18ex\hangafter 1
L     $\rightarrow$ $\ell_{min}$ , minimum orbital angular momentum $\ell$.

\hangindent 18ex\hangafter 1
LMAX  $\rightarrow$ $\ell_{max}$  , maximum orbital angular momentum $\ell$.

\hangindent 18ex\hangafter 1
SN    $\rightarrow$ S$_{min}$  , minimum sum S$_{12}$ of the two nucleons' intrinsic spins
                 (S$_{max}$ = 1.0 always).

\hangindent 18ex\hangafter 1
IA, IB   give core and composite states, as before.

\hangindent 18ex\hangafter 1
JN    $\rightarrow$ J$_{12}$   , total angular momentum of the two-particle state
outside J$_{core}$.

\hangindent 18ex\hangafter 1
KBPOT $\rightarrow$ T     , total Isospin  of the 2-nucleon state (0 or 1)
\\
(used to enforce  $\ell$ + S$_{12}$ + T = odd)

\hangindent 18ex\hangafter 1
KRPOT $\rightarrow$ KNZR, the KN index to a single-particle state $u_{12}(r)$
of KIND 0 or 1, giving the N-N relative motion in the
other participating nucleus (usually in the light ion).
\\
If KNZR $>$ 0, then just the overlap $ u(R)  =  <u_{12}(r) | U(r,R) > $ is
produced, suitable for zero-range two-nucleon transfer calculations.

\hangindent 15ex\hangafter 1
BE    $\rightarrow$ EPS   , the threshold percentage to define components with
square norms sufficiently small to be omitted in the final two-nucleon
state.

\hangindent 8ex\hangafter 1  ISC:
      If ISC $\le$ 0, use Gaussian quadrature grid, in blocks of 6th-order
      positions, for the N-N distance RMIN to RNN ( Card 1).
\\  If ISC $\geq$ 1, use uniform grid for the N-N distances.  Not so accurate.
\\  If ISC $\neq$ 0, print out numerical values of resulting two-nucleon wavefunction
U(r,R).



\hangindent 18ex\hangafter 1
IPC controls the details printed (along with ISC as just above).

\hangindent 8ex\hangafter 1  IPC
      $\geq$ 0 : one-line summary of U(r,R) form factor for each r.
\\  $\geq$ 1 : overall norm and rms radius of total NN state.
\\  $\geq$ 2 : contour plot of the L,$\ell$ components included.
\\  $\geq$ 3 : contour plot of the $U(r,R) \times$ interaction potential



\hangindent 8ex\hangafter 1  NFL
      $<$ 0    : to write two-nucleon wave-function U(r,R) to file
number abs(NFL).
\\  $>$ 0 : to read a previously-written wave function from file number
NFL.  The values of NPAIRS, $l_{min}$, $l_{max}$, $S_{min}$ of
{\em the present}
run are ignored.  Thus NPAIRS can be set to zero.



\hangindent 5ex\hangafter 1
NPAIRS must be less than or equal to the compilation parameter MPAIR.

\bigskip
Data cards 14 are now read in, to give details of the single-particle
wave functions and their amplitudes in the sum of pairs.
%

\begin{verbatim}
Card 14     3(4I3,                         F8.4)
             (NT1(I),NT2(I),NT3(I),NT4(I), COEF(I)), I=1,NPAIRS
\end{verbatim}

\hangindent 0ex\hangafter 1
The sum over I of  COEF(I) $| (l_1,s_1)j_1, (l_2,s_2)j_2 ; J_{12},T \rangle$,
\\
   for  $(l_1,s_1)j_1$  given by state $u_1(r_1)$ of KN1 = NT1(I) $>$ 0,
\\
   and  $(l_2,s_2)j_2$  given by state $u_2(r_2)$ of KN2 = NT2(I) $>$ 0,
\\
is then transformed into the required KIND = 6 format.\\
The $(r_1,r_2)$ coordinates become $(r,R)$ coordinates,
where
$r$ = distance between the two nucleons (angular spin $\ell$), and
$R$ = distance from the core to their centre of mass
     (corresponding angular momentum is L$_{nn}$.

\hangindent 5ex\hangafter 1
If NT3(I) $>$ 0, then the wave functions $u_1(r_1)\times u_2(r_2)$ are further
multiplied by $u_3(r)$ (of KN3 = NT3(I)) before coordinate transformations.
Only the radial shape of KN3 is used, not any angular momentum numbers.
\bigskip

\hangindent 5ex\hangafter 1
If NT1(I) = --1, then an external form factor is read in from Fortran file
number NT2(I), and processed using the subroutine EXTERN1.
At present, this routine is written to read triton wave functions
from the Grenoble Faddeev calculations,
only reading wave functions, {\em not} the potential $\times$
wavefunction (so for e.g. stripping, only use prior interactions).

\hangindent 5ex\hangafter 1
If NT1(I) = --2, then an external form factor is read in from Fortran file
number NT2(I), and processed using the subroutine EXTERN2.
At present, this routine is written to read 3-body wave functions
from HH calculations, reading wave functions, {\em as well as} the
potential $\times$ wavefunction. The 'vrr' file format is assumed,
and the breakup (third) vertex function  is ignored.
\bigskip

\hangindent 5ex\hangafter 1
If NT3(I) $<$ 0, then the I'th component of the pair summation is simply
the product of cluster wave functions $u_1(r)  u_2(R) $
where $u_{2}$ may be KIND = 0 or 1.
\bigskip

\hangindent 5ex\hangafter 1
NT4(I) is not used in this version of FRESCO.
\bigskip

\hangindent 5ex\hangafter 1
For KIND = 6, construct components of the form
$
              L_{nn}, ((\ell, (s_1,s_2)S_{12})j_{12}; J_{12},T\rangle
$
for all different $L_{nn}$, $\ell$, S$_{12}$ \& j$_{12}$ values permitted within the
limits set by $\ell_{min}$, $\ell_{max}$, S$_{min}$, and S$_{max}$
(J$_{12}$ and T are fixed)
by summing over pairs of single-particle wave functions.
\bigskip

For all KINDs, the printout also lists
\bigskip
DZ = derived charge of the bound particle (always positive).
\\
DM = derived mass of the bound particle
\\
K  = the wave number of the bound state asymptotically

\hangindent 5ex\hangafter 1
NORM=overall square norm of this bound state.
The wave functions of the single-particle bound states are always
normalised to unity.

\hangindent 0ex\hangafter 1
RMS= root-mean-square radius of this bound state
\\
$D_{0}$ = zero-range stripping strength for transfers from this stare
\\
$D$  = asymptotic stripping strength, as used e.g. in sub-Coulomb
transfers


\newpage
\subsection{Couplings}
%

\begin{verbatim}
Cards 15-17:
\end{verbatim}

\hangindent 9ex\hangafter 1
Now read in sets of cards for each kind of coupling between partitions
and/or between excited states, ending with a card 18 which has ICTO=0.
Note that if deformed potentials were given as channel optical
potentials then there will already be some couplings between
excited states.

\begin{verbatim}
Card 15
3I4,               3I2,          2F8.2,  2F4.1     2I4
ICTO,ICFROM,KIND,  IP1,IP2,IP3,  P1,P2,  JMAX,RMX, IP4,IP5
\end{verbatim}

\hangindent 5ex\hangafter 1
The coupling is from all the states in partition ICFROM
to all the states in partition ICTO.

\hangindent 9ex\hangafter 1
Couplings in the reverse direction are also included unless
ICTO $<$ 0, except for KINDs 1 \& 2 where finer control is allowed.

\hangindent 5ex\hangafter 1
Couplings are only active for J $<$ JMAX and Radius $<$ RMX,
(if JMAX=0, use JTMAX, and if RMX=0, use  RMATCH from card 1).

\bigskip

\hangindent 0ex\hangafter 1
The couplings can be of 8 KINDs:

\hangindent 8ex\hangafter 0
            = 1 :general spin transfer for projectile/transfer couplings\\
            = 2 :  electromagnetic one-photon couplings\\
            = 3 : single-particle excitation of the projectile\\
            = 4 : single-particle excitation of the target\\
            = 5 : zero-range or LEA transfer with strength P1 and
finite range radius P2 (in fm.)\\
            = 6 : LEA transfer using $D_0$. \& $D$ from bound states\\
            = 7 : finite-range transfer\\
            = 8 : non-orthogonality correction to a KIND 5,6,7 transfer\\
            = 9 : same as KIND=1\\
            =10 : same as KIND=2\\

\hangindent 5ex\hangafter 1
After cards 15 for KINDs 2,3,4,5,6,7 (\& 8 if IP2$>$0),
spectroscopic amplitudes are read in by means of cards 17.
These coupling types are those for which ICTO \& ICFROM are
different, one being a core partition and the other a composite
nucleus. One table is used for all the amplitudes: it is indexed
(besides the partition numbers) by IN,IB,IA \& KN (see glossary),
and stores a real number A.
The table is filled cumulatively,
starting perhaps with cards 13,
so amplitudes need only be
entered after the card for the first KIND of coupling in which
they appear.
(So if all the amplitudes are already entered for a certain KIND
of coupling, then the following card 17 will be blank, to indicate
no more cards 17 are to be read).

\bigskip

\hangindent 12ex\hangafter 1
KIND = 1 :
\underline{General Spin Transfer}
(with local or non-local external form factors on FILE 4)
See Appendix V for more details.

\hangindent 8ex\hangafter 1  IP1
~~~  = 0 : local form factor
\\   1 : non-local (two-dimensional) form factor



\hangindent 7ex\hangafter 1  IP2
~~~  = 0 : read in real values      (only if IP3 $\geq$ 0)
\\   1 : read in imaginary values    (only if IP3 $\geq$ 0)
\\   2 : read in complex values        (only if IP3 $\geq$ 0)



\hangindent 8ex\hangafter 1  IP3
~~~  =3 : read data from JLM folding program jlmP, for projectile couplings only
\\  2 : read data from JLM folding program jlmP, for target couplings only
\\   1 : read data from charge exchange program CHEX2, with appropriate scaling.
\\   0 : no jlmP or CHEX2 scaling factors, only P1, P2 and FSCALE
\\  -1 : write out typical non-local grid $(R,R')$ coordinates
\\  -2 : calculate non-local grid $(R,R')$ coordinates,
and call subroutine FFNL to calculate form factor.
\\  -3 : calculate non-local grid $(R,R')$ coordinates,
call subroutine FNLSET to calculate an initial form factor, and the call
FNLCC for each pair of coupled partial waves, for L-dependent factors.



\hangindent 10ex\hangafter 1
 P1,P2 : scaling for the real and imaginary parts (respectively).\\

   If IP3 $\geq$  0, read FROM FILE 4 the following cards
   (until a blank card):
\begin{verbatim}
     Card 16.6: I4, 3F8.4,          I4,  2F4.0,   2I4,   A35
                NP ,HNP,RFS,FSCALE, LTR, PTR,TTR, IB,IA, COMMENT
            for NP = number of radial points
                HNP = step size
                RFS = radius of first point
                FSCALE = scaling factor to be applied
                LTR = L-transfer
                PTR = projectile spin transfer
                TTR = target spin transfer
                IB  = excited-state pair fed by coupling
                IA  = excited-state pair feeding the coupling
                COMMENT = comment for display in printout.
\end{verbatim}
    ~~~~~~ then free format, repeatedly until sufficient data is read.\\


\hangindent 8ex\hangafter 1  If IP1 =
      0, read local form factor from i=1 to N
\\  1, read successively for j=1,NLO the non-local form factors
 FNL(i,j), i=1 to NP   [NLO = RNL / max(HNL,HCM)]


When IP3=1 for CHEX2 input, the RFS and its F8.4 is omitted from Card 16.6,
with default value RFS=HNP.
\bigskip


\hangindent 12ex\hangafter 1
KIND = 2 :
\underline{Electromagnetic one-photon couplings} (for $E\lambda$ and
$M\lambda$ processes).


\hangindent 5ex\hangafter 1
 ICTO is the gamma partition and ICFROM the particle partition.
 The photon must be on the `projectile' side, and the bound state between
 target states.

\hangindent 15ex\hangafter 1
 IP1 = $\lambda$ : The multipolarity of the radiation.
 If $\lambda >$ 0, include all multipoles 1,...,$\lambda$
permitted by parity, whereas if IP1 $<$ 0, include only the multipole
$\lambda$ = abs(IP1).
%

\hangindent 8ex\hangafter 1  IP2 
~ ~   = 0 : Calculates both electric and magnetic convection couplings
%(reads $g$-factors from P1\&P2)
\\    1 : for electric only
\\    2 : for magnetic only %(reads $g$-factors from P1\&P2)



\hangindent 8ex\hangafter 1  IP3:  {\em not implemented yet}
%~ ~   = 0 : no intrinsic convection couplings
%\\    1 : intrinsic convection coupling for projectile only
%\\    2 : intrinsic convection coupling for target only
%\\    3 : intrinsic convection coupling for projectile and target
%\\    4 : turn on magnetization term for electric component


% P1  = projectile $g$-factor  \\
% P2  = target $g$-factor  \\

\bigskip

\hangindent 15ex\hangafter 1
KIND = 3 or 4 :
\underline{Single-particle excitations of the projectile (3) or target (4)}


\hangindent 5ex\hangafter 1
 For these KINDs, ICTO is the partition of the nuclei being excited,
and ICFROM is used to indicate the
{\em core} partition if the single particle were removed.  No couplings are
generated to or from the ICFROM partition, only within ICTO partition.

\hangindent 15ex\hangafter 1
 IP1 = Q : The multipole order of the deforming potential due to
the colliding nucleus. If Q $>$ 0, include all multipoles 0,1,...,Q
permitted by parity, whereas if IP1 $<$ 0, include only the multipole
Q = abs(IP1).

\hangindent 8ex\hangafter 1
IP2 
\\   = 0 : Coulomb \& nuclear (complex)
\\  = 1 : nuclear (complex) only
\\  = 2 : Coulomb only



\hangindent 8ex\hangafter 1  IP3
\\  = 0 or 10 : include all re-orientation terms
\\  = 1 or 11 : no re-orientation terms for $Q > 0$
\\  = 2 or 12 : ONLY re-orientation terms
\\  = 3 or 13 : include only couplings to and from the ground state,
    but NOT gs reorientation,
\\  = 4 or 14 : include diagonal couplings,
                and  couplings to and from the ground state
\\  = 5 or 15 : include diagonal couplings,
                and  couplings to and from any bound state
\\  $\geq$ 10      : read cards 16.5,
 % \\
repeating FORMAT(6E12.4) as required, for
complex factors \\QSCALE(Q), Q=max(0,-IP1),abs(IP1)
to scale the folded form factors for multipoles Q.

IP4 = $Q_{\rm max}$, the max deformed core potential multipole\\
IP5 = $\Lambda_{\rm max}$, the new multipole order for formfactor reduction
(see Appendix III).

 P1  = FLOAT( potential KP index for fragment - target interaction) \\
 P2  = FLOAT( potential KP index for core     - target interaction) \\
        (only SCALAR parts of the potentials P1 and P2 are used).
\bigskip

KIND = 5 or 6 :
\underline{Zero-range \& LEA transfers for projectiles}

IP1 and IP2 not used.

\hangindent 8ex\hangafter 1  For KIND=5,
\\  P1 = $D_0$    - ZR coupling constant
\\  P2 = FNRNG - Effective finite-range parameter for use
in LEA (in units of fm.)



\hangindent 14ex\hangafter 1
For KIND=6, use $D_0$ and FNRNG = $\sqrt{(D/D_0 - 1)/k^2}$. from the projectile
bound states.  With unbound states, or if IP3 = 1,
use $D_0$ from state, but FNRNG = P2
from input card here.

\hangindent 5ex\hangafter 1
 In both KINDs, read in cards 17, noting that for KIND = 5
spectroscopic factors for the projectile are not needed, and are ignored.

\hangindent 5ex\hangafter 1
Users of these interactions kinds should also pay attention to the
PARAMETER INH as described in Appendix I.
\bigskip

KIND = 7 :
\underline{Finite-range transfers.}


\hangindent 8ex\hangafter 1  IP1 =
~~~  0,-2 : POST interaction
\\  1,-1 : PRIOR interaction
 (N.B. meaning of IP1 thus depends on ICTO \& ICFROM!!!)
\\  $\le$ -1 : Use $\theta$ quadrature from $\theta = \pi$ down to $\theta = 0$.
 (Useful for finite-range knock-on with light projectiles)
\\  $\le$ -3 : `VCORE' interaction:
Use ONLY the core-core interaction potential
 (Useful for finite-range knock-on with light projectiles)



\hangindent 8ex\hangafter 1  IP2 =
~~~  0 : no remnant
\\  1 : full real remnant
\\  -1 : full complex remnant
\\  2 : ``non-orthogonality remnant" - this works by inserting
a KIND = 8 coupling card after this coupling, before any subsequent
couplings.
This is does not affect the one-step amplitudes, and is only useful
if another transfer step follows this coupling.



\hangindent 10ex\hangafter 1
 IP3 = KPCORE : the number KP of the potential to use between
the two cores, in the remnant part of the interaction potential.

\hangindent 10ex\hangafter 1
       If IP3=0, use as KPCORE the optical potential given for
the first pair of excited states in the partition of projectile core.
(this uses the observation that optical potentials tend to depend
more on the projectile than the target, and must clearly be re-examined
if the projectile is heavier than the target.)

\hangindent 10ex\hangafter 1
 P1,P2 : not used.
\bigskip


\hangindent 12ex\hangafter 1
KIND = 8 :
\underline{Non-orthogonality supplement}
appropriate to a previous
KIND 5,6 or 7 interaction.


\hangindent 8ex\hangafter 1  IP1 =
~~~  0 : post
\\  1 : prior
 (N.B. IP1 should be the same as the previous interaction!)



\hangindent 8ex\hangafter 1  IP2
\\  $>$ 0 : read in spectroscopic factors as cards 17
 (Only useful if you did NOT have a previous KIND 5,6,7
interaction, which would have needed the amplitudes then.)
\\  = 0 : no cards 17 to read.


Note that you should {\em either} use KIND=7, $|$IP2$|$=2,
{\em or} use  KIND=7, $|$IP2$|$=1  and a KIND=8.
If you have KIND=7, $|$IP2$|$=2 and a KIND=8 card, then this is
double counting.
\bigskip




%~

\hangindent 15ex\hangafter 1
\begin{verbatim}
Cards 17:
\end{verbatim}
  {\em Spectroscopic amplitudes} {\rm for the overlaps between
partitions ICTO \& ICFROM already defined by a card 15}.

\begin{verbatim}
Read until IN = 0 or IN < 0:
 4X,4I4,         F8.4
    IN,IB,IA,KN, A
\end{verbatim}

\hangindent 5ex\hangafter 1
meaning that the overlap of the composite nucleus in excitation
state IB with the core nucleus in excitation state IA is the
bound-state form factor KN with amplitude A. The IN=1 specifies
projectile overlap, and IN=2 target overlap.

\hangindent 5ex\hangafter 1
 If the form factor KN mixes different IA levels, then the spectroscopic
 amplitudes should rather be specified with a Card 13.

\hangindent 5ex\hangafter 1
N.B. The amplitudes A are signed, and are NOT the spectroscopic
factors, but will typically be the square roots of these factors.
For transfers out of or into closed shells of N antisymetrised nucleons,
the spectroscopic factors will usually contain factors of N,
so the spectroscopic amplitudes needed by FRESCO will typically
need to already contain factors of $\sqrt{N}$.
\\
The sign of A should be consistent with the spin coupling order used in
the program, which is
\beqn
  (\ell,s)j, J_{core}; J_{com}
\eeqn
for binding a $\ell sj$ nucleon onto a core of spin $J_{core}$ to form $J_{com}$.
\\
 If IN$<$0, use abs(IN) in this the last card 17\\
 If IN=0, no more cards 17 to be read.
% %.if SYSOUT = PRINT \newpage

\subsection{Incoming Channel, Energy}

\begin{verbatim}
Card 18: (has the first 4 columns blank or zero : ie card 15 with ICTO=0)

 3I4,      3I2
 0,PEL,EXL,LAB,LIN,LEX
\end{verbatim}

\hangindent 10ex\hangafter 1
Incoming plane waves are present in partition PEL with excitation
pair EXL.  The energies on card 19 are the laboratory energies for
partition LAB's nucleus LIN (1 or 2 for projectile or target)
in excitation pair LEX.

\hangindent 10ex\hangafter 1
The defaults for PEL,EXL,LIN \& LEX are all 1, and the default
for LAB is PEL, so this card can be completely blank usually.
\bigskip

\begin{verbatim}
Card 19 :  Projectile Energies
3(F8.4,I8),F8.4) :  (ELAB(i),NLAB(i),i=1,3),ELAB(4)
\end{verbatim}
A series of runs at different
(real)
energies until ELAB(i) = 0 is found.
 If NLAB(i) $>$ 1, then the range from ELAB(i) to ELAB(i+1) is covered
in NLAB(i) linear intervals.

\subsection{Additional Cards on Fortran input file 4:}
%
\bigskip
All these cards involve reading real or complex floating point numbers
by repeating
\begin{verbatim}
  FORMAT(6E12.4)
\end{verbatim}
Cards A1 :  External Potential Form Factors (SHAPEs 7-9 on Cards 10)
\bigskip

\hangindent 12ex\hangafter 1
Cards A2 :  External Coupling Potentials for KIND 1 \& 2 couplings
(NO $<$ 0 on Cards 16)

\newpage
\appendix
\section*{Appendix I : FRESCO Namelist input in Fortran 90 version}
%
\bigskip
This Fortran 90 version does dynamic allocation of arrays of the
required sizes, and therefore does not need the PARAMETER declarations
which were a feature of previous FRESCO input specfications. The
FRESCO executable can be now completely compiled in advance,
and does not require other files (such as, previously, frxp0.f and
libfrxp.a) to run.

\bigskip
The f90 version reads card sequences as above, but can also read NAMELIST
input, which is more flexible in allowable number formats.

After the usual Card 0 with title information, the line
\begin{verbatim}
Card 1:
NAMELIST
\end{verbatim}
indicates namelist input is to be assumed.
A namelist group starts with a \&NAME and ends with a /. Upper and lower
case names are treated as equivalent. For a sample input, see Appendix II.

If Card 1 begins with 'CDCC', then CDCC input is assumed, see Appendix III.

\bigskip
The first namelist read is \&FRESCO ... / :

\noindent{\\ \bf \&FRESCO namelist}

This contains all the variables in Cards 1,1a, 2, 3, 4, 5 and  18 and 19:\\
{\tt
hcm, rmatch, rintp, hnl, rnl, centre, rasym, accrcy, switch, ajswtch,  \\
jtmin, jtmax, absend, dry, rela, nearfa, \\
kqmax, pp, thmin, thmax, thinc, koords, cutl, cutr, cutc, \\
ips, it0, iter, iblock, pade, iso, nnu, maxl, minl, mtmin, epc,  \\
inh, plane, smallchan, smallcoup, \\
chans, listcc, treneg, cdetr, smats, xstabl, nlpl, waves,  lampl, kfus, wdisk, \\
pel, exl, lab, lin, lex, elab, nlab
}\\
along with new variables \\
{\tt
fatal, nosol, pset, jset, inh, psiren, tmp, masfil, unitmass, finec}.

The new variables are:


\hangindent 5ex\hangafter 1
{\tt fatal} = T/F, if to stop or not when CRC iterations do not converge
        (previously ITER$>$0).

\hangindent 5ex\hangafter 1
{\tt nosol} = T/F, if not to solve the CRC equations, only construct couplings
        (previously |ITER|$<$IT0).

\hangindent 5ex\hangafter 1
{\tt pset} = -1 or +1, to restrict parity of total CRC set (0 = no restriction)
        (previously from JSETS).

\hangindent 5ex\hangafter 1
{\tt jset} = number of CRC sets to calculate before stopping (0 = all sets)
        (previously from JSETS).

\hangindent 5ex\hangafter 1
{\tt inh} = 0 : zero-range transfer forms in intervals of HCM exactly
        (previously a PARAMETER).\\
    = 1 : stored in steps of HCM * (proj. core)/(proj. composite mass)\\
    = 2 : stored in steps of HCM * (targ. core)/(targ. composite mass)\\
    So {\tt inh}=2 corrects for longitudinal recoil during transfers
with zero-range projectiles.

\hangindent 5ex\hangafter 1
{\tt psiren} = T/F, to do simple renormalisation of channel wfs after Pade acceleration

\hangindent 5ex\hangafter 1
{\tt tmp} = name of directory for temporary files: `/tmp' or `.'

\hangindent 5ex\hangafter 1
{\tt masfil} = name of file of masses of isotopes (used to find masses and charges
from the nuclide name, if read-in MASSes are zero).

\hangindent 5ex\hangafter 1
{\tt unitmass} = unit (in amu) for MASS values read in.
Default = 1.000

\hangindent 5ex\hangafter 1
{\tt finec} = fine-structure constant: used to determine electrostatic $e^2$.
Default = 137.03599d0.

\hangindent 5ex\hangafter 1
{\tt jump,jbord} = namelist input jump(1) is stored in jump(2) etc for the program.


\noindent{\\ \bf \&PARTITION namelist}

This has the information on Card 6, but the names have been made unique:
\begin{verbatim}
namep    massp    zp      nex   pwf    namet    masst    zt      qval
(char*8   real    real    int   logical char*8  real     real    real)
\end{verbatim}

\noindent{\\ \bf \&STATES namelist}
Card 7 information: repeat NEX times before the next {\tt partition} namelist:
\begin{verbatim}
Jp,  COPYp,BANDp ,Ep, KKp,Tp, CPOT,PWF,
-------projectile state-----

Jt,  COPYt,BANDt ,Et, KKt,Tt,  EXTRA,SPIFL,IGNORE
-----target state-----------
\end{verbatim}
Here, PTYp and PTYt are synonyms for BANDp and BANDt, respectively.\\
KP has been renamed CPOT, and the band-head projections are KK*.

\bigskip
End reading partitions with an empty namelist {\tt \&partition /} (Card 8)


\noindent{\\ \bf \&POT namelist}
This has the information on Cards 9 and 10.
You can either input individual {\tt \&pot} namelists corresponding to each
separate Card 9 and Card 10, or you can combine TYPEs 0,1,2,3, 10 and 11 potentials
into a larger namelist.

\subparagraph*{Individual namelists:}

Variables {\tt kp, type, shape, itt, p(1:7)}, where {\tt itt} is a logical variable.
The {\tt p(1:7)} is synonymous to the set {\tt p1, p2, p3, p4, p5, p6, P7},
and with the array {\tt def(:)}.

The equivalent to Card 9 is obtained if {\tt shape}=0,
then  A\#1={\tt p1}, A\#2={\tt p2},  R0C={\tt p3}, and AC={\tt p4}.


\subparagraph*{Combined namelists:}
All the following variables may be set independently in the {\tt \&pot} namelist
\begin{verbatim}
kp    shape itt
ap    at    rc    ac
v     rv    av    w    rw   aw
wd    wdr   wda   vd   vdr  vda
vso   rso   aso   vsoi rsoi asoi
defp(:)  deft(:)  mnet(:)  mnep(:)=0
\end{verbatim}
so  A\#1={\tt at},A\#2={\tt ap},  R0C={\tt rc}, and AC={\tt ac} etc.
All the potentials will have the same {\tt shape}.\\
The {\tt mnep(k)} and {\tt mnet(k)} are the TYPE=11 and 12 Coulomb reduced matrix elements,
and
the {\tt defp(k)} and {\tt deft(k)} are the TYPE=11 and 12 nuclear deformation lengths.
matrix elements, for multipolarity {\tt k}.\\
This combined namelist is a pre-processing option only: the FRESCO output
will be the same as from individual namelists.

\bigskip
More complicated potentials  may be formed as usual by having multiple
{\tt \&pot} namelists, but keep only {\bf one} Coulomb potential by having
{\tt rc} non-zero in only {\bf one} of them.

\bigskip
End reading potentials with an empty namelist {\tt \&pot /} (Card 12),
or by having {\tt kp}$<$0 in the last namelist.

\subparagraph*{\&STEP namelist}

The information on Card 11 is give by a {\tt \&step} namelist with variables
{\tt IB,IA,k, STR}.

\noindent{\\ \bf \&OVERLAP namelist}
This namelist has the standard variables of Card 13
{\tt    kn1, kn2, ic1, ic2, in, kind, ch1, nn, l, lmax, sn,
        ia, j, ib, kbpot, krpot, be, isc, ipc, nfl, nam, ampl}
with the same names.\\

\bigskip
Extra variables {\tt keep, dm,nk,er,e} are also defined:

\hangindent 5ex\hangafter 1
{\tt keep} = T to {\bf not} reset all the variables to zero before reading next namelist.

\hangindent 5ex\hangafter 1
{\tt dm} = mass of bound particle, if to override default difference.

\hangindent 5ex\hangafter 1
{\tt nk} = number of integration steps to form continuum bin.

\hangindent 5ex\hangafter 1
{\tt er} = range (width) of continuum bin. 
If er $>$ 0, then {\bf ratio} of the upper and lower bin
energies; and if er $<$ 0, then er is the  {\bf difference} of the energies in MeV.

\hangindent 5ex\hangafter 1
{\tt e} : If {\tt e}$\ne$0, then replace {\tt be $\leftarrow$ - e}.\\
        So {\tt be} = binding energy, and {\tt e} = eigenenergy.

\hangindent 0ex\hangafter 1

\noindent{\\ \bf \&COUPLING namelist}

This gives the information previously on Card 15.
\begin{verbatim}
icto,icfrom,kind,ip1,ip2,ip3,p1,p2,jmax,rmax,kfrag,kcore
int  int    int  int int int 4*real          int   int
\end{verbatim}
If({\tt kfrag}$>$0) {\tt p1} $\leftarrow$ {\tt kfrag}, and
if({\tt kcore}$>$0) {\tt p2} $\leftarrow$ {\tt kcore}.

\subparagraph*{\&INEL namelist}

The information on Card 16 is give by a {\tt \&inel} namelist with variables
{\tt  ib,ia,k,no,kp,a}.


\subparagraph*{\&CFP namelist}

The information on Card 17 is give by a {\tt \&cfp} namelist with variables
{\tt in,ib,ia,kn,a,keep}.

\subparagraph*{\&SCALE namelist}

The information for Coupling KINDS 3 and 4  is given by a {\tt \&scale} namelist with
variables {\tt QSCALE(0:11)}.

% \newpage
\bigskip
\subsection*{Mass and charge constants}
To exactly reproduce previous FRESCO runs using $2*amu/\hbar^2$ = 0.0481960,
you will need to enter to this version the numbers:
{\tt
        unitmass=1.007335d0; finec=137.5648d0.
}
To exactly reproduce using $2*amu/\hbar^2$ = 0.0478326, you need input
{\tt
        unitmass=0.999740d0; finec=137.0455d0,
}
but only the default input of
{\tt
        unitmass=1d0; finec=137.03599d0
}
is completely accurate.


\bigskip
\subsection*{Glossary of Indexing Variables}
%

\hangindent 10ex\hangafter 1
IC,IC1,IC2    index to mass partitions

\hangindent 10ex\hangafter 1
IA,IB         index to projectile-target pairs of excited states
              within each partition

\hangindent 10ex\hangafter 1
KP            index to table of potentials, for scattering, binding,
              and folding

\hangindent 10ex\hangafter 1
KN            index to single-particle bound states

\hangindent 10ex\hangafter 1
KN1,KN2       two-particle bound states are given for a range (RMIN to
              RNN) of internucleon separations, and are stored in the
              MSP table from KN1 to KN2 inclusive.

\hangindent 10ex\hangafter 1
CP            index to the coupling set up

\hangindent 0ex\hangafter 1

\newpage
\section*{Appendix II : A Test Run (a CCBA calculation)}

\hangindent 0ex\hangafter 1


The following test case calculates the inelastic and transfer cross sections
for the reactions induced by $^{19}$F on $^{28}$Si at 60.0 MeV.   A Triton-Oxygen
cluster structure is assumed for $^{19}$F, with the potentials for the triton
and the oxygen core being folded together to give a diagonal Watanabe
potential for elastic $^{19}$F scattering as well as the couplings to the
inelastic 1/2+ state.
These elastic and inelastic channels are coupled together to all orders,
then a finite-range transfer step is calculated for the �$^{16}$O + �$^{31}$P channel,
thus performing a CCBA calculation.

The Card types and line numbers added here only for convenience:
\small
\begin{verbatim}
0   04 Test run: CCBA calculation 28Si(19F,16O) 31P, cluster form factors.
1   05 0.10     25.0   0.50    0.100     3.
2   06   0. 80.   .01   F F
3   07 1 0.0   60.0    2.5
4   08 0.00       1 2 0  30         0 0.0
5   09  1 2 0 0 1 0 0 0 0 0
6   10 19-F     19.0   9.0        2  28-SI   28.0    14.0    0.0
7   11 0.5   +1 0.0       0          0.0   +1 0.0
7   12 2.5   +1 0.2
6   13 16-O     16.0   8.0        1  31-P    31.0    15.0    6.199
7   14 0.0   +1 0.0       0          0.5   +1 0.0
8   15
9   16   1 0  0 28.0     19.0   1.2
9   17   2 0  0 31.0     16.0   1.35
10  18   2 1  0 31.20  1.45    0.47    15.10   1.270   0.310
10  19   2 3  0  0.75  1.24    0.37
9   20   3 0  0 19.0   0.0      1.25    0.65
10  21   3 1  0 115.00 1.250   0.650
10  22   3 3  0  6.30  1.25    0.650
9   23   4 0  0 31.0   0.0      1.25    0.65
10  24   4 1  0 99.000 1.250   0.650
10  25   4 3  0  6.30  1.25    0.650
9   26   5 0  0 28.0      0.0   1.2
10  27   5 1  0 27.00  2.42452 0.460   11.66   2.14456 0.238
9   28   6 0  0 28.0   0.0      1.25    0.65
10  29   6 1  0 160.00 1.07    0.72    37.0    1.35    0.88
10  30   6 3  0  10.0  0.85    0.850
12  31 0
13  32   1    1 2 1 0   4 0   0.5   0.5    3  0  11.73   1  0  0
13  33   2    1 2 2 0   4 0   0.5   0.5    4  0  17.899  1  0  0
13  34   3    1 2 1 0   3 2   0.5   2.5    3  0  11.53   1  0  0
13  35  0
15  36    2   1   7 0 0 0
16  37        1   1   1   1  1.00
16  38        1   2   1   3  1.00
16  39       -2   1   1   2  1.00
15  40    1   2   3 4 1 0  6.0     5.0
16  41
18  42    0   1   1
19  43 60.0
\end{verbatim}

\newpage
A namelist version of the above input is:
\begin{verbatim}
Test run: CCBA calculation 28Si(19F,16O) 31P, cluster form factors.
NAMELIST
 &FRESCO
 hcm=.10 rmatch=25 rintp=0.50  hnl=0.100  rnl=3.
 jtmin=0. jtmax=80 absend=.01 kqmax=1  thmin=0. thmax=60.0  thinc=2.5
 iter=1   iblock=2 nnu=30 chans=1  listcc=2 smats=1 elab=60.0
 /
 &PARTITION
   namep='19-F' massp=19. zp=9 nex=2 namet='28-SI' masst=28 zt=14 /
 &states Jp=0.5   Bandp=+1 Ep=0.0  cpot=1          Jt=0.0   bandt=+1 Et=0.0 /
 &states Jp=2.5   Bandp=+1 Ep=0.2 /

 &PARTITION
   namep='16-O' massp=16. zp=8 nex=1  namet='31-P' masst=31 zt=15 qval=6.199/
 &states Jp=0.0   Bandp=+1 Ep=0.0  cpot=2          Jt=0.5   bandt=+1 Et=0.0 /
 &partition /

 &pot kp=1 At=28.0     Ap=19.0   Rc=1.2 /
 &pot kp=2 At=31.0     Ap=16.0   Rc=1.35
           V=31.20  rv=1.45    av=0.47    W=15.10   rw=1.270   aw=0.310
           Vso=0.75 rso=1.24    aso=0.37 /
 &pot kp=3 At=19.0     Rc=1.25    ac=0.65
           V=115.00 rv=1.250   av=0.650 Vso=6.30 rso=1.25    aso=0.650 /
 &pot kp=4 At=31.0     Rc=1.25    ac=0.65 V=99.000 rv=1.250   av=0.650
           Vso=6.30 rso=1.25    aso=0.650 /
 &pot kp=5 At=28.0     Rc=1.2
           V=27.00  rv=2.42452 av=0.460   W=11.66   rw=2.14456 aw=0.238 /
 &pot kp=6 At=28.0     Rc=1.25    ac=0.65
           V=160.00 rv=1.07    av=0.72    W=37.0    rw=1.35    aw=0.88
           Vso=10.0 rso=0.85    aso=0.850 /
 &pot  /

 &OVERLAP kn1=1 ic1=1 ic2=2 in=1 kind=0 nn=4 l=0 sn=.5 j=0.5 kbpot=3 be=11.73 isc=1 /
 &OVERLAP kn1=2 ic1=1 ic2=2 in=2 kind=0 nn=4 l=0 sn=.5 j=0.5 kbpot=4 be=17.899 isc=1 /
 &OVERLAP kn1=3 ic1=1 ic2=2 in=1 kind=0 nn=3 l=2 sn=.5 j=2.5 kbpot=3 be=11.53 isc=1 /
 &overlap /

 &COUPLING icto=2   icfrom=1   kind=7 ip1=0 ip2=0 ip3=0/
 &cfp     in=1   ib=1   ia=1   kn=1  a=1.00 /
 &cfp     in=1   ib=2   ia=1   kn=3  a=1.00 /
 &cfp     in=2   ib=1   ia=1   kn=2  a=1.00 /
 &cfp     /

 &COUPLING icto=1   icfrom=2   kind=3 ip1=4 ip2=1 ip3=0 kfrag=6 kcore=5/
\end{verbatim}

\newpage
\section*{Appendix III : CDCC Namelist input in Fortran 90 version}

{\bf \&CDCC namelist}

{\tt
hcm, rmatch, rintp, hnl, rnl, centre,  rsp, iter, pset, llmax,  dry,
rasym, accrcy, switch, ajswtch, sinjmax,  cutl, cutr, cutc,
absend, jtmin, jump, jbord, nnu,  rmatr, nrbases, nrbmin,  pralpha, pcon,  meigs,
listcc, smats, veff, chans, xstabl, thmin, thmax, thinc,
smallchan, smallcoup, melfil, nosol, cdetr, numnode, treneg,
nlpl, trans, pel, exl, cdccc, qscale, pade, kfus,
elab,  lab, lin, lex, \\

 hktarg, ncoul, reor, pauli,  nk,
q, ipc, iscgs, ipcgs,  hat, remnant, postprior, quasi,
sumform, qc, la, static, expand, maxcoup
}\\

\vspace{-3mm}

\hangindent 5ex\hangafter 1
are all the same as the {\bf \&FRESCO} namelist, except for:\\
{\tt cdccc} which is an alias for {\tt cdcc} of Card 5,\\
 %
{\tt q} = {\tt ip1}, multiple for the projectile single-particle couplings,\\
{\tt ncoul} = {\tt ip2}, selecting nuclear and/or coulomb,\\
{\tt reor} = {\tt ip3}, selecting diagonal and/or off-diagonal couplings,\\
{\tt qc} = {\tt ip4}, $Q_{\rm max}$ for the deformed core potential multipoles,\\
{\tt la} = {\tt ip5}, $\Lambda_{\rm max}$ for the new multipole orders with formfactor reduction,\\
{\tt hat} (logical, default T) to use mean bin energies (otherwise midpoint),\\
{\tt quasi}: if assigned, set all channel energies as if for this projectile
energy (eg -{\tt be} for adiabatic),\\
{\tt iscgs,ipcgs} = {\tt isc,ipc} (card 13) for the projectile ground state wave
function,\\
{\tt elab} = {\tt ELAB(1)}, just the one projectile lab energy,\\
{\tt hktarg} = target value of $h.K$, where $h$={\tt HCM} and $K$ is the elastic wave number.
(default {\tt hktarg}=0.2). \\
\hspace*{1cm} If {\tt HCM}=0 on Card 1,
then $h$ is calculated from {\tt elab} and {\tt hktarg}.\\



\hangindent 8ex\hangafter 1
{\tt sumform} determines the bin formfactor reductions
(default 2 if there are any {\tt kind}=3 bin states, else 0):\\
~ ~   = 0 : no formfactor reduction:
     $\langle {\tt KN} | K Q \lambda | {\tt KNP} \rangle$
\\    = 1 : $KQ\lambda$ formfactors summed into new multipole $\Lambda$:
     $\langle {\tt KN} | \Lambda | {\tt KNP} \rangle$
\\    = 2 : formfactors summed into composite projectile state:
     $\langle {\tt IB} | {\Lambda} | {\tt IBP} \rangle$.
      This is not allowed if have spin-orbit or transfer couplings, and
      {\tt sumform}=1 will be set in these cases.\\

If have no cc bins, then $K$ multipoles truncated using {\tt ip1=q}.\\
If have cc bins and $Q_{\rm max}${\tt =qc=ip4} and
$\Lambda_{\rm max}${\tt =la=ip5} are unset, then
{\tt q} sets the maximum multipole order of $\Lambda$, and
 all possible values of  $KQ\lambda$ are used that couple to $\Lambda \le {\tt q}$\\
If {\tt qc} and {\tt la} are both set, then control each multipole
individually:\\
\hspace*{1cm} {\tt q}     = $K_{\rm max}$,  {\tt ip1} set in usual way\\
\hspace*{1cm} {\tt ip4=qc}  = $Q_{\rm max}$. \\
\hspace*{1cm} {\tt ip5=la} = $\Lambda_{\rm max}$ maximum order for new
multipole.\\
\hspace*{1cm}  Note: $\lambda$ will always run from $0\rightarrow Q$.\\


\hangindent 8ex\hangafter 1
{\tt trans} determines the number of transfer partitions:\\
~ ~   = 0 : no transfers: no E(jectile) or R(esidue)
\\    $<$ 0 : only R(esidue): the Ejectile and Core are identical
\\    $>$ 0 : both E(jectile) and R(esidue) independently of Core nucleus.

{\tt postprior} = {\tt ip1} for finite-range transfer couplings,\\
{\tt remnant} = {\tt ip2} for finite-range transfer couplings,\\
{\tt pauli} = attempted Pauli blocking using non-orthogonality couplings.


\noindent{\\ \bf \&NUCLEUS namelist}\\
{\tt part,name,mass,charge,spin,parity,be,n,l,j,ia,a,kind,lmax,nch,nce,ampl
}\\
 %
where this card is repeated for each {\tt part} beginning P: projectile, C:
core, V: valence, T: target, E: ejectile, R: residue. The number of nuclei
(4, 5 or 6) depends on {\tt trans}: see above.\\
 %
{\tt name,mass,charge} as {\tt name, mass, zp/t} on Card 6,\\
{\tt spin,parity} as {\tt Jp/t, Bandp/t} on Card 7.\\
{\tt be,n,l,j,ia,a,kind,lmax,nch} prescribe the projectile P bound state,
and with transfers also for residue R.\\
{\tt nce} is the number of Core excited states.

\hangindent 5ex\hangafter 1
If {\tt nce}$>$0, then read that number of {\bf
\&CORESTATES} namelists. \noindent{\\ \bf \&CORESTATES namelist}: {\tt
spin,parity,ex }\\ for spin, parity ($\pm1$) and excitation energy of each Core
state above the ground state.

\noindent{\\ \bf \&BIN namelist}\\
{\tt
spin,parity, step,start,end,energy, n,l,j,isc,ipc, kind,lmax,nch,ia,il,ampl
}\\
These are repeated until an empty  {\bf \&BIN} namelist is encountered
({\tt step}=0).\\
Each bin set has the same {\tt spin, parity, l, j, isc, ipc, kind, lmax, nch, ia,
il, ampl}, but a different energy.
The energy range is divided into ({\tt end-start})/{\tt step} bins.
If {\tt energy} then these are evenly spaced in energy, else they
are evenly spaced in momentum $k \propto \sqrt{E}$.
Changing or starting partial waves is equivalent to {\tt start}= 0.001.
\\
The values of {\tt lmax, nch, ia, il} are only needed coupled channels bins
{\tt kind}=3, with {\tt il} being the channel number of the incoming partial wave,
which, if {\tt il}=0, is defined as the channel with quantum numbers
{\tt l, j} and {\tt ia}. The array {\tt ampl} gives overall multiplicative
spectroscopic amplitudes to the bin. The values of {\tt isc, ipc, kind, lmax}
have the same meaning as in Card 13, with {\tt j} an alias for {\tt jn}.\\
If {\tt l,j,ia} are not set then {\tt kind}=3 coupled-channels bins are generated
using {\tt lmax}, for all incoming waves. Setting {\tt il} will select an
incoming channel.

\noindent{\\ \bf \&POTENTIAL namelist}\\
{\tt
part, a1, a2, rc, ac, v, vr0, a, w, wr0, aw,
wd, wdr0, awd, vso, rso0, aso, shape, freal, fimag,
vsot, rsot0, asot, l, parity, nosub, itt,
beta2, beta3, idef, beta2c, beta3c, beta2m, beta3m
}\\
where this card is repeated for each {\tt part} beginning
P: projectile-target  optical potential, C: core-target optical potential,
V: valence-target  optical potential,
T: projectile (C+V) ground state,
B: projectile channels {\em not} containing the ground state,
T: transfer channel optical potential,
E: ejectile bound state, R: residue bound state.\\
Different {\tt part}=B(in) potentials may be defined depending on {\tt parity}
or partial wave {\tt l}. \\
{\tt nosub} means that the P(rojectile) optical potential is {\em added} to
the CDCC couplings as a diagonal in all projectile state channels.\\

\hangindent 8ex\hangafter 3
For deformations, {\tt beta2m, beta3m} are the nuclear {\em fractional}
deformations, and {\tt beta2c, beta3c} are the Coulomb equivalents (both with default
values  {\tt beta2, beta3} respectively).\\
The Coulomb and nuclear deformations may also be restricted by {\tt idef}:
\\   = 0 : Coulomb \& nuclear (complex)
\\  = 1 : nuclear (complex) only
\\  = 2 : Coulomb only

\hangindent 0ex\hangafter 1


\newpage
\vspace{8mm}
{\bf Sample CDCC input file:}
\begin{verbatim}
11Be+4He spdf; 1+5*10+2*5 chs 0-10 MeV, q=0-3 2200 MeV, 30/100 fm
CDCC
 &CDCC
   hcm=0 rmatch=-30 absend=-50 rasym=100 accrcy=0.001
   elab=2200
   jbord=  0   60 200 2500
   jump =  4    5  20
   thmax=30 thinc=.05 smats=2 xstabl=1  cutr=-10 cutc=0
   nk=50 ncoul=0 reor=0 q=3
   /
 &NUCLEUS part='Proj' name='11Be' spin=0.5 parity=+1 be = 0.500 n=2 l=0 j=0.5 /
 &NUCLEUS part='Core' name='10Be' charge=4 mass=10 /
 &NUCLEUS part='Valence' name='neutron' charge=0 mass=1 spin=0.5/
 &NUCLEUS part='Target' name='4He' charge=2 mass=4 /

 &BIN spin=0.5 parity=+1 step=0.5 end=10. energy=F l=0 j=0.5/
 &BIN spin=0.5 parity=-1 step=0.5 end=10. energy=F l=1 j=0.5/
 &BIN spin=1.5 parity=-1 step=0.5 end=10. energy=F l=1 j=1.5/
 &BIN spin=1.5 parity=+1 step=1.0 end=10. energy=F l=2 j=1.5/
 &BIN spin=2.5 parity=+1 step=1.0 end=10. energy=F l=2 j=2.5/
 &BIN spin=2.5 parity=-1 step=2.0 end=10. energy=F l=3 j=2.5/
 &BIN spin=3.5 parity=-1 step=2.0 end=10. energy=F l=3 j=3.5/
 &BIN /

 &POTENTIAL part='Proj' a1=11 a2=4 rc=1.0  /
 &POTENTIAL part='Core' a1=10 a2=4 rc=1.0
            V=46.92 vr0=1.204 a=0.53 W=23.46 wr0=1.328 aw=0.53 /
 &POTENTIAL part='Valence' a1=4 rc=1.3
            V=37.14 vr0=1.17  a=0.75 W=8.12  wr0=1.26  aw=0.58 /
 &POTENTIAL part='Gs' a1=10 v=51.51 vr0=1.39 a=.52 vso=0.38 rso0=1.39 aso=0.52/
 &POTENTIAL part='Bi' a1=10 v=28.38 vr0=1.39 a=.52 vso=0.38 rso0=1.39 aso=0.52/
\end{verbatim}

\newpage

%\section*{Appendix II : Program Versions}
%%
%\bigskip
%
%The version described here ....
%


\section*{Appendix IV : {\sc Fortran} file allocations}
%
\begin{tabular}{|l | c c c | l | l|}
\hline
File & Format & LRECL  & Access & Routines & Use \\
 no. & (Fix/Var) & (bytes) & (Seq/Ran) &  &  \\
\hline
1    &  V &             & S  &     SFRESCO& FRESCO input when searching\\
2    &  V &             & S  &     SFRESCO& Search specification file\\
3    &  V &             & S  &     FREADF, FR& Temporary namelists file\\
4    &  F &     80      & S  &     INTER & external KIND=1,2 form factors\\
     &    &             &    &     POTENT &external potentials\\
5    &  F &     80      & S  &         &   standard input\\
6    &  V &    142      & S  &          &  standard output\\
7    &  F &     80      & S  &     DISPX & elastic S-matrix elements\\
8 &  F &   sector    & R  &     FR,INTER & s/p wfs, channel wfs\\
9 &  F &  NLL*NLO*16    & R& Q/KERNEL &  complex transfer multipoles\\
10 &  V &             & S  &     FR,CRISS & S-matrix elements (cfs)\\
11 &  F & NLL*NLO*8 &  R & Q/KERNEL &  real transfer multipole\\
12 &  V &NLL*NLO*16 & S  &   KERNEL/SOURCE&transfer kernels\\
13    &  F &    80      & S  &      FR   &    total cross sections for each Elab\\
14  & V &             & S  &   INTER/CPAIR &interaction potentials\\
16    &  F &     80      & S  &       CRISS &  tables of cross sections\\
17    &  F &     80      & S  &       FR  &    output scattering waves\\
18 &  V &             & S  &       FR  &    wfns of 'best' iterate\\
19    &  F &    N      & S  &     FR  &   Local couplings\\
     &  & & & &\\
20--33&    &            &    &     Available for users & (eg bound states, amplitudes)\\
     &  & & & &\\
34    &  F &    80      & S  &     POTENT &   output potentials\\
35    &  F &    80      & S  &     FR & Astrophysical S-factors for Ecm\\
36    &    &            &    &     CRISS & output scattering AMPL amplitudes\\
37    &    &            &    &     CRISS & output scattering FAM amplitudes\\
38    &  F &    80      & S  &     DISPX &    cross sections for each J/pi\\
39    &  F &    80      & S  &      FR   &    2 cross sections for each Elab\\
40    &  F &    80      & S  &      FR   &    all cross sections for each Elab\\
41    &  F &    80      & S  &     SOURCE &   source terms at each iteration\\
42    &  F &    80      & S  &     SOURCE &   bin wavefunctions for each E\\
43    &  F &    80      & S  &     INFORM &   bin phase shifts as k functions\\
44    &  F &    80      & S  &     INFORM &   bin phase shifts as E functions\\
45    &  F &    80      & S  &     ERWIN   &  scat phase shift as E functions\\
46    &  F &    80      & S  &     INFORM &   bs wave functions \& Whit ratios\\
47    &  F &    80      & S  &            &   reduced matrix elements \\
48    & V &   133      & S  &       FR  & misc log file\\
55    &   F &            &    &     INFORM & Single-particle wave functions\\
56    &   F &            &    &     FR & Jtotal fusion \& reaction cross\\
57    &   F &            &    &     FR & Output of CDCC amplitudes\\
58    &   F &            &    &     INFORM & Bound state wave functions \\
59    &   F &            &    &     INFORM & Bound state vertex functions \\
60-62 &   F &            &    &     RMATRIX & Trace of R-matrix calculations\\
66    &  V  &NLL*NLO*16  & S  &     INTER & KIND=1 nonlocal formfactor\\
71    &   F &    132     & S  &     FR   &  20 pw phase shifts as Elab functions\\
75    &  F &            & S  &     FR & Astrophysical S-factors for Elab\\
89    &   F &            &    &     MULTIP& Folded coupling potentials\\
\hline
\end{tabular}

\begin{tabular}{|l | c c c | l | l|}
\hline
File & Format & LRECL  & Access & Routines & Use \\
 no. & (Fix/Var) & (bytes) & (Seq/Ran) &  &  \\
\hline



105     & V &         & S  &     FCN   &  $\chi^2$ progress during fitting\\
200 & F & 80     & S  &     CRISS   & Elastic cross section if not fort.201 \\
201-210 & F & 80     & S  &     CRISS   & Cross sections (cf 14) for states 1-10\\
301 & V &      & S  &     CDCIN   & New Fresco input \\
302 & V &      & S  &     CDCIN   & New Fresco input  (temp) \\
303 & V &      & S  &     SFRESCO   & Input search file \\
304 & V &      & S  &     SFRESCO   & Output plot file \\
305 & V &      & S  &     CDCIN   & New Fresco input from cdcin, col format \\
306 & V &      & S  &     SFRESCO/FRXX0   & Input Fresco file \\
307 & V &      & S  &     SFRESCO/FRXX0   & Initial Output Fresco file \\
308 & V &      & S  &     SFRESCO/FRXX0   & Main Output Fresco file \\
\hline
\end{tabular}


\newpage
\section*{Appendix V : KIND=1 Spin Transfer Couplings}

\large

The definition of the KIND=1 spin transfer couplings is not given
in the Computer Physics Reports article, so in versions after March 1998 these
are redefined for IP3=0 or 1, and new Racah algebra factors included.

\bigskip

We want to calculate the coupling interactions of the monopole operator
$   {\bf S}([\ell,s_p]s_t,s_t)  $, where $s_p$ is the spin transfer of the
projectile $I_p$,  $s_t$ is the spin transfer of the
target $I_t$, and $\ell$ is the orbital angular momentum transfer.
 %
These coupled operators are defined following in Bohr \& Mottelson, Vol.
1, section 1A-5c, as
\begin{equation}
(F_{\lambda_1} G_{\lambda_2})_{\lambda \mu}
  = \sum_{\mu_1 \mu_2} \langle \lambda_1 \mu_1 ,  \lambda_2 \mu_2
  |  \lambda \mu \rangle F_{\lambda_1\mu_1} G_{\lambda_2\mu_2} \ ,
\end{equation}
applied for the case of $F_{s_t} = [\ell,s_p]s_t$ and $ G_{s_t}=s_t$.
The overall ${\bf S}$ operator is a monopole (scalar), for which the tensor
product is thus of the kind
\begin{equation}
(F_{\lambda} G_{\lambda})_{00}
  = (2\lambda+1)^{-1/2} \sum_{\mu} (-1)^{\lambda-\mu} ~ F_{\lambda\mu}
  G_{\lambda-\mu} \ .
\end{equation}
This differs from common definitions (eg of Satchler) by a factor of
$ (-1)^\lambda (2\lambda+1)^{-1/2} $.

\bigskip

Reduced matrix elements are defined everywhere in {\sc Fresco} by:
\begin{equation}
% <j_f m_f | \hat{O}_{j m} | J_i m_i > = { <j_i m_i \ j m |\ j_f m_f > \over
% \hat{j}_f } < j_f || \hat{O}_j || j_i>
\langle j_f m_f | \hat{O}_{\lambda \mu} | J_i m_i \rangle
 = (2j_f+1)^{-1/2} ~ \langle j_i m_i ,\lambda \mu | j_f m_f \rangle ~
\langle j_f || \hat{O}_\lambda || j_i\rangle
\end{equation}
\bigskip

The  matrix elements of this operator are
\begin{eqnarray}
&& \langle(LI_p)J,I_t;J_TM_T | ~{\bf S}([\ell,s_p]s_t,s_t) ~| (L'I_p')J',I_t';J_TM_T \rangle
   \nonumber \\
& = &  (-1)^{s_t+J_T+J'+I_t}
   \left\{ \begin{array}{ccc}
    J'  & I_t' & J_T \\
    I_t & J    & s_t \\
    \end{array} \right\}   \hat{J} \hat{J'}
    \left\{ \begin{array}{ccc}
    L' & I_p' & J'  \\
    \ell & s_p & s_t  \\
    L & I_p & J'  \\
    \end{array} \right\}  ~
     \frac{1}{\sqrt{4\pi}} ~ \hat{\ell} \hat{L} \langle L 0 \ \ell 0 | L'0 \rangle   \nonumber \\
    &  &   \times \langle I_p||s_p||I_p'\rangle \langle I_t||s_t||I_t'\rangle
\end{eqnarray}
In using KIND=1 couplings with IP3=0, the first line of these factors is generated automatically. \\
The product of the
reduced matrix elements for the intrinsic nuclear states,
$\langle I_p||s_p||I_p'\rangle \langle I_t||s_t||I_t'\rangle$,
 has to be included explicitly in the factor FSCALE,
 or in the radial form factors.
The radial shapes have to be read in using Cards 16.7.\\

With IP3=2, implying jlmP input for target-only couplings,
\begin{itemize}
\item
 the projectile diagonal reduced matrix element
$\langle I_p||s_p=0||I_p\rangle = \hat{I_p}$ is supplied by Fresco,
\item
the factor $1/\sqrt{4\pi}$ is omitted above,
\item
and an additional symmetric factor of
%$\sqrt{\hat{I_t'}\hat{I_t}}$
$((2I_t'+1)(2I_t+1))^{1/4}$
is supplied to allow the
monopole radial form factors to have their physical
values for any target spin.
\item
The forward and reverse form factors should be identical for Hermitian couplings.

\end{itemize}

With IP3=3, implying a similar input for projectile-only couplings,
\begin{itemize}
\item
 the target diagonal reduced matrix element
 $\langle I_t||s_t=0||I_t\rangle = \hat{I_t}$ is supplied by Fresco,
 \item
 the factor $1/\sqrt{4\pi}$ is omitted above,
 \item
 and an additional symmetric factor of $((2I_t'+1)(2I_t+1))^{1/4}$
 is supplied to allow the monopole radial form factors to have their physical
 values for any target spin.
 The forward and reverse form factors should be identical for Hermitian couplings.
\end{itemize}


%~\\
%The reduced matrix elements are defined by:
%\begin{equation}
%<j_f m_f | \hat{O}_{j m} | J_i m_i > = { <j_i m_i \ j m |\ j_f m_f > \over
%\hat{j}_f } < j_f || \hat{O}_j || j_i>
%\end{equation}
%
\newpage
\section*{Appendix VI : Search version {\sc Sfresco}}

This is a program which searches for a $\chi^2$ minimum when comparing
the outputs of FRESCO with sets of data, using the {\sc Minuit} search
routines.
%
The inputs for {\sc Sfresco} specify the FRESCO input and output files,
the number and types of search variables, and the experimental data sets.

A {\tt search\_file} format is defined as follows:
\begin{verbatim}
Card 1:
(2*string, 2*integer: free format)
input_file, output_file, nvariables, ndatasets
\end{verbatim}
\smallskip

{\tt input\_file}: name of file (in single quotes) with regular FRESCO input\\
{\tt output\_file}: name of file for the final FRESCO output\\
{\tt nvariables}: number of search variables\\
{\tt ndatasets}:  number of experimental data sets.\\

Repeated {\tt nvariables} times, a namelist with some of the following variables:
\begin{verbatim}
Card 2: (namelist: 'variable')
name,kind,step,valmin,valmax, null,
kp,pline,col,potential,
dataset,datanorm,
nafrac,afrac,
energy, jtot,par,channel,width
\end{verbatim}
{\tt name} is name of variable (up to 10 characters),\\
{\tt kind}: 0=ignore, 1=potential, 2=afrac, 3=R-matrix energy,\\
   \hspace*{10mm} 4=R-mat partial width, 5=dataset normalisation\\
{\tt step}: step size for finding derivatives etc, zero for fixed. (default 0.01)\\
{\tt valmin}: strict lower bound (if non-zero),\\
{\tt valmax}: strict upper bound (if non-zero), \\
{\tt null}: (default -124578) `undefined' value,\\

{\tt kind=}{\bf 1: Variable potential parameter:}\\
{\tt kp}: number of potential as on Card 10,\\
{\tt pline}: order of which Card 10 within potential definition,\\
{\tt col}:   number on Card 10: selecting P0--P7 ,\\
{\tt potential}: value of this potential parameter (default {\tt null}: use {\tt input\_file})\\

{\tt kind=}{\bf 2: Variable spectroscopic amplitude:}\\
{\tt nafrac}: order number of Cards 17 in {\tt input\_file}\\
{\tt afrac}: spectroscopic amplitude A (default {\tt null}: use {\tt input\_file})\\

{\tt kind=}{\bf 3: Variable energy of R-matrix term:}\\
{\tt term}: term number (default 1)\\
{\tt jtot}: J value for additional R-matrix term\\
{\tt par}: parity (+1 or --1)\\
{\tt energy}: energy (cm MeV in entrance channel) of additional R-matrix term (default 0)\\
{\tt nopot}: Disable potential \& Buttle correction for this J/pi set (logical T or F, default F)\\

{\tt kind=}{\bf 4: Variable partial width of R-matrix term:}\\
{\tt channel}: channel number (in order generated by FRESCO)\\
{\tt term}: term number (default 1)\\
{\tt width}: width for this channel (MeV$^{1/2}$) (default 0)\\

{\tt kind=}{\bf 5: Variable dataset normalisation:}\\
{\tt dataset}: index of which dataset (1 up to {\tt ndatasets}) (default 1)\\
{\tt datanorm}: search on absolute normalisation of this dataset (default: 1.0)\\

Repeated {\tt ndatasets} times, namelist with some of the following variables, then the
corresponding data:
\begin{verbatim}
Card 3: (namelist: 'data')
type,data_file,points,delta,xmin,lab,energy,angle,
idir,iscale,abserr,ic,ia,k,q,jtot,par,channel,value,error,
pel,exl,labe,lin,lex
\end{verbatim}

\hangindent 8ex\hangafter 1
{\tt type} (default 0)
\\ = 0 angular distribution for fixed energy
\\ = 1 excitation and angular cross section double distributions
\\ = 2 excitation cross section for fixed angle
\\ = 3 excitation total cross section. Also: ic=0: {\tt ia}=0 is total reaction cross section;
{\tt ia}=1 is total fusion cross section; 1$<${\tt ia}$<$NFUS+1 are fusion from potential
KFUS (on Card 5) for inelastic state {\tt ia}--1.
\\ = 4 excitation phase shift for fixed partial wave
\\ = 5 desired factor for bound state search (binding energy or potential scale
 according to ISC).
\\ = 6 specific experimental constraint on a search parameter {\tt par},
    to be {\tt value} with error {\tt error} ({\tt abserr}=T or F).


{\tt data\_file}: name of data file with data, `$=$' for {\tt search\_file},
 `$<$' for stdin (default `$=$')\\
{\tt points}: number of data points (default: keep reading as many as possible),
\\
{\tt delta}: if non-zero, construct linear x-scale from {\tt xmin} in steps of {\tt delta},
 (default 0)\\
{\tt lab}: T or F for lab angles and cross sections (default F).\\
{\tt energy}: lab energy for this {\tt type}=0 dataset (default: use {\tt ELAB(1)} from Card 19)\\
{\tt pel,exl,labe,lin,lex}: specify input channels for this data set, as in Card 18

\hangindent 8ex\hangafter 1
{\tt idir} (default 0)
\\ =-1 cross-section data are given as astrophysical S-factors, but will be converted to absolute,
\\ = 0 cross-section data are given in absolute units,
\\ = 1 cross-section data are ratio to Rutherford
\\ = 2 cross sections are given in absolute units but will be
         converted to ratio to rutherford.



\hangindent 8ex\hangafter 1
{\tt iscale} (default 2)
\\ = -1 dimensionless (eg ratio to rutherford if {\tt idir}=1),
\\ = 0 absolute cross-section units are fermi-squared/sr.
\\ = 1 absolute scale is barn/sr
\\ = 2 absolute scale is mb/sr
\\ = 3 absolute scale is micro-b/sr


{\tt abserr}: T or F for absolute error (default F)\\
{\tt ic}: partition number of cross section channel (index of Cards 6)  (default 1)\\
{\tt ia}: excitation pair number of cross section (index of Cards 7)  (default 1)\\
{\tt k}: tensor rank of cross section (0, 1, 2,..) (default 0). \\
{\tt q}: index for tensor $T_{kq}$ of cross section (0, 1, 2,..) (default 0).
    $T_{00}$ = angular distribution $\sigma(\theta)$.\\
{\tt jtot}: The {\tt J} value (as on Card 2) for phase shift\\
{\tt par}: The parity (--1 or +1) for phase shift, for {\tt type}=4\\
{\tt par}~~~~~~ or number of search parameter, for {\tt type}=6.\\
{\tt channel}: Channel number of entrance channel in coupled channels set (default 1),
for phase shift\\
{\tt value}: Value of search parameter, for {\tt type}=6\\
{\tt error}: Error in value of search parameter, for {\tt type}=6\\

\hangindent 8ex\hangafter 1
{\bf Data input} (free format):
\\ if {\tt type}=1, read:  {\tt energy, angle, value, error}
\\ if {\tt type}=5, read:  {\tt kn, target, error}
\\ otherwise:
\\ if {\tt delta}$\ne$0, read:  {\tt value, error}
\\ if {\tt delta}=0, read:  {\tt x, value, error}


where {\tt x} is angle for {\tt type}=0, and lab energy for {\tt type}=2, 3 and 4.

%\newpage
\subsection*{{\tt SFresco} Input}
The inputs to {\tt SFresco} are\\
Card 1: name of {\tt search\_file}\\

\hangindent 8ex\hangafter 1
Cards 2: repeatedly one of (in all upper or all lower case):
\\ EX: exit
\\ Q: query status of search variables
\\ SET {\tt var val}: set variable number {\tt var} to value {\tt val}.
\\ FIX {\tt var}: fix variable number {\tt var} (set {\tt step}=0).
\\ STEP {\tt var step}: unfix variable number {\tt var} with step {\tt step}.
\\ SCAN {\tt var val1 val2 step}: scan variable number {\tt var}
from value {\tt val1} to value {\tt val2} in steps of {\tt step}.
\\ SHOW: list all datasets with current predictions and $\chi$ values.
\\ PLOT {\tt plotfile}: write file (default: {\tt search.plot}) for reading by {\tt xmgr, xvgr},
 with data and theoretical curves.
\\ LINE {\tt plotfile}: write file (default: {\tt search.plot}) for reading by {\tt xmgr, xvgr},
 with theoretical curves only.
\\ READ {\tt file}: read plot output {\tt file} for further searches, if not:
\\ READ {\tt snapfile}: if the input string {\tt file} contains the string 'snap',
read to the end of previous snap output {\tt snapfile}, for further searches.
\\ ESCAN {\tt emin emax estep}: scan lab. energy in incident channel, to give
phase shifts in fort.71, fusion \& reaction cross sections in fort.40,
 and S-factors in fort.35 and 75.
This is to look for/at resonances, or at behaviour out of data's energy range,
not for $\chi^2$ fitting.
\\ MIN: call {\tt minuit} interactively, to read from stdin.
Return to {\tt SFresco} by {\tt minuit} command {\tt end}.

\newpage
A example of this {\tt search\_file} input is:
%\small
\begin{verbatim}
'6hep.in' '6hep.out' 7 1
 &variable kind=1 name='V' kp=1 pline=2 col=1 /
 &variable kind=1 name='r0' kp=1 pline=2 col=2 valmin=.7 valmax=2.0/
 &variable kind=1 name='a' kp=1 pline=2 col=3 /
 &variable kind=1 name='W' kp=1 pline=2 col=4 /
 &variable kind=1 name='rw' kp=1 pline=2 col=5 valmin=.7 valmax=2.0/
 &variable kind=1 name='aw' kp=1 pline=2 col=6 /
 &variable kind=5 name='exptnorm' /
 &data type=0 iscale=2  idir=0 lab=F  abserr=T/
 13.99     565.7     28.3
 17.59     478.19    23.91
 21.22     425.41    21.28
 24.92     338.44    16.93
 28.68     284.71    14.24
 36.48     168.59    8.43
 40.56     137.9     6.86
 44.8      104.75    5.24
 53.       30.0      9.14
 53.94     38.45     2.71
 59.       25.62     2.55
 &
\end{verbatim}
\normalsize
A example of the full {\tt SFresco} input, suitable for batch running, is:
%\small
\begin{verbatim}
search4.in
min
fix 4
migrad
end
q
show
plot
\end{verbatim}
\normalsize
where {\tt search4.in} is the above {\tt search\_file} input file name.
\end{document}
