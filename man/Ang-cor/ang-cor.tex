\documentclass[aps,prc,amsfonts,nofootinbib]{revtex4}
\usepackage{bm,amsmath,epsfig,dcolumn,slashbox}

\begin{document}

\title{ Angular Correlations and {\sc fresco}}
\author{ C.~R.~Brune }
\date{December 2, 2009}
\maketitle

\section{Introduction}

The nature of the particle emission resulting from the decay
$B\rightarrow C+c$ following the reaction $A(a,b)B$ is discussed
in Sec.~10.7.4 of Satchler~\cite{Sat83}.
The double differential cross section for detecting $b$ and $c$
is given by Satchler,
Eq.~10.126\footnote{The branching-ratio factor $\Gamma_c/\Gamma$
in Eq.~10.126 has been suppressed.}:
\begin{equation}
\frac{{\rm d}^2 \sigma}{{\rm d}\Omega_b{\rm d}\Omega_c} =
\frac{{\rm d}\sigma}{{\rm d}\Omega_b} \frac{W}{4\pi} .
\end{equation}
The angular correlation function $W$ given by Satchler Eqs.~10.127 and~10.130
\begin{equation}
W=\sum_{kq} t_{kq}(I_B) \, R_k \, C_{kq}^* ,
\end{equation}
where $I_B$ and $t_{kq}(I_B)$ are the spin and polarization tensor of
the nucleus $B$, $R_k$ are the real radiation parameters,
and $C_{kq}$ are related to a the spherical harmonics:
\begin{equation}
C_{kq}=\left [ \frac{4\pi}{2k+1} \right]^{1/2} Y_{kq}.
\end{equation}
Following the convention used by {\sc fresco}~\cite{Tho88}, we will
choose the $z$ axis to be along the incident beam.
We then have  $t_{kq}(I_B)=t_{kq}(\theta_B)$ and
$C_{kq}=C_{kq}(\theta_c,\phi_c)$ where $\theta_B=\pi-\theta_b$ and
$\theta_c$ are the usual polar angles in the c.m. system and $\phi_c$ is
the azimuthal angle between the particles $b$ and $c$.

\section{Radiation Parameters}

The particle radiation parameters $R_k$ are discussed in Sec.~10.7.4.2
of Satchler.
Assuming that $B\rightarrow C+c$ occurs with a single orbital
angular momentum $L$ and channel spin $S$,
\begin{equation}
R_k=(2I_B+1)^{1/2}(2L+1)(-1)^{k+S-I_B}\langle LL00|k0\rangle \,
W(LLI_BI_B;kS),
\end{equation}
where $\langle LL00|k0\rangle$ is a Clebsch-Gordon coefficient
and $W(LLI_BI_B;kS)$ is a Racah coefficient.
Parity conservation implies that $k$ must be even.
For the case of ${}^{19}{\rm Ne}^*\rightarrow
{}^{15}{\rm O}(1/2^-)+\alpha(0^+)$
we have $S=\frac{1}{2}$ and $L$ is uniquely determined for
a given $J^\pi$ state in ${}^{19}{\rm Ne}$.
For the general case, relative partial width amplitudes for the decay
$B\rightarrow C+c$ must be specified.

\section{Polarization Tensors}

The polarization tensor $t_{kq}(I_B)$ describes the polarization
of the final nucleus $B$; the precise definition is given in
Secs.~10.3.2 and~10.3.3 of Satchler.
The polarization tensors can be calculated with {\sc fresco}
by noting that the {\sc fresco} scattering amplitudes
$f_{m'M':mM}$ are equivalent to Satchler's transition
matrix elements $T_{\beta\alpha}$ defined by his Eq.~9.2.
Satchler's Eq.~10.32 reads
\begin{equation}
t_{kq}(I_B)=\frac{tr[{\bm T}{\bm T}^\dag \tau_{kq}(I_B)]}
{tr[{\bm T}{\bm T}^\dag]}.
\end{equation}
In the notation of Thompson (analogous to Eq.~3.33 of Ref.~\cite{Tho88})
this formula becomes
\begin{eqnarray}
t_{kq}(I_B) &=&\frac{tr[{\bm f}{\bm f}^\dag \tau_{kq}(I_B)]}
  {tr[{\bm f}{\bm f}^\dag]} \nonumber \\
&=& \sqrt{2k+1} \, \frac{ \displaystyle \sum_{m'M'mM} f_{m'M':mM}(\theta)^*
  f_{m'M'':mM}(\theta) \langle I_BM'kq|I_BM''\rangle }
  { \displaystyle \sum_{m'M'mM} |f_{m'M':mM}(\theta)|^2 },
\end{eqnarray}
where $M''=M'+q$ is required for the Clebcsh-Gordon coefficient to be non-zero.
Also note that the differential cross section is given by
\begin{equation}
\frac{{\rm d}\sigma}{{\rm d}\Omega_b}=\frac{1}{(2I_A+1)(2I_a+1)}
\displaystyle \sum_{m'M'mM} |f_{m'M':mM}(\theta)|^2.
\end{equation}
The above equations can be used to calculate the angular correlation
using the scattering amplitudes output by {\sc fresco}.
Note that the scattering amplitude
file output by {\sc fresco}  is controlled with the LAMPL parameter;
typically LAMPL=-2 is needed. 

\begin{thebibliography}{9}

\bibitem{Sat83} G.~R.~Satchler, {\it Direct Nuclear Reactions},
  (Clarendon, Oxford, 1983).

\bibitem{Tho88} I.~J.Thompson, Computer Physics Reports {\bf 7}, 167 (1988).

\end{thebibliography}

\end{document}
