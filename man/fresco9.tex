\documentclass[11pt,a4paper]{article}
\input wider.sty
\parindent 0pt
\parskip 10pt
\newcommand{\vecr}{{\bf r}}
\newcommand{\veck}{{\bf k}}
\newcommand{\vecR}{{\bf R}}
\newcommand{\vecl}{{\bf l}}
\newcommand{\vecL}{{\bf L}}
\newcommand{\vecJ}{{\bf J}}
\newcommand{\vecs}{{\bf s}}
\newcommand{\vecx}{{\bf x}}
\newcommand{\Vee}{{\sf V}}
\newcommand{\brho}{{\mbox{\boldmath $\rho$}}}
\newcommand{\half}{\frac{1}{2}}
\begin{document}

\title{\bf Coupled Reaction Channels Calculations\\
in Nuclear Physics}

\author{Ian J. Thompson\\
%
Department of Engineering Mathematics,\\
Bristol University, Bristol, BS8 1TR, UK\\
~\\
Address after 1st August 1988:\\
%
Department of Physics, Surrey University, Guildford, GU2 5XH, UK}

\date{Computer Physics Reports {\bf 7} (1988) 167-212 \\
        Latex version of Jan 2000. With corrections to original article. \\
        Reprinted \today   }
\maketitle
\begin{abstract}
This paper describes the components and methods of a comprehensive
code for coupled reaction channels calculations in nuclear physics.
Procedures are described which are common to the modelling of reactions
induced by light and medium-mass ions, and which are sufficient to
calculate the effects of successive processes to any order.
\end{abstract}
% .sk 1i
% %%.rc 3 on
% Revisions after University of Bristol Report EM/NP/01
% %%.rc 3 off
% .sk 1
% %%.rc 4 on
%  %Revisions to text after first submission to Computer Physics Reports
% %%.rc 4 off
% .sk 1
% %%.rc 5 on
% Revisions to text after publication in Computer Physics Reports
%%.rc 5 off
%.se htitle ='Coupled Reaction Channels'
%.se hauthor = 'I.J. Thompson'
%.se SYSVARH = 'yes'
%.se SYSVART = 'no'
%.se SYSVARX = 'no'

\section{Introduction}

When two nuclei approach each other
they may interact in several ways.
In the first approximation they may be regarded as clusters of nucleons,
and their primary interaction results from the inter-nucleon two-body force,
which has an
average effect found by folding the force over the internal configurations
of the two clusters.
However, these configurations are not static, and
one or more rearrangement processes may occur during the
time in which nuclei are together during a collision.
As well as elastic scattering, with the projectile and the target
remaining in their ground states,
various kinds of non-elastic interactions may have time to operate.

Inelastic excitations may occur, for example when one or both of the
nuclei are deformed or deformable, with the result that
higher-energy states of the nuclei may become populated.
Single-particle excitations are another kind of inelastic
process, when a particle in one of the nuclei is excited during the reaction
from its initial bound state to another state which may be bound
or unbound.
Nucleons may also transfer from nucleus to the other, either singly,
or as the simultaneous transfer of two nucleons as a particle cluster.

In this paper I will consider some mathematical models sufficient to
describe these processes, and the principal interest will be in calculating
the effects of their occurring successively as multi-step
processes.  One-step processes have been traditional described with the
Distorted Wave Born Approximation (DWBA), and although second-order DWBA
expressions can be written down and computed, I shall be mainly concerned
with {\em coupled-channels} formalisms, in order to predict the
effects of multi-step processes to any or all orders.

The present work is in the framework of Direct Reaction theory, which
attempts to solve the Schr\"odinger equation for a specific model
of the components thought to be important in the reaction, and of their
interaction potentials.
In direct reaction theories, the phases describing the coherence
of all components of the wave function are coherently maintained,
and the potentials typically include imaginary components to model
how flux is lost from the channels of the model to other channels.
By contrast, a theory of compound-nucleus processes would make
approximations as to the statistical distribution of the inelastic excitations.
Direct reaction theory would describe these effects with an imaginary potential,
with the argument that because the compound
nucleus channels are incoherent with respect to each other, their effects
back on the direct-reaction channels are also incoherent, and may hence be
represented as a statistical loss of flux that occurs when the nuclei
overlap each other to any significant extent.

Comprehensive accounts of the physical assumptions, methods and results
of direct reaction theory is given in the papers by Tamura et al. \cite{tam74},
and in the books by Austern \cite{aust70} and Satchler \cite{SATCH83}.
The aim of the present paper is to show how a large subset of the
direct reaction mechanisms can be modelled in a general purpose computer
program.  For definiteness, I am following the methods used in the recent
code FRESCO, while also mentioning, where appropriate, additional features that
could well be included within its framework. Brief descriptions will also be
given of alternative methods, and the relative merits of the different
procedures will be discussed.

The code of ref.\cite{FRESCO} has not been developed to include any
special treatment of the long-range Coulomb mechanisms that are significant
when heavy ions are incident on strongly-deformed nuclei.
For methods of
dealing with these processes efficiently, the reader is referred
to refs. \cite{AROSA},~\cite{AROSA2},~\cite{rbrown80}, and~\cite{PIECAN},~\cite{TOLSMA2}.

The organisation of this paper is as follows. Section 2 will give a
derivation of the coupled reaction channels (CRC) equations within the
framework of the Feshbach formalism for direct reactions, and show how
one-step and two-step DWBA (etc.) are special cases of the CRC equations. In
both the CRC and DWBA formalisms, particular attention is paid to the
treatment of the so-called `non-orthogonality terms' which arise
with couplings between different mass partitions.

The wave functions needed to specify scattering states and nuclear
eigenstates are given in section 3. Single-nucleon wave functions are
defined for both bound and unbound energies, and a method for solving the
coupled-channels bound state eigen-problem is presented. Two-nucleon wave
functions are described in both the centre-of-mass and independent-radii
coordinates. Finally, the formulae are given for calculating the
observable cross sections and polarisations in terms of the S-matrix
elements of the scattering wave functions.

Section 4 specifies some of the different kinds of potentials that exist
between nuclei, or can couple together the excited states of a single
nucleus because of the interaction with its reaction partner. Details of
the rotational model, single-particle excitations and particle transfers
are given.

The methods used to solve the CRC equations are described in section 5,
along with the procedures for calculating the transfer form factors in
terms of a two-dimensional kernel function.
Appendix A defines some of the notation and phase conventions used, while
Appendix B summarises the more widely-known coupled-channels codes which
have been written to solve problems in nuclear physics.
%%.rc 4 off
%%%% Introduction ends on page &$PN., at &$LC lines from end.
%%%
\section{Coupled Reaction Channels Formalism}
%%%% CRC begins on page &$PN

The coupled reaction channels (CRC) model of direct reactions in nuclear
physics proceeds by constructing a model of the system wave function, and
solving Schr\"odinger's equation as accurately as possible within
that model space.  The model used here projects the complete wave function
$\overline{\Psi }$ onto a product $\phi_i \equiv\phi_{ip} *\phi_{it}$
of projectile and target states with a wave function $\psi_i (\vecR _
i)$ describing their relative motion:
\begin{eqnarray} \label{discsum}
  P\overline{\Psi } \equiv\Psi = \sum _ i ^ N \phi_i\psi_i (\vecR_i)
\end{eqnarray}
The basis states $\phi_{ip}$ and $\phi_{it}$ can be bound states
of their respective nuclei,
or they may be discrete representations of continuum levels.
In the former case we have a `bound state approximation',
and in the second case we have a `coupled discrete continuum channels'
\cite{CDCC,CDCC2} (CDCC) approximation.
The states $\phi_i$ can be in different mass partitions, or they
can be different excited states of the projectile and/or the target in any
one of the partitions.
What is essential to the CRC framework is that there be a finite set ($N$
say) of square-integrable basis states, as this leads to a finite set of
equations coupling the channel wave functions $\psi_i (\vecR _
i)$ as unknowns.

For a complete Hamitonian $\overline{{\cal H} }$ and total energy $E$,
Schr\"odinger's equation $ [ \overline{{\cal H} } - E ]\overline{\Psi } = 0$
becomes $ [ {\cal H} - E ]\Psi = 0$ in the model space with \cite{fesh58}
\begin{eqnarray} \label{Fesh}
 {\cal H} = P \overline{{\cal H} } P -
  P \overline{{\cal H} } Q {1 \over {Q \overline{{\cal H} } Q - E - i \epsilon}} ~
  Q \overline{{\cal H} } P ,
\end{eqnarray}
%%.rc 4 on
where $Q \equiv 1 - P$ and $\epsilon$ is a positive infinitesimal quantity whose presence ensures that
the excluded channels have a time-retarded propagator, and hence only
{\em remove} flux from the model space.
%%.rc 4 off
The second term as a whole describes the effects of the excluded channels on
the model subspace $ P\overline{\Psi } $.  These effects could be, for example,
from compound nucleus formation, which we have excluded from explicit
consideration within direct reaction theory.
In the absence of detailled knowledge of these effects, we construct our
model Hamiltonian ${\cal H}$ using effective potentials which we believe
approximate (in some average manner) the processes described by equation
(\ref{Fesh}). The effective potentials will often be optical potentials with
real and imaginary components fitted to some simpler kinds of reactions,
and the effects of compound nucleus formation on these potentials is to
contribute to their imaginary component.

The model Hamiltonian ${\cal H}$ for the CRC system can now be projected
onto the individual basis states $\phi_i$. If $ E_i$
is the asymptotic kinetic energy in the $i$'th channel, then the
channel-projected Hamiltonian $ H_i $ satisfies
\begin{eqnarray}
H_i - E_i =\langle\phi_i | {\cal H} - E |\phi_i\rangle
\end{eqnarray}
and will be composed of a kinetic energy term and a diagonal optical
potential.  The `interaction potential' $V_i$ is then
defined to be everything in ${\cal H}$ not included in $ H_i$, so
\begin{eqnarray}
H_i - E_i + V_i = {\cal H} - E .
\end{eqnarray}
This construction gives $ V_i$ which have vanishing diagonal matrix
elements $\langle\phi_i | V_i |\phi_i\rangle = 0 $.
%%%
\subsection{Coupled Equation Set for $N$ bound state pairs}

If we take the model Schr\"odinger's equation
$ [ {\cal H} - E ]\Psi = 0, $ and project separately onto the
different basis states $\phi_i , $ we derive the set of equations
\begin{eqnarray} \label{CRC1}
\left [ E_i - H_i \right ]\psi_i ( \vecR_i ) =
   \sum _ {j \neq i}
         \left\langle\phi_i | {\cal H}-E |\phi_j \right\rangle
         \psi_j ( \vecR_j ) .
\end{eqnarray}
which couple together the unknown wave functions $\psi_i (\vecR_i).$
The matrix element $\langle\phi_i | {\cal H} - E |\phi_k\rangle $
has two different forms, depending on whether we expand
\begin{eqnarray}
{\cal H} - E & =& H_i - E_i + V_i \mbox{ (the `post' form)} \nonumber \\
             & =& H_j - E_j + V_j \mbox{ (the `prior' form)}.\nonumber
\end{eqnarray}
Thus
\begin{eqnarray} \label{popr}
 \left\langle\phi_i | {\cal H} - E |\phi_j \right\rangle & = &
   V_{ij}^{\rm post} + [H_i - E_i ] K_{ij} {\rm (post)} \\
  {\rm or } & =& V_{ij}^{\rm prior} + K_{ij} [H_j - E_j ]
                      {\rm (prior)}  \nonumber
\end{eqnarray}
where
\begin{eqnarray}
  V_{ij}^{\rm post}  \equiv\langle\phi_i | V_i |\phi_j\rangle , ~~~
   V_{ij}^{\rm prior}   \equiv\langle\phi_i | V_j |\phi_j\rangle , ~~~
   K_{ij}              \equiv\langle\phi_i | \phi_j\rangle .
\end{eqnarray}
The overlap function $K_{ij} =\langle\phi_i |\phi_j\rangle $ in
equation (\ref{popr}) arises from the well-known non-orthogonality between the
basis states $\phi_i $ and $\phi_j $ if these are in
different mass partitions. We will see below that this term disappears in
first-order DWBA, and can be made to disappear in second-order DWBA, if
the first and second steps use the prior and post interactions
respectively.
%%%
\subsection{$N$-step DWBA}

If the coupling interactions $ V_i $ in equation (\ref{popr}) are
weak, or if the back coupling effects of these interactions are already
included in the optical potentials of the prior channel, then it becomes
reasonable to use a distorted wave Born approximation (DWBA). This
approximation always feeds flux `forwards' in the sequence $ 1
\rightarrow 2 \rightarrow \cdots \rightarrow N+1 $ neglecting the back couplings. In
the elastic channel the wave function is governed by the optical potential
defined there, and the wave function in the $i$'th channel is
governed by the equation
%%.rc 4 on
\begin{eqnarray}
\left [ E_i - H_i \right ]\psi_i ( \vecR_i ) =
   \sum _ {j=1} ^ {j=i-1}
         \left\langle\phi_i | {\cal H}-E |\phi_j \right\rangle
         \psi_j ( \vecR_j )
\end{eqnarray}
%%.rc 4 off
{\bf Initial channel:}
%%.rc 4 on
\begin{eqnarray} \nonumber
\left [ E_1 - H_1 \right ]\psi_1 ( \vecR_1 ) = 0
\end{eqnarray}
%%.rc 4 off
{\bf Second channel:}
%%.rc 4 on
\begin{eqnarray}
\left [ E_2 - H_2 \right ]\psi_2 ( \vecR_2 ) =
         \left\langle\phi_2 | {\cal H}-E |\phi_1 \right\rangle
         \psi_1 ( \vecR_1 )
\end{eqnarray}
%%.rc 4 off
If the {\em prior} interaction is used, the right hand side becomes
\begin{eqnarray}  \nonumber
    &=&\langle\phi_2 | V_1 |\phi_1\rangle\psi_1
         + \langle\phi_2 |\phi_1\rangle [H_1 - E_1 ]\psi_1 \\
    &=&\langle\phi_2 | V_1 |\phi_1\rangle\psi_1
          \mbox{ as } \psi_1 \mbox{ is on-shell.} \\
    &=& V_{21}^{\rm prior}                   \psi_1
\end{eqnarray}
{\bf Final channel: } ($c=N+1$)
%%.rc 4 on
\begin{eqnarray}
\left [ E_c - H_c \right ]\psi_c ( \vecR_c ) =
   \sum _ {j=1} ^{j=c-1}
         \left\langle\phi_c | {\cal H}-E |\phi_j \right\rangle
         \psi_j ( \vecR_j )
\end{eqnarray}
%%.rc 4 off
If the {\em post} interaction
had been used for all the couplings to this last channel, then
%%.rc 4 on
\begin{eqnarray}
\left [ E_c - H_c \right ]\psi_c ( \vecR_c ) =
   \sum_ {j=1} ^ {j=c-1}
          \langle\phi_c | V_c |\phi_j\rangle\psi_j
  + [H_c - E_c ]~   \sum_{j=1} ^ {j=c-1}
          \langle\phi_c |\phi_j\rangle   \psi_j
\end{eqnarray}
%%.rc 4 off
so
%%.rc 4 on
\begin{eqnarray}
\left [ E_c - H_c \right ] \chi_c ( \vecR_c ) =
   \sum_{j=1} ^ {j=c-1}
           V_{cj}^{\rm post}                    \psi_j
\end{eqnarray}
%%.rc 4 off
%%%           \langle\phi_c | V_c |\phi_j\rangle\psi_j
where
\begin{eqnarray} \nonumber
   \chi_c ( \vecR_c ) & =&\psi_c
         + \sum_{j=1} ^ {j=c-1} \langle\phi_c |\phi_j\rangle\psi_j \\
    &=&\langle\phi_c |\Psi\rangle  \nonumber
\end{eqnarray}
Note that, as all the $\phi_j$ are square-integrable
and hence decay faster than $r ^ {-1}$ at large radii,
the $\psi_c$  and $\chi_c $
are the same asymptotically.
They differ only by an `off-shell transformation',
and hence yield the same (on-shell) scattering amplitudes.
The equation for $\chi_c$ has no non-orthogonality terms once
the {\em post} interaction is used in the final channel:
this is what is meant by saying that the final channel is
`effectively on-shell'.

These results imply that in $N$-step DWBA, some non-orthogonality terms
can be made to
disappear if `prior' interactions are used for the first step, and/or if
`post' interactions are used for the final step.  This means that the
non-orthogonality term never appears in the first-order DWBA, irrespective
of the choice of prior or post forms.  In second-order DWBA, the
prior-post combination must be chosen \cite{uda73}
to avoid the non-orthogonality
terms.  It should be also clear that non-orthogonality terms will have to
be evaluated if the DWBA is continued beyond second order.
%%%
\subsection{Full CRC solution by iteration}
\label{nono}
%.se nono = &@xref
%%%% Nono section is &nono

There are a number of different ways of solving the CRC equations with the
non-orthogonality terms present: for discussions of different approaches see refs.
\cite{iman74}, \cite{CV} and the survey of ref.[ch. 3]\cite{SATCH83}.

There are schemes available which can
iterate all channels with an arbitrary choice of post or
prior interactions for all the couplings.
Define
\begin{eqnarray}
\theta_{ij} & = & \mbox{ 0 or 1 : presence of post on the }
                                 j \rightarrow i \mbox{ coupling,}
 \\
 \mbox{ so } 1 - \theta_{ij} & = &\mbox{ 1 or 0 : presence of prior}.
\end{eqnarray}

The following iterative scheme
\cite{OSAKA}
($n$=1,2,..) on convergence then solves the CRC equations (\ref{CRC1}):\\

{\bf For $n$ =0}, start with
\begin{eqnarray}
\psi_i^{(0)} & = &\delta(i,i_0)\psi_{\rm elastic}
\\
  \delta S_i^{(0)} & = &\delta\psi_i^{(0)} = 0
\end{eqnarray}
{\bf For $n =1\rightarrow N+1$} (for $N$-step DWBA) solve
\begin{eqnarray}
 [ H_i - E_i ] \chi_i^{(n)} + S_i^{(n-1)} = 0
\end{eqnarray}
with
\begin{eqnarray}
S_i^{(n-1)} = \sum _ j
   [ \theta_{ij} V_{ij}^{\rm post} + (1 - \theta_{ij} ) V_{ij}^{\rm prior} ] ~
        \psi_j^{(n-1)} - \delta S_i^{(n-1)}
\end{eqnarray}
then calculate for subsequent iterations
\begin{eqnarray}
 \delta\psi_i^{(n)} &=& \sum _ j  \theta_{ij}
                            \langle\phi_i |\phi_j\rangle\psi_j^{(n-1)}
\\
 \delta  S _i^{(n)} &=& \sum _ j  (1-\theta_{ij})
                            \langle\phi_i |\phi_j\rangle
                    [ S_j^{(n-1)} + [H_j - E_j ] \delta\psi_j^{(n)} ]
\\
\psi_i^{(n)} &=& \chi_i^{(n)} - \delta\psi_i^{(n)}
\end{eqnarray}
This scheme avoids numerical differentations except in an higher-order
correction to $\delta S_i $ that arises in some circumstances.

When the non-orthogonality terms are included properly, it becomes merely
a matter of convenience whether post or prior couplings are used,
for one, two, and multistep calculations.
The equivalence of the two coupling forms can be confirmed in practice
(see, for example, refs.\cite{OSAKA}, and \cite{lill83}),
and used as one check on the accuracy of the numerical methods employed.
%%%
\section{Wave Functions for Scattering and Bound States}
%%%
%%.rc 4 on

In order to describe details of the nuclear transitions realistically,
it is necessary to specify in sufficient detail the initial and final
states of the nuclei involved. To start with, the excitation energies,
spins and parities of all the states in each mass partition need to be
specified, along with the nuclear masses, charges and relative
$Q$-values of the partitions.  Each {\em pair} projectile and target
excited states is then a distinct channel with its own scattering wave
function and boundary conditions. The initial projectile and target states
will select one such channel as the `incoming channel', with its
boundary conditions specifying an incoming plane wave. All channels
(including the incoming channel) will have outgoing spherical waves.
Particular attention must be given to the consistent
placement of $ i^L $ factors in these definitions.

The individual nuclear states are then specified in sufficient
detail for the particular reaction mechanisms involved. It is not
necessary to specify the full quantum mechanical states of all the
nucleons in the nucleus, but rather, only the states of those changed in the
reactions being considered. In particular, one and two-nucleon wave
functions will have to be described, if those nucleons are to be
transferred to other nuclei.  If a nuclear state consists of a particle of
spin $s$ bound outside a nucleus with possible core states
$\phi_I $, then the bound state radial wave functions
$ u_{\ell s j I} (r) $ will have to be found by solving a
coupled-channels set of equations for negative energy eigen-solutions. If
the particle is {\em not} bound, in the other hand, then its continuum range of
energies must be discretised into a finite collection of `bin'
states which can be scaled to unit normalisation.
If the nuclear state consists of two particles of intrinsic spins
$ s_1 $ and $ s_2 $ outside a core,
then it is usually specified by a
shell-model or by a Sturmian-basis calculation in terms of the independent
coordinates $ \vecr_1 $ and $ \vecr_2 $.
To calculate transfer rates, however, the two-particle wave
functions need to be given in terms of the collective coordinates
(usually $ \vecr =\half ( \vecr_1 + \vecr_2 ) $ and
$ {\brho} = \vecr_1 - \vecr_2 $).
In order to use the states in a reaction calculation, therefore, equations are given
for the transformation from the independent coordinates.

When we have calculated the scattering wave functions, or at least their
asymptotic parts in terms of their $S$-matrix elements, we can find the
cross sections for each outgoing pair of projectile and target states in
each partition. Furthermore, if the initial projectile has non-zero spin
$ J_p $, then the effect on these cross sections of polarisation
of the projectile is specified by the tensor analysing powers
$ T_{kq} $ (for $ 1 \leq k \leq 2J_p $
and $ 0 \leq q \leq k $).
Integrated cross sections and fusion polarisations can also be found using
the $S$-matrix elements.
%%.rc 4 off
%%%
%%% \subsection{Total wave function $\Psi$}
\subsection{Total wave function}

In each partition $\kappa$ of the system into a projectile of mass
$ A_{\kappa p}$ and a target of mass $ A_{\kappa t} $,
the coupling order is
\begin{eqnarray} \label{spinrep}
  \vecL + \vecJ_p = \vecJ ;~~~ \vecJ + \vecJ_t =  \vecJ_T ,
\end{eqnarray}
which may be defined by writing
\begin{eqnarray}
 \Psi_{\kappa J_T}^{M_T} =
  \left | ~(L~J_p )J, J_t ; J_T \right\rangle
\end{eqnarray}
where
$ J_p$
= projectile spin,
$ J_t$
= target spin,
$ L$
= orbital partial wave, and
$ J_T$
= total system angular momentum.

The set $\{\kappa pt,~(L~J_p )J, J_t ; J_T \} $
will be abbreviated by the single variable $\alpha$.
Thus, in each partition,
\begin{eqnarray} \nonumber
\Psi_{\kappa J_T}^{M_T}
   \left ( \vecR_\kappa , \xi_p , \xi_t \right )=
 \sum_{\begin{array}{cc}L J_p J J_t\\M \mu_p M_J \mu_t\end{array}}
  & \phi_{J_p} ( \xi_p )~\phi_{J_t} ( \xi_t )~
      i^L Y_L^M ({\hat \vecR_\kappa} ) {1 \over {R_\kappa}} ~
        f^{~\kappa J_T}_{(LJ_p )J,J_t} (R_\kappa)
\\
&  \left\langle LM J_p \mu_p | JM_J \right\rangle
        \left\langle JM_J  J_t \mu_t | J_T M_T \right\rangle\label{Psieq}
\end{eqnarray}
here $ \vecR_\kappa $
= radial coordinate {\em from} the target {\em to} the projectile
in partition $\kappa$, $ \xi_p$
= internal coordinates of projectile,
$ \xi_t$
= internal coordinates of target, and
\begin{eqnarray}
f^{~\kappa J_T}_{(LJ_p )J,J_t} (R)
 \equiv f _\alpha (R)
\end{eqnarray}
are the radial wave functions.
The $i^L$ factors arise from the spherical Bessel expansion of the
incoming plane wave.  Some formalisms include extra powers of $i$
in the equation (\ref{Psieq}), in order to make the coupling interactions
$V_{\alpha :\alpha'} $ real.
Inelastic coupling interactions can be made real (for integer-valued spins,
at least), by including a factor $i^{J_p + J_t} $ in the
definition of $\Psi$, and transfer couplings can be made real by including
a factor $i^ \ell $ for {\em orbital} angular momentum
$\ell$ of the bound particle state in this partition.
In a general purpose code \cite{FRESCO}, however,
there may be clashes between these different conventions. The ground state
of $^7$Li, for example, would have a factor of $i^ {3/2} $
on the rotational model convention, but a factor of $ i^1 $
if the state were regarded as a $ \ell = 1 $ bound state of a $\alpha$ core
and a triton cluster.
It seems simplest, therefore, to omit these addition phase factors completely.
The coupling interactions can very often be made real, nevertheless,
if the $ i ^ L $ factors are included explicitly in the CRC equations,
as in the next section.

The wave function $\Psi$ could also have been defined using the `channel
spin' representation (as in \cite{OUKID})
$\Psi = | L, (J_p J_t)S; J_T\rangle $,
which is symmetric upon projectile $\rightarrow$ target interchange except for
a phase factor $ (-1)^{S - J_p - J_t} $.
This would simplify the subsequent description of the coupling elements in
section 4, as the formulae for projectile mechanisms and target mechanisms
would differ only by this phase factor.  However, the channel spin
representation has the disadvantage that the projectile spin-orbit force
is not diagonal in this basis.  This would not matter if coupled-channels
solutions were always sought, but one of the advantages of sometimes
solving the CRC equations iteratively is that the DWBA solutions of first and
second order (etc.) may be obtained. In order for the partially-iterated
CRC solutions to reproduce the results of DWBA codes, it is necessary to
treat spin-orbit forces without approximation, and since spin-orbit forces
almost always are those of the projectile, the asymmetric representation
of channel (\ref{spinrep}) is advisable.
%%%
\paragraph{Identical Nuclei}

If one partition ($\kappa$ say)
is identical to another $\kappa$ except that the projectile
and target nuclei are exchanged,
then the total wave function should be formed from
$(1 + \pi P_{\kappa\kappa'}) $ times the above expression,
where $\pi =\pm1$ is the intrinsic parity of the two nuclei under
exchange.
A simple method of dealing with this exchange is to first form the
wave function of equation (\ref{Psieq}), and then operate with
$(1 + \pi P_{\kappa\kappa'}) $
on both the wave functions and the S-matrix elements,
before cross sections are calculated.
This is equivalent to the replacement
\begin{eqnarray}
{f _\alpha (R) \leftarrow f _\alpha (R) + c_{\alpha ,\alpha'} f_{\alpha'} (R) }
\end{eqnarray}
where
\begin{eqnarray}
c_{\alpha ,\alpha'}  =
     \pi (-1)^L \delta_{L,L'} (-1)^{J_T + L - J - J'}
     \hat{J} \hat{J'} W(J_p L' J_T J_t ; J J')
\end{eqnarray}
with $\alpha' = | (L' J_t)J' , J_p ; J_T\rangle $.

\subsection{Coupled equations}
%.se RK = 'R_\kappa'
%.se RKP = ' ( R_\kappa )'
\newcommand{\RK}{{(R_\kappa)}}
\newcommand{\RKP}{{(R_{\kappa'})}}

The CRC equations are in many cases of the form
\begin{eqnarray} \nonumber
\left [ E_{\kappa pt } - T_{\kappa L } \RK - U_\kappa \RK \right ]
       f_\alpha \RK  &=& \sum_{\alpha' , \Gamma> 0 }
           { i ^ {L' - L} ~
           V^\Gamma_{\alpha :\alpha' } \RKP  f_{\alpha'} \RKP }
\\
  &+ & \sum_{\alpha' ,\kappa' \neq\kappa } i ^ {L' - L} ~
   {\int_ 0 ^ {R_m }
     { V_{\alpha :\alpha' }( \RK , {R_{\kappa'}} )
          f_{\alpha'} (R_{\kappa'})
            d R_{\kappa'} }}\label{CRC}
\end{eqnarray}
where the kinetic energy term is
\begin{eqnarray}
   T_{\kappa L} (R) = - {\hbar^2 \over {2 \mu_\kappa}} ~
       \left ( {d^2 \over dR^2} - {{L(L+1)} \over R^2} \right ) ,
\end{eqnarray}
$U_\kappa (R_\kappa)$ is the diagonal optical potential
with nuclear and Coulomb components,
%%.rc 4 on
and $R_m$ is a radius limit larger than the ranges of
$U_\kappa (R_\kappa)$ and of the coupling terms.
%%.rc 4 off
The equations (\ref{CRC}) are in their most common form:
they become more complicated when
non-orthogonalities are included by the method of section \ref{nono}.
The $   V^\Gamma_{\alpha :\alpha' } \RKP $ are the local
coupling interactions of multipolarity $\Gamma$, and the
$ V_{\alpha :\alpha' }( R_\kappa , {R_{\kappa'}} ) $
are the non-local couplings between mass
partitions that arise from particle transfers.

For incoming channel $\alpha_0$, the
$ f_\alpha \RK$
satisfy the boundary conditions
\begin{eqnarray} \label{asymp}
 f_\alpha \RKP = _ {R_\kappa > R_m }
  {i \over 2} \left [ \delta_{\alpha\alpha_0 }
             H^{(-)}_{L {\eta _\alpha}} ( K_\alpha \RK )
    - S_{\alpha_0\alpha }
             H^{(+)}_{L {\eta _\alpha}} ( K_\alpha \RK )
      \right ]
\end{eqnarray}
where
$ H^{(-)}_{L \eta}$
and
$ H^{(+)}_{L \eta}$
are the Coulomb functions \cite{COULCC}
with incoming and outgoing boundary conditions
respectively.
The asymptotic kinetic energies are
\begin{eqnarray}
     E_{\kappa pt} = E + Q_\kappa - \epsilon_p - \epsilon_t
\end{eqnarray}
for excited state energies
$ \epsilon_p , \epsilon_t$
and Q-value $ Q_\kappa$
in partition $\kappa$, and
\begin{eqnarray}
   K_\alpha = \sqrt {\left [ {2 \mu_\kappa} \over {\hbar^2}
                      E_{\kappa pt}  \right ] }
                      \end{eqnarray}
where
$
\mu_\kappa =
  A_{\kappa p} A_{\kappa t} / (A_{\kappa p} +  A_{\kappa t})
$
is the reduced mass in the channel with partition $\kappa$,
and
\begin{eqnarray} \label{EQeta}
  \eta _\alpha = {{2 \mu_\kappa} \over \hbar^2} ~
            { Z_{\kappa p} Z_{\kappa t} e^2  \over 2 K _\alpha }
\end{eqnarray}
is the Sommerfeld parameter for the Coulomb wave functions.
%%%
\subsection{Single-nucleon states}

If $\phi_{JM} ( \xi ) $ is a core+particle bound state, then for coupling
order $ \left | ( \ell s)j,I;~JM \right\rangle , $
the wave function is
\begin{eqnarray}
 &&\phi_{JM} ( \xi_c , \vecr) =
   \sum_{\ell jI}  A_{\ell sj}^{jIJ} ~
     \left [\phi_I (\xi_c)  \varphi_{\ell sj} (\vecr) \right ]
     _{JM}
\\
   &=&
   \sum_{ {\ell jI} ,{m \mu m_s m_\ell}}
     A_{\ell sj}^{jIJ} \left\langle jmI \mu | JM \right\rangle
    \phi_{I \mu} (\xi_c )
              \left\langle \ell m_\ell s m_s | jm \right\rangle
        Y_\ell^{m_\ell} ( \hat \vecr)\phi_s^{m_s}
        \frac{1}{r} u_{\ell sjI} (r) \nonumber
\end{eqnarray}
where $ \xi_c $ = core internal coordinates,
      $\phi_{I mu} (\xi_c ) $ = core internal state,
      $\phi_s^{m_s} $ = particle internal spin state,
      $ u_{\ell sjI} (r) $ = particle core radial wave function,
and
$A_{\ell sj}^{jIJ}$ is the coefficient of fractional parentage.

\subsubsection{Bound States}

If the single-particle is bound at negative energy $E$ around the core,
then its wave function may be found as the eigen-solution of a
given potential:
%%.rc 4 on/off
\begin{eqnarray}
 [ T_\ell (r) + V (r) + \epsilon_I - E ]  u_{\ell sjI} (r)
    + \sum_{\ell' j' I', ~\Gamma>0 }
       V^\Gamma_{\ell sjI : \ell' s j' I' } (r)
            u_{\ell' s j' I'} (r) = 0
\end{eqnarray}
with boundary conditions $ u_{\ell sjI} (0) = 0 $ and,
as $ r \geq R_m $, of
$ u_{\ell sjI} (r) \propto W_{\ell} (k_I r) $
where $ W_\ell (\rho) $ is the Whittaker function
and $ k_I^2 = { 2 \mu |E - \epsilon_I | / \hbar^2 } $
is the asymptotic wave number.

If the core cannot be excited, then these coupled equations reduce
to one uncoupled equation,
but solving this equation can still be regarded as a special case of the
coupled bound state problem.
Eigen-solutions can be found by solving either for
the bound state energy $E$, or by varying the depth of the binding potential.
In general, however, we can choose to vary any multipole of any part of
the binding potentials (except the Coulomb component), so one method
of solving the full coupled bound-state problem will be outlined below.

To define the phase ($\pm 1$) of the overall wave function, some
convention has to be adopted.
One component (say that around a core $I=0$ state)
can be set to either positive towards the origin ($r\rightarrow 0$),
or positive towards large distances ($r\rightarrow \infty$).
The former choice is made in the FRESCO code,
following the Mayer-Jensen phase convention,
which is also used for harmonic oscillator
wave functions in many structure calculations.
%%%
\subsubsection{Solution of the Coupled-Channels Eigenvalue Problem}

When, for example, the problem is to find the bound state of a particle in
a deformed potential, then
several channels with different angular momenta will be coupled together.
There are various techniques for calculating the wave functions of these bound states:
for a review see ref. \cite{ogle71}.
The Sturmian expansion method \cite{Sturm}
can be used, or the coupled equations can be solved iteratively.
The Sturmian method has the advantage that {\em all} solutions in the
deformed potential are found, where sometimes the
iterative method has difficulty in converging to a particular solution if there
are other permitted solutions near in energy.
%%.rc 4 on
The iterative method has the advantage that the radial wave functions
(once found) are subject only to the discretisation error for the
Schr\"odinger's equation, and are not dependent on the
(time-consuming) diagonalisation
of large matrices, often of the order of 100 or more.
As they satisfy the correct boundary conditions independently of the size
of a basis-state set,
the radial wave functions of the iterative method therefore more accurately reflect
the details of the coupling potentials and of the core excitation energies.
As nuclear reactions are often confined to the surface region,
it is important to satisfy the exterior boundary conditions
as accurately as possible.
%%.rc 4 off

A method for solving the uncoupled eigenstate problem has to be
included in a reaction code in any case,
and since it can be generalised as described in this section
to solving the coupled problem, it seems a worthwhile facility to have available.
Bound states from a previous Sturmian solution can still be included as explicit
linear combinations of the single particle (uncoupled) basis states used in the
Sturmian expansion.

The general problem of finding eigen-solutions of a set $M$
coupled-channels equations can be represented as the problem of finding
$\lambda$ such that the equations
\begin{eqnarray} \label{bseq}
\left [ {d^2 \over dr^2} -
        {\ell_i (\ell_i + 1) \over r^2} \right ]
 \psi_i (r)
 + \sum _ {j=1} ^M \left [ U_{ij} (r) + \lambda V_{ij} (r) \right ]
  \psi_j (r) = 0
\end{eqnarray}
with boundary conditions
\begin{eqnarray}
 \psi_i (R) &=& a_i W_{\ell_i \eta_i} (k_i R)  \\
 \psi_i (R + \delta R) &=& a_i W_{\ell_i \eta_i} (k_i
                              (R + \delta R))\\
 \psi_i (0) &=& 0
\end{eqnarray}
(with $k_i^2 \equiv\kappa^2_i + \theta \lambda$ and
 $\eta_i \equiv {nu_i} / (2 k_i)$)
for given partial waves $\ell_i ,$ fixed potentials
$U_{ij} (r),$ variable potentials $V_{ij} (r),$
matching radius $R$,
and Coulomb proportionality constants $\nu_i$.
%%.rc 4 on
The energy constants $\kappa^2_i$ are the asymptotic components
of the diagonal $U_{ii} (r) $, and $\theta$ is the asymptotic component
of the diagonal $V_{ii} (r) $ (assumed all equal).
%%.rc 4 off

The solution begins by constructing the trial integration functions for a
trial value of $\lambda$, on either side of an intermediate matching point
$r = \rho $:

$f_{i;j}^{\rm in} (r)$ by integrating $r$ from $h$ to $\rho$,\\
starting with $f_{i;j}^{\rm in} (h) = \delta_{i,j} ~
  h^{\ell_i + 1} / (2 \ell_i + 1)!! $,
and

$f_{i;j}^{\rm out} (r)$ by integrating $r$ from $R$ in to $\rho$,\\
starting with $f_{i;j}^{\rm in} (R) = \delta_{i,j} ~
  W_{\ell_i \eta_i} (k_i (R + \delta R)). $

The intermediate point $r = \rho $ should be chosen where the
wave functions are oscillatory, to avoid having to integrate outwards in
the classically forbidden region.

The solution is therefore
\begin{eqnarray} \label{eigencomp}
\psi_i (r) = \left \{ \begin{array}{ll}
                    \sum _ j  b_j f_{i;j}^{\rm in} (r)
                         \mbox{  for } r < \rho \\
                    \sum _ j c_j f_{i;j}^{\rm out} (r)
                         \mbox{  for } r \geq \rho ,
                      \end{array} \right. ~
\end{eqnarray}
and the matching conditions are the equality of the two expressions and
their derivatives at $ r = \rho $. The normalisation is still
arbitrary, so we may fix $ c_1 = 1$. In general the
equations (\ref{bseq}) have no solution as $\lambda$ is not exactly an eigenvalue.
The method therefore uses the discrepancy in the matching conditions to
estimate how $\lambda$ should be changed to $\lambda+\delta\lambda$ to
reduce that discrepancy, and iterates this process to reduce $\delta\lambda$.

Thus at each iteration we first solve as simultaneous equations the
2$M$-1 of the matching conditions
\begin{eqnarray}
      \sum _ j  b_j f_{i;j}^{\rm in} (\rho)
&=& \sum _ j  c_j f_{i;j}^{\rm out} (\rho) \mbox{  for all } i
\\
      \sum _ j  b_j f_{i;j}^{\rm in} (\rho)'
&=& \sum _ j  c_j f_{i;j}^{\rm out} (\rho) \mbox{  for all } i \neq 1
\end{eqnarray}
along with $c_1 =1$ for the 2$M$ unknowns $b_i , c_i$. If the function
$\psi_i (r) $ is then constructed using equation (\ref{eigencomp}),
there will be a discrepancy as
\begin{eqnarray}
\psi'_ {\rm in}  \equiv\psi_1 (r)  | _ {r< \rho}  ~~~ \neq ~~~
\psi'_ {\rm out} \equiv\psi_1 (r)  | _ {r> \rho} ,
\end{eqnarray}
and this difference will generate $\delta \lambda$ via
\begin{eqnarray}
\delta \lambda \sum _ {ij} ^ M
   \int_ 0 ^ R \psi_i (r) V_{ij} (r)\psi_j (r) dr
=\psi_1 (\rho) [ \psi'_ {\rm out} -\psi'_ {\rm in} ].
\end{eqnarray}

It is necessary while iterating in this manner to monitor the number of
nodes in one or more selected components of the wave function,
as in general a given potential will have different eigensolutions with
different numbers of radial nodes.  When the iterations have converged to
some accuracy criterion on the size of $\delta\lambda$, the set of wave
functions can be normalised in the usual manner:
\begin{eqnarray}
   \sum _ i ^ M \int _ 0 ^ \infty
      | \psi_i (r)| ^2 dr  = 1
\end{eqnarray}
and perhaps some of the components $i$ omitted if their contribution to
this norm is below some preset threshold.

%%%
\subsubsection{Continuum States}
%%%
%%%

If the initial and/or final single-particle states of a transfer step
are unbound $E - \epsilon> 0$, the use of single energy eigenstates
$\phi_k (r)$ will result in calculations of the
transfer form factors which will not converge, as the continuum wave
functions do not decay to zero as $ r \rightarrow \infty$
sufficiently fast as to have square norms.
One way \cite{CDCC}, \cite{CDCC2}
of dealing with this divergence is to take continuum states
not at a single energy, but averaged over a range of energies.
These `bin' states that result {\em are} square integrable,
and if defined as
\begin{eqnarray} \label{binwf}
 \Phi (r) &=& \sqrt {2 \over \pi N} ~~
            \int_ {k_1} ^ {k_2} w(k)\phi_k (r) dk
\\
\mbox{ with } N &=& \int_ {k_1} ^ {k_2} |w(k)| ^ 2 dk
\end{eqnarray}
for some weight function $w(k)$,
then they are normalised $\langle\Phi |\Phi\rangle = 1 $
provided a sufficiently large maximum radius for $r$ is taken,
and that the
$\phi_k$ are eigensolutions of a potential which is
energy-independent.
They are orthogonal to any bound states, and are orthogonal to other bin
states if their energy ranges do not overlap.
The construction can be easily generalised to give coupled-channels
bin wave functions.

The weight function $ w(k) $ is best chosen
(\cite{CDCC2} p. 148)
to include some of the effects known to be caused by the variation of
$\phi_k (r) $ within the bin range
$ k_1 \leq k \leq k_2 $.
If $ w(k) = \exp ( - i \delta_k ) $, where $\delta_k$
is the scattering phase shift for $\phi_k (r) $,
then it includes the effects of the overall phase variations
of $\phi_k  $, at least in the
DWBA limit.
If, however,
$ w(k) = \exp ( - i \delta_k ) \sin {\delta_k}
         \equiv T_k^* $, where $T_k$
is the $T$-matrix element for $\phi_k (r) $,
then it includes in addition a scale factor which is useful if the
$ | T_k | $ varies significantly, as it does, for example, over
resonances. Both choices result in a real-valued wave function
$\Phi (r) $ (for real potentials), which is computationally
advantageous.
%%.rc 4 off

If the maximum radius ($R_m $ say) is not sufficiently large,
then the wave functions $\Phi$ will not be normalised to unity by the factors
given in equation (\ref{binwf}).
The rms radius of a bin wave function increases as the bin width
$k_2 - k_1 $ decreases, approximately as
$1/(k_2 - k_1)$.
These bin constructions can be used to describe the narrow resonant wave
functions of say the $3^+$ state in $^6$Li,
or the $ 7/2^-$ state in $^7$Li,
but these states will require a large limiting radius $R_m $
unless the
$ w(k) = T_k^* $ weighting factor is used to emphasise
the increase in the interior wave function over the resonance.
The $ 3^+ $ state in $^6$Li at 0.71 MeV, for example, for which the
resonance width is approximately 40 keV, yields the normalisations shown in
\ref{bins}. It can be seen that without a scale factor which
emphasises the resonance peak, very large radii $ R_m$ will be
needed to obtain unit normalisation.

\begin{figure}
\label{bins}
\begin{tabular}{c|cccc}
$R_m$ & $\delta E = 0.1$ MeV &&$\delta E = 0.5$ MeV& \\
(fm)& $e^{-i \delta_k}$& $T_k^* $ & $e^{-i \delta_k}$ & $T_k^* $\\
\hline
10&0.105&0.923&0.108&0.977\\
20&0.109&0.939&0.132&0.985\\
40&0.112&0.951&0.258&0.993\\
80&0.122&0.971&0.411&0.996\\
160&0.142&0.986&0.614&0.998\\
\hline
\end{tabular}
%%% \caption{Normalisations &\|.$\Phi$.&\| of a continuum bin state.
\caption{Normalisations of a continuum bin state.
For this $ 3^+ $ state in $^6$Li at 0.71 MeV,
Saxon-Woods potentials were used with $V$ = 77.05 MeV,
$R$ = $R_c$ = 1.2 * 4$^{1/3}$ fm., and $a$ = 0.65 fm.}
\end{figure}
%%.rc 4 off
%%%
\subsection{Two-particle bound states}
\label{NNBS}
%.se NNBS = &@xref
%%%
\subsubsection{Centre-of-mass coordinates}

If $\phi_{JT} (\xi_c , \vecr , {\brho} ) $ is a two-particle
bound state with total spin {\bf J} and isospin $T$, then for coupling order
$ \left | \{~L , (\ell , (s_1 s_2) S ) j \}
           J_{12} , I ;~J \right\rangle $
we have
\begin{eqnarray} \nonumber
  \phi_{J M} =
    \sum_{\begin{array} {ccc} L  \ell S\\j  J_{12} I\end{array}}
     && A_{L j J_{12}}^{J_{12} I J} ~
       \phi_{I \mu_I} (\xi_c)  .
       \phi_{s_1}^{\sigma_1} ~\phi_{s_2}^{\sigma_2} ~
        Y_{L}^{\Lambda} ( \hat\vecr ) ~
        Y_\ell^\mu (  \hat {\brho} ) ~
        {1 \over r \rho} u_{12} (r, \rho)
\\  \label{tntwf}
    &&\langle J_{12} M_{12}  I \mu_I | J M\rangle
       \langle L \Lambda j m_{12} | J_{12} M_{12}\rangle
       \langle \ell \mu S \Sigma         | j m_{12}\rangle
       \langle s_1 \sigma_1 s_2 \sigma_2  | S \Sigma\rangle
\end{eqnarray}
where
$ A_{L j J_{12}}^{J_{12} I J} $ is the coefficients of fractional parentage,
 and $\phi_{s_1}^{\sigma_1} ~\phi_{s_2}^{\sigma_2}$ are the
intrinsic spins of the two particles.

Note that two neutron transfer can be viewed as the transfer of a
`structured particle'
$  ( \ell , (s_1 s_2) S ) j  $, and then becomes similar to single-particle transfers of above.

The radial wave function $u_{12} (r, \rho) $
can be given either as a cluster product of single-particle wave functions
$ u_{12} (r, \rho)  =\Phi_L (r)\phi_\ell (\rho), $
or input directly as a two-dimensional distribution e.g. from a Faddeev bound-sate
calculation, or calculated from the
correlated sum of products of single-particle states, as in the next section.
\subsubsection{Independent Coordinates}

Two-particle states from shell-model calculations or from Sturmian-basis
calculations \cite{vaag79}, and are then usually described by means of the
$ | \vecr_1 , \vecr_2\rangle $
coordinates, and then transformed internally into the centre-of-mass coordinates
$ | \vecr ,  {\brho} \rangle $ of equation (\ref{tntwf})
using $ \vecr_i = x_i \vecr + y_i {\brho} $.
%%.rc 4 on
For equal mass particles, $x_1 = x_2 = 1 $, and
$y_1 = - y_2 =\half $.
%%.rc 4 off
The second description is as
\begin{eqnarray}
  \varphi_{12} ( \vecr_1 , \vecr_2 ) &=&
   \sum _ i c_i ~
   \left | ( \ell_1 (i), s_1 )j_1 (i),
            ( \ell_2 (i), s_2 )j_2 (i);~ J_{12} T \right\rangle
 \\
& &\rightarrow %|_{\mbox{ coordinate transformation}}
  \sum _ u {c_i} \sum_{ L \ell S j}
   \left | L , ( \ell , (s_1 s_2 )S )j ;
                         J_{12} T \right\rangle
 \phi^{J_{12} T,i}_{L(\ell S)j} (r, \rho)
\end{eqnarray}
The transformation of the $i$'th component in the cluster basis is
\begin{eqnarray}
\phi^{J_{12} T,i}_{L(\ell S)j} (r, \rho) &=&
  \left\langle L , ( \ell , (s_1 s_2 )S )j ;J_{12} T \right |
  \left ( \ell_1 (i), s_1 )j_1 (i),
       ( \ell_2 (i), s_2 )j_2 (i);~ J_{12} T \right\rangle
\\
&& \times\langle
   \left [  Y_L (\hat \vecr) Y_\ell (\hat \rho) \right ] _ \lambda
    |
   \left [ \varphi_{\ell_1 s_1 j_1} (\vecr_1)\varphi_{\ell_2 s_2 j_2} (\vecr_2)
   \right ] _ {J_{12} T}  \rangle
\end{eqnarray}
where (suppressing the $i$ indices for clarity)
\begin{eqnarray} \nonumber
\langle L , ( \ell , (s_1 s_2 )S )j ;J_{12} T |
       ( \ell_1 , s_1 )j_1 ,
       ( \ell_2 , s_2 )j_2 ;~ J_{12} T\rangle =
\end{eqnarray}
\begin{eqnarray}
 \sum _ \lambda && \hat \lambda \hat S \hat {j_1} \hat {j_2}
  \left ( \begin{array}{ccc}\ell_1 & \ell_2 & \lambda\\
                            s_1    & s_2    & S \\
                            j_1    & j_2    & J_{12} \end{array}  \right )
         {1 + (-1)^{\ell +S+T} \over
    \sqrt {2(1 + \delta_{\ell_1 , \ell_2} \delta_{ j_1 ,j_2}) }} ~
\hat j \hat \lambda W(L \ell J_{12} S; \lambda j) (-1)^{\ell+L-\lambda} .
\end{eqnarray}
%%.rc 4 on
The radial overlap integral can be derived by means of harmonic-oscillator
expansions \cite{mosh60}, with the Bayman-Kallio expansion \cite{bay67}
or using the Moshinsky solid-harmonic expansion\cite{MOSH}. This last method gives
%%.rc 4 off
\begin{eqnarray}
 K^\lambda_{\ell L: \ell_1 \ell_2} (r, \rho)
 &=&\langle
   \left [  Y_L (\hat\vecr) Y_\ell (\hat{\brho}) \right ] _ \lambda
    |
   \left [ \varphi_{\ell_1} (\vecr_1)\varphi_{\ell_2} (\vecr_2)
   \right ] ^ \lambda \rangle \nonumber
\\
 &=  &\sum_{n_1 n_2} ~
      \left ( \begin{array}{c} 2 \ell_1 +1\\2n_1\end{array} \right ) ^\half
      \left ( \begin{array}{c} 2 \ell_w +1\\2n_2\end{array} \right ) ^\half
      (x_1 r)^{\ell_1 - n_1}
      (y_1 \rho)^{n_1}
      (x_2 r)^{n_2}
      (y_2 \rho)^{\ell_2 - n_2} \nonumber
\\
&&\times \sum _ Q
      {\bf q}_{\ell_1 \ell_2}^Q (r, \rho) ~
     (2Q+1)~ \hat {\ell_1} \hat {\ell_2} \hat {\ell_1 - n_1} \hat {\ell_2 - n_2} ~
             \hat L \hat \ell  \nonumber
\\
&&\times  \sum_{\Lambda_1 \Lambda_2}
       \left ( \begin{array}{ccc}\ell_1 - n_1&n_2&\Lambda_1\\0&0&0\end{array} \right )
       \left ( \begin{array}{ccc}\ell_w - n_2&n_1&\Lambda_2\\0&0&0\end{array} \right )
       \left ( \begin{array}{ccc}\Lambda_1&L   &Q\\0&0&0\end{array} \right )
       \left ( \begin{array}{ccc}\Lambda_2&\ell&Q\\0&0&0\end{array} \right )  \nonumber
\\
&&\times (-1)^{\ell_1 + \ell_2 +L+ \Lambda_2} (2 \Lambda_1 + 1) (2 \Lambda_2 + 1)
        W(\Lambda_1 L \Lambda_2 \ell; Q \lambda)
        \left ( \begin{array}{ccc} \ell_1-n_1& n_2& \Lambda_1\\
                                   n_1       & \ell_2-n_2& \Lambda_2\\
                                   \ell_1    & \ell_2 & \lambda \end{array} \right )  .
\end{eqnarray}
%%.rc 4 on
where $ \left ( \begin{array}{c}a \\ b \end{array}\right ) $ is the binomial coefficient
(see Appendix A).
%%.rc 4 off

The kernel function
${\bf q}_{\ell_1 \ell_2}Q (r, \rho) $
which appears in this expression is the Legendre expansion of the product of
the two radial wavefunctions in terms of $u$,
the cosine of the angle between $\vecr$ and $ {\brho} $:
\begin{eqnarray}
 {\bf q}^Q_ {\ell_1 , \ell_2} (r, \rho)
 =\half
\int_ {-1}^{+1}
         {u_{\ell_1} (r_1) \over r_1}^{\ell_1 +1}
       {u_{\ell_2} (r_2)\over r_2}^{\ell_2 +1} ~
         P_Q (u) du
\end{eqnarray}

\subsection{Scattering Amplitudes}

%%.rc 5 on
The Rutherford amplitude for pure Coulomb scattering
(with no $ e^{2i \sigma_0} $ factor) is
\begin{eqnarray} \label{ruther}
F_c ( \theta ) = - {\eta \over 2k} ~~
               { \exp (-2 i \eta \ln(\sin \theta /2))  \over  \sin^2 \theta /2}
\end{eqnarray}
%%.rc 5 off
The Legendre coefficients for the scattering to the projectile state
$ J'_ p $ and target state $ J'_ t $
from initial projectile state $ J_p $
and target state $ J_t $ are given by
\begin{eqnarray}  \nonumber
 A^{L'}_{m' M' ;mM} &=&
   \sum_{L,J,J' ,J_T }
    \langle L0 J_p m | Jm\rangle\langle Jm J_t M |J_T M_T\rangle \\
&& \langle L' M_{L'} J'_ p m' | J' M_{L'} + m'\rangle
    \langle J' M_{L'}  +m' J'_ t M' |J_T M_T\rangle \nonumber\\
&&  \nonumber
  {4 \pi \over k} \sqrt {\frac{k'}{\mu'}\frac{\mu}{k}}
    e^{i( \sigma_L - \sigma_0 )}
      e^{i( \sigma'_ {L'} - \sigma'_ 0 )}\\
&&  \label{ALeg}
  \left ( {i \over 2} \right )
      \left [ \delta_{\alpha ,\alpha'} - S^{J_T}_{\alpha ,\alpha'} \right ]
    \sqrt {{2L+1 \over {4 \pi}}}~ Y_c (L' ,M_{L'})
\end{eqnarray}
where $Y_c (L,M)$ is the coefficient of
$P_L^{|M|} (\cos \theta )
    e^{iM\phi}$ in $Y_L^M (\theta ,\phi )$,
$  \sigma_L = \arg \Gamma (1 + L + i \eta )$ is the Coulomb phase shift,
$\alpha'$ refers to the primed values
$ L'  J'_ p J'_ t  k' \mu' $ etc.,
and $\alpha$ refers to the unprimed values
$ L J_p J_t k \mu $.

For each outgoing channel $J'_ p , J'_ t $,
we may then calculate the angular-dependent scattering amplitudes
\begin{eqnarray}
f_{m' M' : mM} (\theta) =
    \delta_{J_p , J'_ p}
    \delta_{J_t , J'_ t}
     F_c (\theta) +
  \sum_{L'} A_{m' M' : mM}^{L'}
                    P_{L'}^{m' +M' -m-M} (\cos \theta)
\end{eqnarray}
in terms of which the differential cross section is
\begin{eqnarray}
  {d \sigma(\theta) \over d \Omega} =
   {1 \over (2J_p + 1)(2J_t + 1) }
   \sum_{m' M' m M}
           \left | f_{m' M' : mM} (\theta) \right | ^2 .
\end{eqnarray}
The near-side and far-side decompositions \cite{berry66}
of this cross section are
defined by the same process, with $P^M_L (u)$ replaced by
$\half [P^M_L (u)\pm 2i / \pi  Q^M_L (u)]$ respectively.
The Coulomb scattering of equation (\ref{ruther}) is included
in the near-side component \cite{row78}.

The spherical tensor analysing powers $T_{kq}$
describe how the {\em outgoing}
cross section depends on the {\em incoming}
polarisation state of the {\em projectile}.
If the spherical tensor $\tau_{kq}$
is an operator with matrix elements
\begin{eqnarray} \nonumber
(\tau_{kq})_{m m''} =
  \sqrt{2k+1}\langle J_p m k q | J_p m''\rangle,
\end{eqnarray}
we have
\begin{eqnarray}
T_{kq} (\theta)
             &=& {Tr ({\bf f} \tau_{kq} {\bf f}^+)} \over
                    {Tr ({\bf f}              {\bf f}^+)}
\\
           &=& \hat{k}
           {   \sum_{m' M'  m  M}
                      f_{m' M' : m M} (\theta)^*
             \langle J_p m  k q | J_p m''\rangle
                      f_{m' M' : m''  M} (\theta)
                    \over
                     \sum_{m' M'  m  M}
                      | f_{m' M' : mM} (\theta) | ^2 }
\end{eqnarray}
%%.rc 4 on
The polarisations in the `transversity frame' \cite{simon74}
are then
%%.rc 4 off
\begin{eqnarray}
  ^T T_{10} &=& \sqrt 2 i T_{11}
\\
  ^T T_{20}  &=& -\half (T_{20} + \sqrt 6 T_{22})
\\
  ^T T_{30} &=& -\half ( \sqrt 3 i T_{31} + \sqrt 5 i T_{33} ) .
\end{eqnarray}

The S-matrix elements can also be used to directly calculate the
integrated cross sections
%%% \sigma = intplex c\int_ {4 \pi}
%%%            {~ {d \sigma(theta)} over {d Omega} d Omega  }
\begin{eqnarray}
\sigma = \int_ {4 \pi}
           {~ {d \sigma(\theta) \over d \Omega} d \Omega  }
\end{eqnarray}
to give
\begin{eqnarray}
\sigma = {1 \over k^2} {k' \over \mu'} {\mu \over k} ~
        {4 \pi \over (2J_p + 1)(2J_t + 1)}
        \sum_{J_T ~\alpha ~\alpha'}
         (2J_T + 1) \left | S^{J_T}_{\alpha ,\alpha'}
                        \right | ^2 .
\end{eqnarray}

The fusion cross section is defined as that amount of flux which leaves
the coupled-channels set because of the imaginary parts of the optical
potentials. If the incoming projectile is not spherical, then
the fusion rate will depend on its orientation, and hence on the magnetic
substate quantum number $m$.  One can therefore define the
{\em fusion polarisation} as the distribution
$ \sigma^{\rm fus}_m $, which can be calculated from the S-matrix
elements as
\begin{eqnarray} \nonumber
  \sigma^{\rm fus}_m  &=& {\pi \over k^2}
     \sum_{J_T \geq m} (2J_T + 1)  \\
 &&\times { \left [ 1 - {1 \over 2J_t + 1}
     \sum_{J'_ p L' M \omega} ~
  \left | \sum _ {LJ}
   \langle J_p m J - m | L 0\rangle
   \langle J m J_t M | J_T M + m\rangle
    e^{i \sigma_L} S^{J_T \omega}_{\alpha ,\alpha'}
  \right | ^ 2  \right ] }
\end{eqnarray}
where $\omega$ is the parity ($\pm 1$) of the coupled-channels set for each
total angular momentum $J_T$

{\em Partial Wave Interpolation:}
Heavy ion reactions typically involve a range of partial waves $L$ up to
several hundred or more, especially when Coulomb excitations dominate the
highest partial waves.  In such cases it is often advantageous to solve
the coupled channels sets (\ref{CRC}) for, say, every $n$'th value of
$J_T$, and interpolate the intermediate values.  Different values
of $n$ can be used in different reaction regions: $n$ can be small (1 or
2) for the grazing partial waves, and up to 5 or 10 for the
Coulomb-dominated peripheral processes, and can be adjusted for
the required balance between speed and accuracy.

This interpolation may be performed on the S-matrix elements themselves,
or on the Legendre amplitudes of equation (\ref{ALeg})
In this second method (that used in ref. \cite{FRESCO}),
cubic spline interpolations
are used. The main factor limiting the accuracy of this process is
that the rate of change with $J_T$ of the Coulomb phase shifts
$\exp i(\sigma_L + \sigma'_ {L'})$ does {\em not}
diminish as $J_T$ increases.  For that reason, it is advisable to
interpolate not on the $A^{L'} $ of equation (\ref{ALeg}), but on
a $\tilde A^{L'}$ defined with a {\em revised} phase shift factor
$\exp i(\sigma_L - \sigma'_ {L'}) .$  Since $L$ and $L'$ both tend to be near $J_T$, it is only the
{\em difference} the phase shifts which limits the accuracy of the
interpolation. It will therefore be more accurate for smaller projectile
and target spins, and incoming and outgoing channels with similar
Sommerfeld parameter $\eta$ (equation \ref{EQeta}).
%%%
\section{Coupling Interactions}
\label{potls}
%%.se potls = substr &@xref 1 1
%%%% potentials section = &potls

%%.rc 4 on
When two nuclei interact, a variety of kinds of elastic and inelastic
potentials may be needed to describe their interaction.
As well as the scalar nuclear attractions and scalar Coulomb repulsions,
if either of the nuclei has spin $ J \neq 0$, then there can be
higher-order tensor interactions which couple together the spin and the
orbital motion. If a nucleus has spin $ J \geq\half$, then there can be
a spin-orbit component $ V_{ls} (R) 2 \vecl \cdot \vecJ $ in the
Hamiltonian ${\cal H}$. and if its spin is one or greater
($J \geq 1$), there can be tensor forces of various kinds.
The most commonly used tensor force is a $ T_r $ potential
of the form
$V_{Tr} (R) {\bf R}_2 ( \vecR , \vecR ) {\bf .S}_2 (J,J) $.
Similar tensor forces are also generated if the projectile and target
spins coupled together can reach $J_p + J_t \geq 1 $:
such is the case with the tensor force between the proton and the neutron
within the deuteron.

Inelastic potentials (\ref{potls}.2) arise when one or both of the
nuclei have permanent deformations (as seen in their intrinsic frame),
or are vibrationally deformable. The inelastic potentials which come from
rotating a permanently deformed nucleus are described in the Hamiltonian
by terms of the form
\begin{eqnarray} \label{rotdef}
{\bf V}_\lambda =   \sum _ \mu \Vee_\lambda (R) D^\lambda_{\mu 0}
         Y^\mu_\lambda (\hat \vecR)
\end{eqnarray}
where the form factors $\Vee_\lambda (R)$ have both nuclear and
Coulomb components for angular momentum transfers $\lambda$.
Their nuclear component is approximately proportional
to the derivative of the scalar potential between the two reaction
partners. Simultaneous excitations of both nuclei are also possible
(see e.g. \cite{iman69}), but have not been included in the present code.
Vibrational excitations of a nucleus have more complicated form factors in
general \cite{TAM65}, but can still be expanded in the form of
equation (\ref{rotdef}).
For the more intricate level schemes of strongly-deformed nuclei, it will
in general be necessary for each allowed transition to have  its own
transition rate specified independently of a particular rotational or
vibrational model.

Inelastic potentials also arise when one of the
nuclei can be decomposed into a `core' + `valence particle'
structure \ref{potls}.3),
such that the opposing nucleus interacts with the two
components with distinct potentials acting on distinct centres-of-mass.
The valence particle can be a single nucleon, as in the case of
$^{17}$O = $^{16}$O + n, or it can be a cluster of nucleons,
as in $^6$Li =$\alpha$ + $^2$H, or
$^7$Li =$\alpha$ + $^3$H.  In all these cases, there arise
inelastic potentials which can re-orient the ground state of the
composite nucleus, or can excite the valence particle into higher-energy
eigenstates.

Finally, transfer interactions (\ref{potls}.4) arise when the
reaction brings about the transfer of a valence particle from one nucleus
into a bound state around the other. As the incoming and outgoing
projectiles have different centres-of-mass, with the targets likewise,
the correct treatment of transfer interactions requires taking into
account the effects of recoil and of the finite ranges of the binding
potentials. These result in the coupling form factors becoming non-local,
so that they must be specified by the two-dimensional kernel functions
$ V_{\alpha :\alpha'} (R_\kappa , R_{\kappa'} )$
in equation (\ref{CRC}). They also require that the coupled equations be solved
by iteration, as will be discussed in section \ref{solns}.
If the effects of recoil are neglected, the `no-recoil' (NR)
approximation is obtained, but in general\cite{LOLA}
this is inaccurate in ways which are difficult to predict. For that reason
the NR approximation is not included in the present code.
For many light-ion reactions, however, another `zero-range'
approximation is available, and this does remove many of the finite-range
requirements. Alternatively, a first-order correction for the finite-range
effects may be estimated, to give the `local energy approximation'.
These two special cases are discussed at the end of the section.
%%.rc 4 off
%%%
%%%
\subsection{Matrix Elements of Tensor Forces}

%\newcommand{\Vee}{{\sf V}}
\newcommand{\Jp}{J_p}
\newcommand{\Jt}{J_t}
\newcommand{\JT}{J_T}
\newcommand{\Jpq}{\ensuremath{J'_p}}
\newcommand{\Jtq}{\ensuremath{J'_t}}
\newcommand{\Lq}{L'}
%\newcommand{J'_1}{\ensuremath{J_1^'}}
%\newcommand{J'_1}{\ensuremath{J_1}}
%\newcommand{J'_2}{\ensuremath{J'_2}}
%\newcommand{J_1}{\ensuremath{J_1}}
%\newcommand{J_2}{\ensuremath{J_2}}
\newcommand{\Sq}{\ensuremath{S'}}
%%%

This section presents the matrix elements for spin-orbit forces and a
variety of tensor interactions.  The radial form factors
$\Vee_Q (R)$ which multiply these matrix elements are not
specified, since these are usually determined by a fitting procedure
in an optical-model search code, and a wide variety of
parameterised forms have been used.

We shall use the $ | (LJ_p) J_1 ,J_t ;J_T\rangle $ representation
for the order of coupling the spins, as in equation (\ref{spinrep}).
%%%
\subsubsection{Spin-orbit Interactions}

For the projectile spin-orbit force $ {\bf L\cdot J}_p $
\begin{eqnarray}\nonumber
&&\left\langle (L \Jp ) J_1 , \Jt ; \JT | {\bf L\cdot J}_p
                             | ( \Lq \Jp ) J'_1 , \Jt ; \JT \right\rangle
\\
&&= \delta_{L \Lq } \delta_{ J_1  J'_1 }
    \half [ J_1 ( J_1 +1) - L(L+1) - \Jp ( \Jp + 1) ]
\end{eqnarray}
This convention amounts to a $ 2 \vecl \cdot \vecs $ spin-orbit force, rather
than one based on $ \vecl \cdot {\bf \sigma}$. These are the same for nucleons
and spin $\half$ nuclei, but it means, for example, that the
spin-orbit strengths for deuterons and $^7$Li will
have to be decreased as they have $ \vecs$ = 1 and 3/2 respectively.

For the target spin-orbit interaction $ {\bf L\cdot J}_t$, we first transform
\begin{eqnarray} \nonumber
 | (LJ_p) J_1 ,J_t ;J_T\rangle &=& (-1)^{J_1 - L - J_p }
                                    | (J_p L) J_1 ,J_t ;J_T\rangle
\\
 &  =& (-1)^{J_1 - L - J_p } \sum _ {J_2}
 | J_p , (LJ_t ) J_2 ; J_T\rangle \hat J_1 \hat J_2
  W( J_p L J_T J_t ; J_1 J_2 )
\end{eqnarray}
so
\begin{eqnarray*} \nonumber
&&\left\langle (L J_p ) J_1 , J_t ; J_T | {\bf L\cdot J}_t
                             | ( L' J_p ) J'_1 , J_t ; J_T \right\rangle\\
 &&= (-1)^{J_1 - J'_1 + L' - L} \hat J_1 \hat J'_1
  \sum _ {J_2} (2 J_2 + 1)  W( J_p L J_T J_t ; J_1 J_2 )
                           W( J_p L J_T J_t ; J'_1 J_2 )
\\
 && \delta_{L L' }\half [ J_2 ( J_2 +1) - L(L+1) - J_t ( J_t + 1) ]
\end{eqnarray*}

\subsubsection{Second-rank Tensor Forces}

We use the notations of ref. \cite{satch60}:
\begin{eqnarray}
\langle S \| {\bf S}_2 \| S\rangle = {1 \over \sqrt 6} ~
      { 3K^2 - S(S+1) \over \langle SK20 | SK\rangle}
      \mbox{ for  any } | K | \leq S ,
\end{eqnarray}
and
\begin{eqnarray}
\langle L' \| {\bf R}_2 \| L\rangle = \sqrt {2 \over 3}
       {\hat L \over \hat{L'}}\langle L~0~2~0~ | L' ~0~\rangle
\end{eqnarray}
for the reduced matrix elements of the second-rank spin and radial tensors
respectively.
With the projectile $ {\bf T_r} $ tensor force
$ {\bf R}_2 \cdot {\bf S}_2 ( J_p J_p ) $,
the coupling interactions are
\begin{eqnarray} \nonumber
\left\langle (L J_p ) J_1 , J_t ; J_T |  {\bf R}_2 \cdot {\bf S}_2 ( J_p J_p )
                             | ( L' J_p ) J'_1 , J_t ; J_T \right\rangle
\end{eqnarray}
\begin{eqnarray}
 ~=~
     \delta_{ J_1 J'_1 } \hat L \hat J_p (-1)^{ J_1 - L - J_p }
    W(L L' J_p J_p ; 2 J_1 )\langle L \| {\bf R}_2 \| L'\rangle
                              \langle J_p \| {\bf S}_2 \| J_p\rangle
\end{eqnarray}

For the target $ {\bf T_r} $ tensor force
$ {\bf R}_2 \cdot {\bf S}_2 ( J_t J_t ) $
the coupling interactions are
\begin{eqnarray} \nonumber
\left\langle (L J_p ) J_1 , J_t ; J_T |  {\bf R}_2 \cdot {\bf S}_2 ( J_t J_t )
                             | ( L' J_p ) J'_1 , J_t ; J_T \right\rangle
\end{eqnarray}
\begin{eqnarray}  \nonumber
 ~=~
    (-1)^{J_1 - J_1' + L' - L}
 \sum _ {J_2} \hat J_1 \hat J'_1 (2 J_2 + 1)
  W( J_p L J_T J_t ; J_1 J_2 ) W( J_p L' J_T J_t ; J_1' J_2 )
\end{eqnarray}
\begin{eqnarray}
   \times  \hat L \hat J_t (-1)^{ J_2 - L - J_t }
    W(L L' J_t J_t ; 2 J_2 )\langle L \| {\bf R}_2 \| L'\rangle
                              \langle J_t \| {\bf S}_2 \| J_t\rangle
\end{eqnarray}

For the combined target-projectile $ {\bf T_r} $ tensor force
$ {\bf R}_2 \cdot {\bf S}_2 ( J_p J_t ) $
the coupling interactions are
\begin{eqnarray} \nonumber
\left\langle (L J_p ) J_1 , J_t ; J_T |  {\bf R}_2 \cdot {\bf S}_2 ( J_p J_t )
                             | ( L' J_p ) J'_1 , J_t ; J_T \right\rangle
\end{eqnarray}\begin{eqnarray}  \nonumber
 ~=~
 \sum_{S S' }   \hat J_1 \hat J'_1 \hat S \hat S'
  W( L J_p J_T J_t ; J_1 S ) W( L' J_p J_T J_t ; J'_1 S' )
\end{eqnarray}
\begin{eqnarray}
\times \hat L \hat S\ (-1)^{J_T - L - S' } W(L L' S S' ; 2 J_T )
  \langle L \| {\bf R}_2 \| L'\rangle
  \langle ( J_p J_t )S \| {\bf S}_2 \| ( J_p J_t ) S'\rangle
\end{eqnarray}
where the second-rank reduced matrix element is
\begin{eqnarray} \nonumber
 \langle ( J_p J_t )S \| {\bf S}_2 \| ( J_p J_t ) S'\rangle
=
 \hat S' \hat 2 \hat J_p \hat J_t  ~
 \left ( \begin{array}{ccc}S & J_p & J_t \\S' & J_p & J_t\\2 & 1& 1\end{array} \right )  ~
\sqrt { J_p ( J_p + 1) } \sqrt { J_t ( J_t + 1) }
\end{eqnarray}
%%%
\subsection{Inelastic Excitations}
\subsubsection{Nuclear Rotational Model}

Consider a deformed nucleus with deformation lengths $\delta_\lambda$.
The effect of these deformations can be expressed as a change in the radius
at which we evaluate the optical potentials, the change depending on the
relative orientations of the radius vector to the intrinsic orientation
of the nucleus.  Deformation lengths are used to specify the these
changes, rather than fractional deformations $\beta_\lambda$,
to remove a dependence on the `average potential radius'
$R_U$. This is desirable because often the real and imaginary
parts of the potential have different radii, and it is not clear which is
to be used. It also removes a dependence on exactly how the `average
radius' of a potential is to be defined.

When $U(R)$ is the potential shape to be deformed,
the coupling interaction is
\begin{eqnarray}
 {\bf V}( {\bf \xi} , \vecR) = U(R - \delta( \hat\vecR ,  {\bf \xi} ))
\end{eqnarray}
where the `shift function' has the multipole expansion
\begin{eqnarray}
 \delta( \hat{\vecR'}) = \sum_{\lambda \neq 0}  \delta_\lambda
         Y^0_\lambda (\hat{\vecR'})
\end{eqnarray}
($\hat \vecR' $ is the vector $ \hat\vecR$ in the
body-centred frame of coordinates defined by $ {\bf \xi} $).
Transforming to the space-fixed frame of reference,
and projecting onto the spherical harmonics,
the multipole expansion becomes
\begin{eqnarray}
 {\bf V}( {\bf \xi} , \vecR) =
   & \sum_{\lambda \mu} \Vee_\lambda (R) D^\lambda_{\mu 0}
         Y^\mu_\lambda (\hat\vecR )
\\
 \mbox{where }  \Vee_\lambda (R) =&\half \int _{-1}^{+1}
                U(~r(R,\cos \theta)~) Y_\lambda^\mu (\theta ,0) ~d(\cos \theta)
\\
 \mbox{and } r(R,u) = &R - \sqrt{ {2 \lambda+1 \over 4 \pi} } P_\lambda (u) \delta_\lambda
                        + \epsilon
\\
 \mbox{with }  \epsilon = &{\sum _ \lambda {\delta_\lambda}^2} / (4 \pi R_U )
\end{eqnarray}
The correction $\epsilon$ is designed (\cite{Sturm})
to ensure that the volume integral of the monopole potential
$\Vee_0 (R)$ is the same as that of $U(R)$, and is correct
to second order in the $\{\delta_\lambda\}\}$.

%%.rc 5 on
When the $\{\delta_\lambda\}\}$ are small, the above multipole functions
are simply the first derivatives of the $U(R)$ function:
\begin{eqnarray}
\Vee_\lambda (R) = - {\delta_\lambda \over \sqrt {4 \pi}} ~{dU(R) \over dR} ,
\end{eqnarray}
with the same shape for all multipoles $\lambda> 0$.
%%.rc 5 off
%%%
\subsubsection{Coulomb Deformations}

The deformations of the Coulomb potential can also be defined by the
$\delta_\lambda$, but unfortunately an average potential radius is
again introduced.  The dependence on models for average radii can be
reduced by defining the Coulomb deformations in terms of a reduced matrix
element such as that of Brink and Satchler \cite{bs68}, or that of Alder and Winther
\cite{alder66}.
For the present purposes we adopt that of Alder and Winther, as it is
hermitian upon interchanging the forward and reverse directions.
We include, however,
a simple phase factor to keep it real-valued.  The new deformation
parameter is called $M(E \lambda)$ and has units of $e . {\rm fm}^\lambda $.
In terms of the Alder and Winther reduced matrix element it is
\begin{eqnarray}
  M(E \lambda) = i^{ I - I' + | I - I' |} \times
            \langle I' \| E \lambda \| I\rangle
\end{eqnarray}
and is directly related to the observable electro-magnetic transition
rate without any model-dependent parameters entering (except a sign):
\begin{eqnarray}
   M(E \lambda) =\pm\sqrt { (2I+1) B(E \lambda , I \rightarrow I') }.
\end{eqnarray}

A model dependent radius parameter $R_c$
only enters in the relation to the
deformation lengths of the rotational model:
\begin{eqnarray} \label{MEk}
   M(E \lambda , I \rightarrow I') =
               {3Z \delta_\lambda {R_c}^{\lambda-1} \over 4 \pi} ~
               i^{ I - I' + | I - I' |} ~
               \sqrt {2I+1} ~~\langle I K \lambda 0 | I' K\rangle
\end{eqnarray}
for transitions from a state of spin $I$ to one of spin $I'$
in a rotational band of projection $K$ in a nucleus of charge $Z$.
Within $K$=0 bands, $M(E \lambda , 0 \rightarrow I) = M(E \lambda , I \rightarrow 0) $
have the same sign as $\delta_\lambda$.

The only disadvantage of using reduced matrix elements as input parameters
in this way is
that the transitions in a rotational band do not all have the same
matrix elements $M(E \lambda , I \rightarrow I')$,
even when the deformation length is constant.

The radial form factors for Coulomb inelastic processes may be simply
derived from the multipole expansion of
$| \vecr - \vecr' |^{-1}$, giving
%%.rc 3 on
\begin{eqnarray}
\Vee_\lambda^c (R) = M(E \lambda ) ~
             { \sqrt {4 \pi} e^2 \over 2 \lambda+1} ~~
          \left \{  \begin{array}{ll}
                {R^\lambda  /   {R_c}^{2 \lambda+1} ~~(R \leq R_c)}\\
                 {1  /   R^{\lambda+1} ~~ (R> R_c) }
                 \end{array} \right. ~
\end{eqnarray}
%%.rc 3 off
remembering that a factor $ \delta_\lambda {R_c}^{\lambda - 1} =
\beta_\lambda {R_c}^\lambda $ is already included in the matrix element
%%.rc 3 on
of equation (\ref{MEk}) which appears in this form factor.
This form factor is to be multiplied by the angular momentum coupling
coefficients of the next section, and also by the charge of the
opposing nucleus.
%%.rc 3 off
%%%
\subsubsection{Angular Momentum Coupling Coefficients}

\newcommand{\BX}{{\bf X}}
The basic rotational coupling coefficient, with $ {\bf V}_\lambda $
given by equation (\ref{rotdef}), is
\begin{eqnarray}
 \BX^{J \lambda}_{LI:L' I'} (R) =
     \langle LI; J | {\bf V}_\lambda | L' I' ; J\rangle
\end{eqnarray}
%%.rc 5 on
The Coulomb form factors $\Vee_\lambda^c (R)$ have
coupling coefficients
\begin{eqnarray}
  \BX^{J \lambda}_{LI:L' I'} (R) =
     \hat L \hat {L'} (-1)^{J-I' -L+L'}
     W(L L' I I' ; \lambda J)
    \langle L 0 L' 0 | \lambda 0\rangle
       ~~ \Vee_\lambda^c (R)
\end{eqnarray}
whereas the nuclear form factors $ \Vee_\lambda (R) $ defined for
a rotational band with projection $K$ have coupling coefficients
\begin{eqnarray}\nonumber
  \BX^{J \lambda}_{LI:L' I'} (R) &=&
 \hat L \hat {L'} (-1)^{J-I' -L+L'}
     W(L L'  I  I' ;  \lambda J)
    \langle L  0  L'  0 | \lambda 0\rangle
       ~~ \Vee_\lambda (R) \\
 &&\hat{I' } \langle I'  K  \lambda  0  |  I  K\rangle .
\end{eqnarray}
%%.rc 5 off

For projectile inelastic excitation, this coupling coefficient may be used
directly as
\begin{eqnarray}
\langle (L J_p)J, J_t ; J_T | {\bf V}_\lambda |
    (L' J'_ p)J' , J'_ t ; J_T\rangle
=
 \delta_{J_t , J'_ t} ~\delta_{J , J'} ~
 \BX^{J \lambda}_{LJ_p :L' J'_ p} (R)
\end{eqnarray}
whereas for target excitations,
\begin{eqnarray}
\nonumber
&&\langle (L J_p)J, J_t ; J_T | {\bf V}_\lambda |
    (L' J'_ p)J' , J'_ t ; J_T\rangle
=
  \delta_{J_p , J'_ p} ~
  (-1)^{J-J' -L+L'} \hat J \hat {J'}
\\
&& \times \sum_{J_2}    (2J_2 + 1)
      W(J_p L     J_T J    _t ; J     J_2)
      W(J_p L' J_T J'_ t ; J' J_2)
      \BX^{J_2 \lambda}_{LJ_t :L' J'_ t} (R)
\end{eqnarray}
%%%
\subsection{Single Particle Excitations}

When a nucleus consists of a single particle outside a core,
the state of the particle can be disturbed by the interaction with1
another nucleus, as the force of that nucleus can act differentially
on the particle and the core.
If $V_{cc} (\vecR_c)$ and $ V_p (\vecr') $
are the interactions of the second nucleus with the
core and particle respectively,
then the excitation coupling from state
$| (\ell' L') \lambda\rangle$ to state
$| (\ell L) \Lambda\rangle$ is given by the single-folding expression
\begin{eqnarray}
\BX^\Lambda_{\ell L: \ell' L'} (R) =
\langle (\ell L) \Lambda
     | V_{cc} (\vecR_c) + V_p (\vecr') - U_{\rm opt} (\vecR) |
    (\ell' L') \Lambda\rangle
\end{eqnarray}
where $U_{\rm opt} (\vecR) $ is the optical potential already defined for
these channels.
This optical potential is subtracted to avoid double counting of either
the Coulomb or the nuclear potentials,
rather than disabling the potentials which have already been defined.
This means that the `monopole' potential $ \Vee_0 (R,r) $
(to be constructed) will have no long-range Coulomb component,
and will not disturb the matching of the wave functions to the
asymptotic Coulomb functions.
It also means that if a nuclear well has already been defined,
the new monopole form factor will be simply the difference between this well
and that desired well calculated from the folding procedure.

If the potentials $V_{cc} (\vecR_c)$ and $U_{\rm opt} (\vecR)$
contain only scalar components, then
the $R$- and $r$- dependent Legendre multipole potentials
can be formed as
\begin{eqnarray}
  \Vee_K (R,r) =\half \int_{-1}^{+1}
    \left [ V_{cc} (\vecR_c) + V_p (\vecr') - U_{\rm opt} (\vecR) \right ]
      . P_K (u) du
\end{eqnarray}
where
\begin{eqnarray} \nonumber
  K &=& \mbox{ the multipole moment,}
\nonumber
\\
  u &=& \hat\vecr \cdot \hat\vecR
            \mbox{ is the cosine of the angle between } \vecr \mbox{ and } \vecR ,
\nonumber
\\
  \vecr &=& a \vecR + b \vecr \mbox{ is the particle-core vector,}
\nonumber
\\
  \mbox{and } \vecR_c &=& p \vecR + q \vecr \mbox{ is the core-nucleus vector.}
\nonumber
\end{eqnarray}
The coupling form factor between states $u_{\ell'} (r) $
and $u_\ell (r) $ is then
\begin{eqnarray} \nonumber
 \BX^\Lambda_{\ell L: \ell' L'} (R)
     &=&\half \sum _ K
     { \int_ 0^ {R_m} u_\ell (r)^*
                \Vee_K (R,r) u_{\ell'} (r) dr}
                (-1)^{\Lambda+K} \hat\ell \hat L \hat {\ell'} \hat {L'}\\
 &\times&
       (2K+1) W(\ell \ell' L L' ;K \Lambda)
        \left ( \begin{array}{ccc}K &\ell& \ell'\\0&0&0 \end{array} \right )
        \left ( \begin{array}{ccc}K&L&L'\\0&0&0 \end{array} \right )
\end{eqnarray}
%%%
\subsubsection{Projectile Single-Particle Mechanisms}

If the {\em projectile} has the particle - core composition, then the
coupling interaction is
\begin{eqnarray}
 V^{J_T}_{\alpha :\alpha'} (R) =
\langle (L       J      _p)J, J_t ;J_T | {\bf V} |
   (L' J'_ p)J, J_t ;J_T\rangle
\end{eqnarray}
where the initial (primed) and final (unprimed) states are
\begin{eqnarray}
   \phi_{J'_ p} (\xi_p , r) =
    \sum_{\ell' sj'}
      A_{\ell ' sj'}^{j' I_p J'_ p}
      | ( \ell' s) j' , I_p ; J'_ p\rangle
       \mbox{  and }
   \phi_{J_p} (\xi_p , r) =
    \sum_{\ell sj}
      A_{\ell  sj}^{jI_p J_p}
    | ( \ell s) j , I_p ; J_p\rangle ,
\end{eqnarray}
respectively, and $I_p$ is the (fixed) spin of the core.
Then
\begin{eqnarray} \nonumber
 V^{J_T}_{\alpha :\alpha'} (R)
     &=&
 \sum_{\begin{array}{c}F \Lambda I_p\\j j' \ell \ell'\end{array}}
     {\hat j \hat {j'} (2F+1) \hat {J_p} \hat {J'_ p} (2 \Lambda+1)
     W(\ell s J_p I_p ; j F)
     W(\ell' s J'_ p I_p ; j' F)  }
\\
 && \times
  A_{\ell ' sj'}^{j' I_p J'_ p} ~
  A_{\ell  sj}^{jI_p J_p} ~
  W(L \ell J F; \Lambda J_p)
  W(L' \ell' J F; \Lambda J'_ p)
  \times
  \BX^\Lambda_{\ell L: \ell' L'} (R)
\end{eqnarray}
%%%
\subsubsection{Target Single-Particle Mechanisms}

If the {\em target} has the particle - core composition, then the
coupling interaction is
\begin{eqnarray}
 V^{J_T}_{\alpha :\alpha'} (R) =
\langle(L       J_p)J, J'_ t ;J_T | {\bf V} |
   (L' J_p)J, J    _t ;J_T\rangle
\end{eqnarray}
where the initial (primed) and final (unprimed) states are
\begin{eqnarray}
   \phi_{J'_ t} (\xi_t , r) =
    \sum_{\ell' sj'}
      A_{\ell ' sj'}^{j' I_t J'_ t}
      | ( \ell' s) j' , I_t ; J'_ t\rangle
    \mbox{  and }
   \phi_{J_t} (\xi_t , r) =
    \sum_{\ell sj}
      A_{\ell  sj}^{jI_t J_t}
    | ( \ell s) j , I_t ; J_t\rangle ,
\end{eqnarray}
respectively, and $I_t$ is the (fixed) spin of the core in the target.
Then
\begin{eqnarray}
 V^{J_T}_{\alpha :\alpha'} (R)
   &=&
 \sum_{I_t j j' \ell \ell'}
  A_{\ell ' sj'}^{j' I_t J'_ t} ~
  A_{\ell  sj}^{jI_t J_t}
\nonumber
\\
&& \times \sum_{J_a}
    (2J_a + 1) \hat{J'_ t} \hat {J_t}        ~
     W(J j J_T I_t ; J_a J_t) ~
     W(J' j' J_T I_t ; J_a J'_ t)
\nonumber
\\
&& \times {\sum_{\Lambda s_a } }
    \left \{ \begin{array}{ccc} L&\ell&\Lambda\\J_p&s&s_a\\J&j&J_a\end{array}\right \}
    \left \{ \begin{array}{ccc} L'&\ell'&\Lambda\\J_p&s&s_a\\J'&j'&J_a\end{array}\right \}
  \BX^\Lambda_{\ell L: \ell' L'} (R)
\end{eqnarray}
%%%
%.se scl = '\ell'
%%%
\subsection{Particle Transfers}
\subsubsection{Finite Range Transfers}
\label{EFR}
%.se EFR = &@xref

To calculate the coupling term that arises when a particle is transferred,
for example from a target bound state to being bound in the projectile,
we need to evaluate source terms of the form
\begin{eqnarray}
    S_\alpha (R) = \int_ 0 ^ \infty
\langle (LJ_p)J,J_t ;J_T | {\bf V} |
   (L' J'_ p)J' ,J'_ t ;J_T\rangle ~
   f^{J_T}
    _{ (L' J'_ p)J' ,J'_ t} (R' ) dR'
\end{eqnarray}
where the initial (primed) state has a composite target
with internal coordinates
$ \xi'_ t \equiv \{ \xi_t , \vecr' \}: $
$
 \phi_{J'_ t} (\xi_t , \vecr' ) =
    | ( \ell' s) j' , J_t ; J'_ t\rangle
$
and the final (unprimed) state has a composite projectile
with internal coordinates
$ \xi_p \equiv \{ \xi_{p'} , \vecr \}: $
$
   \phi_{J_p} (\xi'_ p , \vecr) =
    | ( \ell s) j , J'_ p ; J_p\rangle .
$

The ${\bf V}$ is the interaction potential, of which the prior form is
\begin{eqnarray}
  {\bf V} = V_{\ell sj} (\vecr) + U_{cc} (R_c) - U_{\alpha'} (\vecR')
\end{eqnarray}
and the post form is
\begin{eqnarray}
 {\bf V} = V_{\ell' sj'} (\vecr') + U_{cc} (R_c) - U _\alpha (\vecR)
\end{eqnarray}
where $ V_\beta (\vecr)$ is the potential which binds
$ \varphi_\beta (\vecr),$ $U _\alpha (\vecR)$
are the optical potentials, and $U_{cc} (\vecR_c)$
is the `core-core' potential, here between the $p'$
and the $t$ nuclei.
The $ V_\beta$ will be real, but the $U _\alpha$ and
$U_{cc}$ will typically have both real and imaginary components.

This source function $S_\alpha (R)$ evaluates a non-local
integral operator, as it operates on the function
$f_{\alpha'} (R' ) $ to produce a function of $R$.
This section therefore derives the non-local kernel
$V_{\alpha ,\alpha'} (R,R' ) $
so that the source term, which initially involves a five
dimensional integral over $ \vecr $ and $ \hat \vecR$,
may be calculated by means
of a one-dimensional integral over $R'$:
\begin{eqnarray} \label{sourceint}
S_\alpha (R) = \int_ 0 ^ {R_m}
                  V_{\alpha ,\alpha'} (R,R' )
                  f_{\alpha'} (R' ) dR' .
\end{eqnarray}
Note that when the initial and final single-particle states are real, then
the kernel function is symmetric
\begin{eqnarray}
  V_{\alpha ,\alpha'} (R,R' ) =
  V_{\alpha' ,\alpha} (R' ,R),
\end{eqnarray}
whereas if the states are unbound and complex-valued, then the kernel
function is hermitian provided the interaction potential
${\bf V}$ is real. If the particle states and the interaction potential
are complex, then both the forward and reverse kernels must be
each calculated independently.

When the potential {\bf V} contains only scalar potentials, the
kernel calculation can be reduced to the problem of finding
$\BX^\Lambda_{\ell L: \ell' L'} (R,R' ) $
such that, given
\begin{eqnarray}
&&\langle (LJ_p)J,J_t ;J_T | {\bf V}|(L' J'_ p)J' ,J'_ t ;J_T\rangle=\sum_{\Lambda F}
(-1)^{s+J'_ p - F} \hat J \hat {J'_ t} \hat j \hat F \hat {J_p} \hat \Lambda
\left \{ \begin{array}{ccc}L'&J'_p&J'\\ \ell'&s'&j'\\ \Lambda&F&J\end{array} \right \} \nonumber \\
 && \times W(J_t j' J_T J' ;~ J'_ t J)
             W(ls J_p J'_ p ;~jF)
             W(L \ell  J F;~ \Lambda J_p )
            \langle \ell L; \Lambda | {\bf V} | \ell' L' ; \Lambda\rangle  ,
        \label{Vtransf}
\end{eqnarray}
the integral operator
$
\langle \ell L; \Lambda | {\bf V} | \ell' L' ; \Lambda\rangle
\mbox{ has the kernel function }
 \BX^\Lambda_{\ell L: \ell' L'} (R,R' ).
$
Note that the $F$ summation may be performed in an inner loop that does
not evaluate the kernel function.

Now the $\vecr$ and $\vecr'$ are linear combinations of the channel vectors
$\vecR$ and $\vecR'$:
$ \vecr = a \vecR + b \vecR' \mbox{ and } \vecr' = a' \vecR + b' \vecR' $
where,
when $ \varphi_\ell (\vecr) $ is the projectile bound state,
\begin{eqnarray}
a = \nu_t \omega , ~~~ b = - \omega , ~~~ a' = \omega , ~~~
b' = - \nu_p \omega ,
\end{eqnarray}
with
$ \nu_p \equiv A_{\kappa' p} / A_{\kappa p}$ ,
 $\nu_t \equiv A_{\kappa t} / A_{\kappa' t}$ ,
  and $ \omega = (1 - \nu_p \nu_t ) ^ {-1}$ .
%%.rc 5 on
When $ \varphi_\ell (\vecr) $ is the target bound state
\begin{eqnarray}
a = - \nu_p \omega, ~~~ b = \omega , ~~~ a' = - \omega ,
b' = \nu_t \omega
, ~~~
\end{eqnarray}
%%.rc 5 off
with
$\nu_p \equiv A_{\kappa p} / A_{\kappa' p}$ ,
$\nu_t \equiv A_{\kappa' t} / A_{\kappa t}$ ,
  and $ \omega = (1 - \nu_p \nu_t ) ^ {-1}$ .
%%.rc 5 on
The `core-core' vector is always $ \vecR_c = \vecr' - \vecr
 = (a' - a) \vecR + (b' - b) \vecR' . $
%%.rc 5 off

Thus the spherical harmonics
$ Y_\ell ( \hat \vecr ) $ and $ Y_{\ell'} ( \hat\vecr' ) $
can be given in terms of the spherical harmonics
$ Y_n ( \hat\vecR ) $ and $ Y_{n'} ( \hat\vecR' ) $
by means of the Moshinsky \cite{MOSH}
solid-harmonic expansion (see also refs. \cite{aust64}
and~\cite{OHMURA}
\begin{eqnarray}
Y_\ell^m ( \hat \vecr) =
   \sqrt {4 \pi} \sum_{n \lambda} c( \ell ,n)
   {(a R)^{\ell  - n} (b R' )^n  \over r^ \ell }
   Y_{\ell  - n}^{m - \lambda} ( \hat\vecR)
   Y_n^\lambda ( \hat\vecR')
  \langle \ell  - n m - \lambda n \lambda | \ell m\rangle
\end{eqnarray}
where
\begin{eqnarray} \nonumber
c( \ell ,n) = \sqrt { {1 \over 2n+1} ~~
                        \left ( \begin{array}{c}2 \ell + 1\\ 2n\end{array} \right ) } ,
\end{eqnarray}
with $\left ( \begin{array}{c}x\\y\end{array} \right )$  the binomial coefficient (Appendix A).


%%.rc 5 on
We now perform the Legendre expansion
\begin{eqnarray}
{\bf V} {u_{\ell sj} (r) \over r^{\ell +1}} ~
      {u_{\ell' sj'} (r') \over {r'}^{\ell' +1} }~
 =
  \sum _ T  (2T+1) {\bf q}^T_ {\ell , \ell'} (R,R')
                P_T (u)
\end{eqnarray}
where the Legendre polynomials $ P_T (u)$
%%.rc 5 off
are functions of $u$, the cosine of the angle between
$\vecR $ and $\vecR' ,$ by using
$r = (a^2 R^2 +b^2 {R'}^2 + 2abRR' u)^{1/2} $
(with $r'$ analogously) in the numerical quadrature of the integral
\begin{eqnarray} \label{genq}
 {\bf q}^T_ {\ell , \ell'} (R,R')
 =
   \half \int_ {-1} ^ {+1}
         {\bf V} {u_{\ell sj} (r)\over r^{\ell + 1}} ~
               {u_{\ell' sj'} (r')\over {r'}^{\ell' +1}} ~
              P_T (u) du
\end{eqnarray}
The quadrature methods used here, and the accuracy attained, are
discussed in section 5.3.

%%.rc 5 on
Using the Legendre expansion, the radial kernel function
\begin{eqnarray} \nonumber
&&\BX^\Lambda_{\ell L: \ell' L'} (R,R' )
   =  {| b |^3 \over 2} \sum_{nn'} c(\ell n) c(\ell' n')
        R R' (aR)^{\ell - n} (bR')^n
        (a' R)^{\ell' - n'} (b' R')^{n'}
\nonumber
\\\nonumber
 && \times \sum_T  {\bf q}^T_ {\ell , \ell'} (R,R')
        (2T+1) (-1)^{\Lambda+T+L+L'} ~
          \hat \ell \hat \ell' ~\hat{(\ell - n)}~ \hat{(\ell' - n')}
              \hat n \hat {n'} \hat L \hat {L'}
\nonumber
\\\nonumber
 && \times \sum_{K K'} (2K+1)(2K' + 1)
        \left ( \begin{array}{ccc}\ell - n&n'&K\\0&0&0 \end{array} \right )
        \left ( \begin{array}{ccc}\ell'-n'&n&K'\\0&0&0 \end{array} \right )
        \left ( \begin{array}{ccc}K&L &T\\0&0&0 \end{array} \right )
        \left ( \begin{array}{ccc}K'&L'&T\\0&0&0 \end{array} \right ) \nonumber
\\
  && \times \sum_{Q} (2Q+1)
        W(\ell L \ell' L' ; \Lambda Q)
        W(K L K' L' ; T Q)
        \left ( \begin{array}{ccc} \ell'&Q&\ell\\ n'&K&\ell-n\\\ell' - n'&K'&n \end{array} \right )
        \label{useq}
\end{eqnarray}
%%.rc 5 off
These formulae can also be used with $ {\bf V} \equiv 1$ to calculate the kernel
functions $ K^\Lambda_{\ell L: \ell' L'} (R,R' ) $
for the wave function overlap operator
$K_{ij} \equiv\langle\phi_i |\phi_j\rangle $
needed in evaluating the non-orthogonality terms of section \ref{nono}.

One disadvantage of this method of calculating the two-dimensional
radial kernels
$\BX^\Lambda_{\ell L: \ell' L'} (R,R' ) $
is that in the process of transforming the solid harmonics of
$ \vecr $ and $\vecr' $ into those of
$ \vecR $ and $\vecR' $, there appears summations containing
high powers of the coefficients $ a, b, a' $ and $b'$
These products will become larger than unity by several orders of
magnitude, will the summed result is typically of the order of
unity.  This means that
the summations involve large cancellations,
and as the degree of cancellation gets worse for large
$ \ell $ and $ \ell' ,$ the cancellation places a limit
on the maximum value $ \ell + \ell' $ of the
transferred angular momentum.

One way of circumventing this loss of accuracy is that proposed by
Tamura and Udagawa \cite{TUM}, whereby solid harmonics are avoided in favour of a suitable
choice of axes to render it practical to calculate $m$-dependent
form factors directly.  If the ${\bf z} $ axis is not (as
usual) parallel to the incident momentum, but set parallel to
$\vecR$, and the $\vecx' $ axis set in the plane determined
by $\vecR $ and $ \vecR' ,$ then the $\vecr $ and $ \vecr'$
vectors are also in this plane.
The radial kernels may then be calculated as a sum of $m$-dependent
integrals over $ \cos \theta = \hat \vecR \cdot \hat{\vecR'} $,
as before the cosine of the angle between $\vecR $ and $ \vecR'$.
Although there are hence a larger number of radial integrals to be
performed, there are no large cancellations between the separate terms,
and there is no limit on the size of the transferred angular momentum.

A third method \cite{robson72}
of calculating the transfer form factors is that involving
expanding the initial and final channel wave functions in terms of
spherical Bessel functions:
\begin{eqnarray} \label{expan}
f _\alpha (R) = \sum_{n=1}^ {N(L)} {a _\alpha (K_n) R ~
                j_L (K_n R) } .
\end{eqnarray}
Using then the Fourier transform of the bound state wave functions
$ u_\ell (r) $ and $ u_{\ell'} (r')$,
a transfer T-matrix element may be written as a sum of a set of
one-dimensional integrals over a momentum variable.
Efficient codes \cite{nagel76}
have been written for CCBA calculations of transfers induced by light
ions (up to masses $\sim$ 10 to 15 amu).

This plane-wave expansion method has however several disadvantages  when
it comes to solving problems with coupled reaction channels.   If
transfers are to be calculated at each iteration of the coupled
equations, then the expansion (\ref{expan}) has to be recalculated at each step.
Another difficulty is that the method is not suited to heavy-ion induced
transfers, as the large degree of absorption inside the nuclei in these
cases requires a large number of momentum basis states  $ K_n $
to be represented accurately. The plane-wave expansion  becomes
uneconomical, and sometimes the determination of the
$ a _\alpha (K_n)$ coefficients becomes numerically ill-conditioned

We will see in section 5.3.1, however, that
if the cancellation which occurs in the first method is monitored,
and steps taken to keep it to a minimum, a workable code
\cite{FRESCO}
results which can produce accurate results for L-transfers up to around 6.
%%%
\subsubsection{Zero Range Transfers}

When the projectile wave functions
$ \varphi_\ell ( \vecr ) $
are all $s$-states ($\ell=0$ and
the interaction potential is of zero-range
$ ({\bf V} \varphi (\vecr) \sim D_0 \delta ( \vecr )~) , $
then the form factor
$ \BX^\Lambda_{\ell L: \ell' L'} (R,R' ) $
of equation (\ref{useq}) can be simplified to
\begin{eqnarray}
\BX^L_{0L: \ell' L'} (R,R') = D_0  ~
          {(-1)^{L' - \ell'} \over \hat L} ~
          {\hat{\ell'} \hat L \hat {L'}\over \sqrt{4 \pi}} ~
          \left ( \begin {array}{ccc}\ell'&L &L'\\0&0&0\end{array} \right ) ~
          {1 \over R} u_{\ell' sj'} (R ) ~
          {b^2 \over a} \delta (aR+bR') .
\end{eqnarray}
This can be made local by defining a new step size
$h' = -ah/b \equiv \nu_t h$
in the stripping channel $\alpha'$.
%%%
\subsubsection{Local Energy Approximation}

If the interaction potential is of small range, though not zero,
and the projectile still contains only $s$-states,
then a first-order correction may be made to the above form factor.
This correction will depend on the rate of oscillation of the source wave
function
$   f^{J_T} _{ (L' J'_ p),J' ,J'_ t} (R' ) $
within a `finite-range effective radius' $\rho$.
The rate of oscillation is estimated from the local energy
in the entrance and exit channels,
%%% DF.
%%%   K_{\alpha'} (R') = \sqrt \left [ {2 mu'} over \hbar^2 ~
%%%              \left ( E_{\alpha'} - U_{\alpha'} \right ) \right ] .
%%% EDF.
and the result \cite{buttle64}
is to replace $u_{\ell' sj'} (R) $
in the previous section by
\begin{eqnarray} \label{lea}
u_{\ell' sj'} (R) \rightarrow
u_{\ell' sj'} (R)
    \left [ 1 + \rho^2 {2 \mu _\alpha^{(p)}\over \hbar^2} ~
    \left ( U_{\alpha'} ( R) + V_{\ell' sj'} (R)
          -U _\alpha (R) + \epsilon _\alpha \right ) \right ]
\end{eqnarray}
where the $U(R)$ are the optical potentials, with
$V_{\ell' sj'} (r)$ the single-particle binding potential in the target.
The $ \mu _\alpha^{(p)}$ is the reduced mass of the particle in the projectile,
and $ \epsilon _\alpha $ its binding energy.

At sub-Coulomb incident energies \cite{gold68}, the details of the nuclear potentials
in equation (\ref{lea}) become invisible, and as the longer-ranged Coulomb
potentials cancel by charge conservation, the form factor can be simplified to
\begin{eqnarray}
 D_0 u_{\ell' sj'} (R) \rightarrow
     u_{\ell' sj'} (R) D_0
       \left [ 1 + \rho^2 {2 \mu _\alpha^{(p)}\over \hbar^2} \epsilon _\alpha \right ]
   &=& u_{\ell' sj'} (R) D
\end{eqnarray}
where
\begin{eqnarray} \label{DD0}
  D = D_0 \left [ 1 + \left ( \rho k _\alpha^{(p)} \right ) ^2 \right ]
\end{eqnarray}
is the effective zero-range coupling constant for sub-Coulomb transfers.

The parameters $~ D_0 $ and $ D $ can be derived
from the details of the projectile bound state $ \varphi_{0ss} ( \vecr)$.
The zero-range constant $ D_0$ may be defined as
\begin{eqnarray}
    D_0 = \sqrt {4 \pi} \int_ 0^\infty r V(r) u_{0ss} (r) dr.
\end{eqnarray}
The parameter $D$, on the other hand, reflects the asymptotic
strength of the wave function $u_{0ss} (r)$ as $r \rightarrow \infty$,
as it is the magnitude of this tail which is important in sub-Coulomb reactions:
\begin{eqnarray}
    u_{0ss} (r) = _ {r \rightarrow \infty}
         {2 \mu _\alpha^{(p)}\over \hbar^2} {1 \over \sqrt {4 \pi}} ~
         D e^{-k _\alpha^{(p)} r} .
\end{eqnarray}
It may be also found, using Schr\"odinger's equation, from the
integral
\begin{eqnarray}
    D  = \sqrt {4 \pi} \int_ 0 ^\infty
                {\sinh (k^{(p)} _\alpha r)\over k^{(p)} _\alpha} ~
                  V(r) u_{0ss} (r) dr.
\end{eqnarray}
From this equation we can see that as the range of the potential becomes smaller,
$D$ approaches $D_0$. The `finite-range effective radius'
$\rho$ of equation (\ref{DD0}) is thus some measure of the mean radius
of the potential $V(r).$
%%%
\section{Numerical Solutions}
\label{solns}
%%.rc 4 on

This section discusses the methods used to solve the coupled reaction
channels equations (\ref{CRC}), when in general there are both  local
couplings $ V_{\alpha :\alpha'}^\Gamma (R_\kappa)$
{\em and} non-local kernels
$ V_{\alpha :\alpha'} (R_\kappa , R_{\kappa'} )$.
Now a group of $m$ equations can be solved `exactly' (subject only
to radial discretisation errors) by finding \cite{BSH}
a set of $m$ linearly independent groups of solutions
$ g_{i,k} (R) $, and taking a linear combination of these which satisfies
the required boundary conditions.
This method is only practicable, however, if there are not too many
equations (the numerical effort can rise as $m^3$), and if there
are only local couplings. For in that case the independent solutions can
be found in a single outward `sweep' of $m^2$ radial
functions. Non-local couplings mean, unfortunately, that the source terms
at a given radius depend on the wave functions at other radii both larger
and smaller, so that this `exact' method becomes impractical.

In many cases of interest in nuclear physics, however, the non-local
couplings are not too strong, and can be treated as successive
perturbations. They can then be applied iteratively until further
applications have progressively smaller efffects, and the solutions have
converged (to some preset criterion of accuracy).
Some failures of convergence can remedied by the use of Pad\'e
approximants.

When both local and non-local couplings are present, and the local
couplings are too strong to allow an iterative scheme to converge, a
combination of the exact and iterative schemes is possible.
In this approach, several channels can be `blocked' together, and
treated as one unit during the iterations, while solving the couplings
{\em within} the block by the exact method.

There are several other features of typical nuclear reaction calculations
which simplify the numerical methods:
\begin{enumerate}
\item If the sum of the incoming projectile and target spins is greater than
one, then solutions will often be required for the {\em same} set of
CRC equations, only with different boundary conditions.
\item The diagonal potentials $U_\kappa (R_\kappa )$ usually have
a significant imaginary component for small $ R_\kappa $, and
hence damp the solutions  $f_\alpha (R_\kappa )$
in this region. This enables lower radial cutoffs to be used for
$ R_\kappa $ near zero, with little loss of accuracy.
\item
The bound states $u_{\ell sjI} (r) $ used in transfer
reactions decay exponentially outside the surface region of the nuclei.
This means that the integrand in equation (\ref{genq}) for the transfer kernels
will often decay exponentially both as
$| R  - R' | $ increases, and as
$ u \equiv cos \theta \equiv \hat {\vecR } \cdot \hat {\vecR'}$
decreases from unity.
\end{enumerate}
%%.rc 4 off
%%%% Ch. 5 intro ended on page &$PN., at &$LC lines from end.
%%%
\subsection{Integration of the Radial Equations}
\label{ExCC}
%.se ExCC = &@xref

If the non-local interactions $ V_{\alpha ,\alpha'} (R,R')$
in the CRC equations (\ref{CRC}) are present, then it will always be necessary
to solve the coupled channels by iteration. With the local couplings
$ V_{\alpha ,\alpha'}^\Gamma (R), $ however, we have a choice
whether to iterate, or to include them in the exact solutions of
the close-coupling method.  A simple option is to allow a specifiable
number $b$ of channels to be coupled
exactly, with the remainder only being fed after one or more iterations.
This would be useful, for example, if the channels for the low-lying
states of a highly-deformed target were included in this block of $b$
channels, and if the remaining channels (e.g. for transfers) were not fed
by more than 2 or 3 steps beyond this initial block.
Restricting these iterations to one is equivalent to solving a CCBA model.

Whether the coupled equations are of the simpler form of equation (\ref{CRC}),
or of the more complex form of section \ref{nono}, a particular $n$'th
iteration will require solving set of $m$ equations of the form
\begin{eqnarray} \label{diffeq}
  {d^2 \over dR^2} f_i (R) &=& \sum _ {j=1} ^ b
      C_{ij} (R) f_j (R) + S_i (R)
    \mbox{  for } i=1 \cdots b ,
\\
 \mbox{  and } {d^2 \over dR^2} f_i (R) &=&
      C_{ii} (R) f_i (R) + S_i (R)
    \mbox{  for } i=b+1 \cdots m ,
\end{eqnarray}
where $ S_i (R) $ is the source term constructed by means of the
wave functions $ f_i^{(n-1)} (R) $ of previous iterations :
\begin{eqnarray}
    S_i (R) = \sum_{j = j_{\rm min}}^ m
                  C_{ij} (R) f_j^{(n-1)} (R)
\end{eqnarray}
where $ j_{\rm min} = b+1 \mbox{ if } i \leq b $ and
$ j_{\rm min} = 1 $ if $ i> b. $

These coupled differential equations can be solved, following the method
of ref. \cite{BSH} by forming the linearly independent solution
sets $ g_{i,k} (R) $, where the $k$'th solution consists of
a set of all channels ($i=1 \cdots m$) which is made independent of
the other sets by having a distinctive starting value
\begin{eqnarray} \label{rstart}
    g_{i,k} (R_{\rm min} - h) = 0, ~~~~~
    g_{i,k} (R_{\rm min}) = {1 \over (2L_i + 1)!!} ~
       (K_i R_{\rm min})^{L_i + 1} \delta_{i,k}
\end{eqnarray}
for the initial conditions in the radial integration of equations (\ref{diffeq}).
For this integration, the code FRESCO uses the modified Numerov method,
and other codes such as Tamura's JUPITOR \cite{tam67}
have used Euler's method to start with near the origin ($R$=0),
and then St\"ormer's 6-point method to continue.
%%.rc 3 on
A general discussion of numerical integration schemes
is given in Melkanoff et al. \cite{melk66}, along with error analyses of the different methods.
%%.rc 3 off
%%%


The independent solutions $g_{i,k} (R) $ are required for $m$+1
values of $k$. The solution vectors for $k=1 \cdots m$ are solved
starting with equation (\ref{rstart}) but with {\em no} source term in
the equation (\ref{diffeq}): these will contribute to the complementary
solution of the homogeneous equation.
We also need a particular solution $g_{i,0} (R) $ of the
inhomogeneous equation, solved {\em with} the source terms but with
{\em no} non-zero values in equation (\ref{rstart}). These partial
solutions may be conveniently laid out as in figure \ref{CCsoln}.
If, however, it is known that the wave functions of certain channels are
not required (if, for example, they are only fed in the last iteration),
then it is not necessary to store their components in the array, for their
S-matrix elements can still be calculated.

The solutions $f_i (R) $ are the linear combination of the
$g_{i,k} (R) $
\begin{eqnarray} \label{linsum}
     f_i (R)  = \sum _{k=0} ^ m a_k g_{i,k} (R)
\end{eqnarray}
satisfying the boundary conditions of equation (\ref{asymp}) at
$R = R_m$ and say $R = R_m - 5h. $
The coefficient $ a_0 $ is always unity, to match the source terms
correctly.  The S-matrix elements are a by-product of the linear matching
equations (\ref{asymp}).

\begin{figure} \label{CCsoln}
\begin{tabular}{c|ccccccc|l}
\multicolumn{1}{c}{IT}&&&&&&&\multicolumn{1}{c}{}&\\
\cline{2-8}
1&$g_{1,0}$&$g_{2,0}$&..&  &   &   &$g_{m,0}$&{\em in}homogeneous solns.\\ \cline{2-8}
2&&(not used)&&\multicolumn{1}{c|}{}& ${\bf g}_{b+1,b+1}$&..&${\bf g}_{m,m}$ &uncoupled regular solns.\\ \cline{2-8}
3&${\bf g}_{1,1}$&$g_{2,1}$&..&\multicolumn{1}{c|}{}&\\ \cline{2-5}
4&$g_{1,2}$&${\bf g}_{2,2}$&..&\multicolumn{1}{c|}{}& & coupled regular solutions \\ \cline{2-5}
$\downarrow$&..&..&..&\multicolumn{1}{c|}{}& & (equations $1\rightarrow b$) \\ \cline{2-5}
$b+2$&$g_{1,b}$&$g_{2,b}$&..&\multicolumn{1}{c|}{${\bf g}_{b,b}$}\\ \cline{2-5}
\end{tabular}
\caption{ Independent Solution Vectors:
Layout of the independent radial wave functions for solving a system of
$m$ equations, of which $b$ are to coupled exactly. Each entry
represents a vector of $n$ radial points, and the entries in bold are
those with a non-zero initial values for their outward radial integration.}
\end{figure}

Note that the independent solutions $ g_{i,k} (R)$ for $k \geq 1$
need only be calculated the {\em first} time this coupled channels
set is used. If they are stored as in figure \ref{CCsoln},
subsequent iterations need only recalculate the first row (IT=1)
as the source terms vary.  Furthermore, if there are multiple incoming
channels for fixed total spin $J_T$ and parity $\omega$, then
solutions after the first can also use the $g_{i,k} (R)$ already
stored. The first iteration for these subsequent incoming channels will in
fact not require any radial integrations whatsoever, merely finding a new
set of $ \{ a_k \} $ from the new matching conditions,
and recalculating the sum (\ref{linsum}) if the
wave functions are required.

Tolsma and Veltkamp \cite{TOLV}
point out one difficulty with this method, which is that if the couplings
$ C_{i,j} $ are strong for $ i \neq j$, then the linear
independence of the $ g_{i,j} (R)$ will be reduced as $R$
increases through a classically forbidden region.  This is because the
components with negative local kinetic energy will generally consist of an
exponentially growing part and an exponentially decreasing part.  The
former is responsible for the tendency to destroy the initially generated
linear independence of the solution vectors. The longer the integration
continues through a classically forbidden region, the stronger this
tendency will be; for instance, it will occur in scattering problems of
nuclear physics with energies near or below the Coulomb barrier. It will
also occur if inelastic form factors are used which are not approximately
derivatives of the diagonal potential, but which extend more than usual
into the interior of the nucleus that is classically forbidden because
of the centrifugal potentials.

Tolsma et al. \cite{TOLV} propose a stabilization procedure to
monitor and if necessary re-orthogonalise the solution vectors.  If this
were not done, there would be large cancellations in the sum of equation
(\ref{linsum}), resulting if severe in complete loss of accuracy of the
S-matrix elements and the solution wave functions.

A simpler approach is to increase the starting radius $R_{\rm min}$
at which the radial integrations begin. It is advisable in any case for
reasons of stability at small radii to have a minimum radius proportional to
some angular momentum $\overline L $ typical of the coupled channels set:
\begin{eqnarray} \label{Rmin}
    R_{\rm min} \geq c \overline L h
\end{eqnarray}
for some constant $c$ in the region of 1 or 2, where $h$ is the radial
step size.  This constant could be increased to avoid the loss of
independence in the present problem, but this is not always satisfactory,
as the absorptive effects of the optical potentials at intermediate radii
might thereby be lost.  An alternative remedy
(adopted in ref.\cite{FRESCO}) is to have a specifiable radial
cutoff $R^(c)_min$ for the {\em off}-diagonal coupling
terms only. This allows the absorption in the diagonal potentials to be
effective at all radii outside $R_{\rm min}$ of equation (\ref{Rmin}),
but does not allow any strong coupling terms to lead to loss of
independence until some larger radius which can be adjusted to keep the
loss of accuracy to an acceptable level. It thus should be a regular
policy in a computer code to integrate the equations (\ref{diffeq}) to a
precision of at least 12 to 16 significant figures, to
monitor the degree of cancellation in equation (\ref{linsum}),
and to notify the user should this approach within 2 or
3 powers of ten of the precision limit of the computer.
Note that it is {\em not} necessary for the coupling terms
$ C_{ij} (R) $ (etc)
to be {\em accurate} to full machine precision, only that they should
be consistently {\em precise} when converted to that precision.
%%%
\subsection{Convergence of the Iterative Method}

The iterative method of solving the CRC equations (\ref{CRC1}, \ref{CRC})
will converge if the couplings are sufficiently small.
The procedure will however diverge if the the couplings are too large, or if
the system is too near a resonance.
%%% or some other kind of pole
%%% in the analytic continuation of the solution
%%% as a function of the strength of the couplings being iterated.
On divergence, the successive wave functions $\psi_i^{(n)}$
will become larger and larger as $n$ increases, and not converge to
any fixed limit. Unitarity will of course be violated as the S-matrix
elements will become much larger than unity.
\subsubsection{Improving the Convergence Rate}

There are several ways of dealing with this problem:
\begin{enumerate}
\item Solving some of the local couplings exactly by the methods of section \ref{ExCC},
and iterating only on the non-local couplings and the remaining
local couplings.
\item
Solving all the channels simultaneously via a very lar ge system of linear equations,
with each radial point in each channel as a separate unknown
\cite{bang83}.
\item
Find a separable expansion for the non-local kernels,
so that they can be included exactly in the coupled-channels solution \cite{PHD}.
\item
Expand the wave functions with a range of basis states
of Coulomb and (say) Gaussian \cite{KAWAI} or Airy \cite{PIECAN}
functions, and take the coefficients in this basis as the unknowns in a system of
linear equations.
\item
Use Pad\'e approximants to accelerate the convergence of the sequence
$S^{(n)} _\alpha $ of S-matrix elements \cite{ECIS1,ptolemy}.
\item Iterating the equations sequentially
as in \cite{ECIS1} and \cite{ptolemy}, rather than all equations as a block.
\item The inwards-outwards method of refs. \cite{ald77}, \cite{ichi0000}
and \cite{TOLSMA2}.
\end{enumerate}
For the range of heavy and light-ion reactions that we are considering here,
the methods (1) and (5) above have been adopted.

The method (2) is not used because of the size of the matrix that results.
Initially, the matrix would be sparse, with selected elements away from the
diagonal being non-zero because of the coupling potentials.
The kinetic energy operators occupy a band of width three along
the diagonal. Although a Gaussian elimination procedure
would allow potentials of arbitrary coupling strength to be included,
it will fill in large regions of the matrix as the solution proceeds,
and this makes the storage requirements prohibitive.

The separable expansion method (3), while useful for light-ion reactions,
is unsatisfactory for heavy-ion transfers. This is because if the
masses of the initial and final nuclei become large relative
to the mass of the transferred particle,
the form factor for the transfers becomes more nearly local.
As we approach the no-recoil limit (which makes the form factors exactly local)
a separable expansion of a nearly-local kernel will require
a large number of terms.  In the limit of a local form factor,
the separable expansion will require the same number of terms
as there are radial points.

The method (4) of expanding the wave functions in Gaussians
could have been used, provided the characteristic widths in R-space
of the basis states were chosen in accordance with the wave number
$K _\alpha$ in the relevant channel.
This requirement is less severe with light-ion reactions, where the
wave numbers are typically $ \leq 1$ fm$^{-1}$.
For heavy-ion reactions, however, the oscillation rates are much larger,
and a more sensible method is to expand in terms of Airy functions
that are depend explicitly on the local wave number over some radial region.

It is very useful to be able to iterate the coupled equations in
a conventional manner, as then 1, 2 and 3 step DWBA results (etc.)
can be recovered by stopping the iterations short of full convergence.
This recovery of DWBA results is more difficult with sequential iteration (6),
but both that method and the method of (7)
would be definitely advantageous when, say, exciting a long
rotational band by successive application of a quadrupole coupling.
Using Pad\'e acceleration has the advantages that it need only
be employed if ordinary iterations are seen to diverge,
and that it transforms the previously-divergent results
with little new computational effort.
%%%
\subsubsection{Pad\'e Approximants for Sequence Extrapolation}

A given sequence $S_0 , S_1 , \cdots $ of S-matrix elements
that result from iterating the CRC equations
can be regarded as the successive partial sums of the polynomial
\begin{eqnarray}
  f(\lambda) = S_0 + (S_1 - S_0) \lambda
                   + (S_2 - S_1) \lambda ^ 2 + \cdots
\end{eqnarray}
evaluated at $\lambda$=1.
This polynomial will clearly convergence for $\lambda$ sufficiently small,
but will necessarily diverge if the analytic continuation of the
$f (\lambda) $ function has any pole or singularities inside the circle
$ | \lambda |> 1 $ in the complex $\lambda$-plane.
The problem that Pad\'e approximants solve is that of finding a computable
approximation to the analytic continuation of the $f (\lambda) $ function.
This is accomplished by finding a rational approximation
\begin{eqnarray}
 P_{[n,m]} (\lambda) = {p_0 + p_1 \lambda + p_2 \lambda^2 + \cdots + p_n \lambda^n
                  \over
                   1       + q_1 \lambda + q_2 \lambda^2 + \cdots + q_m \lambda^m }
\end{eqnarray}
which agrees with the $f(\lambda)$ function in the region where the latter
does converge, as tested by matching the coefficients in the polynomial
expansion of $ P_{[n,m]} (\lambda) $ up to and including the coefficient of
$ \lambda^{n+m} $.

There are many different ways \cite{Pade}
of evaluating the coefficients
\{$p_m , q_n$\}, but for the present problem we can use
Wynn's $\epsilon$-algorithm \cite{wynn66}, which is a method of evaluating
the upper right half of the Pad\'e table at $\lambda$=1 directly
in terms of the original sequence  $S_0 , S_1 , \cdots $.

\subsubsection{Wynn's epsilon Algorithm}

Initialising $\epsilon_0^{(j)} = S_j $
and $ \epsilon^{(j)}_{-1} = 0 $,
we form an array
% \begin{eqnarray}
% { {\epsilon sup(0)_-1} .vab .vab {\epsilon sup(1)_-1}
%                        .vab .vab {\epsilon sup(2)_-1}
%                        .vab .vab {\epsilon sup(3)_-1} .vab .vab {..} } ~~~
% {                           .vab {\epsilon sup(0)_0}
%                        .vab .vab {\epsilon sup(1)_0}
%                        .vab .vab {\epsilon sup(2)_0} .vab .vab {..} } ~~~
% {                    .vab .vab
%   {\epsilon sup(0)_ 1} .vab .vab {\epsilon sup(1)_ 1} .vab .vab {..} } ~~~
%             { .vab .vab .vab {\epsilon sup(0)_ 2} .vab .vab {..} }
%             { .vab .vab .vab .vab .vab .vab {..}              }
% \end{eqnarray}
using the relation $ \epsilon^(j)_{k+1} = \epsilon^{(j+1)}_{k-1} +
  ( \epsilon_k^{(j+1)} - \epsilon_k^{(j)} ) ^ {-1} .  $
Thus we can construct the table given the second column from the initial sequence
$ S_j $.
The table then gives the transposed upper right half of the Pad\'e table,
including the diagonal:
\begin{eqnarray}
  \epsilon^{(j)}_{2k}  = P_{[k,k+j]} (1) .
\end{eqnarray}
Experience has shown that for typical sequences the most accurate Pad\'e
approximants are those near the diagonal of the Pad\'e table,
and these are just the right-most
$ \epsilon^{(0)}_{2k} $ in the $\epsilon$ table.

When accelerating a {\em vector} S-matrix elements
$ {\bf S}_j $, with a component for each coupled channel,
then it is important to accelerate the vector as a whole.
Wynn \cite{wynn61}
pointed out that this can be done using the Samuelson inverse
\begin{eqnarray}
  {\bf x}^{-1} = ({\bf x} \cdot {\bf x}^* ) ^ {-1} {\bf x}^*
\end{eqnarray}
where $ {\bf x}^* $ is the complex conjugate of ${\bf x}$.
Otherwise there will be problems when iterating (say) a two-channel
system with alternating backwards and forwards coupling,
because of zero divisors in the $\epsilon$ algorithm.
%%%
\subsection{Transfer Form Factors}
\subsubsection{The Cancellation Problem}

As discussed in section \ref{EFR},
the summations over $T$ in equation (\ref{useq}) involve large cancellations,
and as the degree of cancellation gets worse for large
$ \ell $ and $ \ell' $, this places a limit
on the maximum value $ \ell + \ell' $ of the
transferred angular momentum.

Typically, however, the transfer form factors are only needed to be accurate to
around 0.1 to 1\%,
so if computer arithmetic is used which is accurate to 14 or
16 decimal digits, then cancellations up to 12 or 13 orders of magnitude
should in principle not result in catastrophic errors in the transfer
rates.  With careful programming, this accuracy can be achieved.
What is necessary is to be careful that all quantities in the equations
(\ref{useq}, \ref{genq}) above which depend on the Legendre order $T$ are calculated
to the full computer precision.  It is not necessary, for example,
for the channel wave functions $ f _\alpha (R) $, the bound
state wave functions $u_{\ell sjI} (r) $ or the
quadrature of the integral (\ref{genq}) to be accurate to full precision
(which in any case would be impossible). It is only necessary
that all these quantities have {\em exactly the same computer
precision} when the coefficients over $T$ (the
${\bf q}_{\ell , \ell'}^T (R,R') )$ are evaluated,
and when the sums over $T$ (in equation \ref{useq}) are performed.
This will require principally that the `radial framework' that gives
$ \vecr $ and $ \vecr' $ in terms of $\vecR$ and $\vecR'$
be accurate to full machine precision, as also the Racah algebra
coefficients in equation (\ref{useq}). In fact, the channel wave functions
$ f _\alpha (R) $ and the bound state wave functions
$u_{\ell sjI} (r) $ may be calculated with reduced precisions
using shorter computer words and faster arithmetic should
these be available. It is also not necessary for the coefficients and sums
over $T$ be consistent to full accuracy for different $R$ and
$R'$ values, as the large cancellations only occur between
different $T$ values for each separate $R$ and $R'$ combination.

Since the accuracy of the quadrature in the equation (\ref{genq}) is not critical
to the overall accuracy of the transfers, calculations may be speeded up if we
economise on the range of the $u$ variable and on the
number of intermediate steps required.
Even in light ion reactions it is not necessary to integrate $u$ to $-1$
($\theta$ to 180$^\circ$) as was done in the code LOLA
\cite{LOLA}
for example.  An efficient procedure to adopt is that used in the DWBA
code DAISY\cite{DAISY}, where, for each successive $R$ value, the code monitors the rate
of decay of the integrand as $\theta$ increases.  For a given accuracy
criterion, an estimate can then be made of an adequate upper limit
for the $\theta$ integration at the next $R$ value.  Typically, the upper
limits of $\theta$ decrease monotonically as $R$ increases from 0 to the
upper limit $R_m$.
Because the integrand is largest for $\theta$=0, the accuracy of the angular
integration for small $\theta$ is improved by a change of variable from $u$
to $x$ as in ref.\cite{DAISY}:
\begin{eqnarray}
 \theta = \frac{1}{4} (3 x^2 + 1) x \theta_{\rm max}
\end{eqnarray}
for $0 \leq x \leq 1. $  The quadrature over $u$ of equation (\ref{genq}) then becomes
\begin{eqnarray}
 {\bf q}^T_ {\ell , \ell'} (R,R')
 =
   \half \int_ 0 ^ 1
         {\bf V} {u_{\ell sj} (r)\over r^{\ell + 1}} ~
               {u_{\ell' sj'} (r')\over {r'}^{\ell' +1}} ~
              P_T (u) \sin(\theta) {d \theta \over dx} dx.
\end{eqnarray}
The parameter $\theta_{\rm max}$ is adjusted for each successive
value of $R$. according to the rate at which the integrand is observed to decay
as $\theta$ increases, as described earlier.
%%%
\subsubsection{Radial Grids}

The methods used to calculate, store and use the non-local form factors
$ qu^T_{\ell , \ell'} (R,R') $ (equation \ref{genq}) and
$V_{\alpha ,\alpha'} (R,R') $ (equation \ref{Vtransf})
have to be efficient in a wide variety of reactions, from light-ion
reactions such as
$^3$He($^3$H,$^4$He)$^2$H or $^{16}$O($^{20}$Ne,$^{24}$Mg)$^{12}$C
to heavy-ion reactions, such as nickel on tin one-nucleon transfers.
In the former cases, the radial form factors
$V_{\alpha ,\alpha'} (R,R') $
will be non-zero over large regions of the $R-R'$ space, so
(following ref \cite{SM1}) interpolation procedures should prove effective.

However, when small masses are transferred between two larger nuclei
the form factor is nearly local, and only large around
$R \sim R' .$
If the whole $(R,R')$ array had to be calculated and stored in
these cases,
modelling heavy-ion transfers would become inefficient,
even with interpolation methods.
The form factor now varies rapidly as a function of
$\delta R \equiv R - R'$ (especially for heavy ion reactions,
as the Jacobian $b^3$ in equation (\ref{useq}) becomes large),
and varies only slowly with $R$ (if $\delta R$ is constant),
as this variation follows the radial dependence of the bound state
wave functions.
The best procedure is thus \cite{SM1}
to first change to the coordinate pair
$\delta R$ and $R$, and then to use different interpolatory
intervals $h_\delta$ and $h_R$ in the two directions
respectively. Then, when nuclear masses become large compared with the mass
of the transferred particle, $h_\delta$ can become smaller,
perhaps even smaller than $h$, the basic radial step size.

The method adopted in FRESCO is to let the user specify
$h_\delta$ and $h_R $ as multiples or
submultiples of $h$. The value of $h_R $ is very often always 3 to 5
times larger than $h$, as this reflects the typical rate at which bound
state wave functions vary.
If the bound state wave functions have many internal nodes, then the
interpolation interval $h_R$ cannot be so large
(this is often the case with $\alpha$-particle bound states).

The $h_\delta ,$
on the other hand, will be larger than $h$ for light-ion reactions
(as described in \cite{TUM}), but will be comparable with or smaller than
$h$ for few-nucleon transfers between heavy ions.  The user also
specifies the maximum and minimum values of the range of $\delta R ,$
which again will be large ( $\sim$ nuclear radii) for light ions,
and small ( $\sim$ 1 or 2 fm.) for heavy ion reactions.  The accuracy of these
choices is checked retrospectively by collecting statistics on the
distributions of the functions
$ {\bf q}^T_{\ell , \ell'} (R, \delta R),$ averaging
over $R$. and all partial waves $T$, $\ell$, and $\ell' .$

When $h_\delta$ or $h_R $ are {\em multiples} $h$,
then (say) cubic splines in two dimensions can be used to expand the form
factors for the integrals of equation (\ref{sourceint}).
If, however, $h_\delta$ is a {\em submultiple} $h$,
as is the case in many heavy-ion reactions,
then a more efficient procedure is possible.

Suppose, say, we wish to evaluate the numerical integral
\begin{eqnarray}
   {\cal I} = \sum _ j V(x_j) \overline f (x_j) ,
\end{eqnarray}
where the $\overline f (x_j)$ are the interpolated values of the
function $f(x)$ between its stored values $f_i$ at
$x = (i-1)h .$ Let the interpolation method be linear:
\begin{eqnarray}
  \overline f (x) = \sum _ m  a_m (x) f_m
\end{eqnarray}
for some $x$-dependent coefficients $a_m (x)$ from (say) fitting
cubic splines over some range (most of the $a_m$ will be zero except
for $ m\sim i\pm 2 $).  Then $\cal I$ can be evaluated directly
in terms of the $f_m : $
\begin{eqnarray}
 {\cal I} &=& \sum _ j V(x_j)
            \sum _ m  a_m (x_j)  f_m
\\
 &=& \sum _ m \overline V_m f_m
\end{eqnarray}
where
\begin{eqnarray} \label{preempt}
\overline V_m \equiv \sum _ j V(x_j) a_m (x_j)
\end{eqnarray}
is a new effective form factor
This means that when $h_\delta$ is a submultiple of $h$,
we do not need to store a form factor at intervals of $h_\delta ,$
only at intervals of $h$, if we use the `preemptive
interpolation' of equation (\ref{preempt}).
This has the further advantage that as the no-recoil limit is approached
(as the mass of the transferred particle becomes a smaller fraction of the
interacting nuclei), then the form factors
$ \overline{\bf q}^T_{\ell , \ell'} (R, \delta R) $ and
$\overline V_{\alpha ,\alpha'} (R, \delta R) $
need fewer grid points in the
$\delta R$ direction. Less arithmetic is needed to evaluate
the source functions of equation (\ref{sourceint}), which change from
\begin{eqnarray}
  S _\alpha (R) = \int_ {\delta R_{\rm min}} ^ {\delta R_{\rm max}}
           \overline V_{\alpha ,\alpha' } (R, \delta R) ~
           f_{\alpha'} (R - \delta R)  d( \delta R)           .
\end{eqnarray}
to
\begin{eqnarray}
 S _\alpha (i h_R) = h \sum _ j
          \overline V_{\alpha ,\alpha' } (ih_R , j h)
          f_{\alpha'} ((in_R - j)h)
          \mbox{  where } n_R \equiv h_R / h
\end{eqnarray}
even when the original kernel functions
vary rapidly as $\delta R$ changes in steps of $h$ (with $R$ constant).

{\em Simultaneous Two-Nucleon Transfers}:
A similar `preemptive' summation is possible when calculating the
form factors for the simultaneous transfer of two nucleons between states
of the form of equation (\ref{tntwf}) in the projectile and in the target.
As mentioned in section \ref{NNBS},
two-nucleon transfer can be viewed as the transfer of a
`structured particle' with internal coordinates
$ ( \ell , (s_1 s_2) S)j $ and
$\rho$, the distance between the two nucleons.  A transfer is only possible
if the initial and final states have identical values for these `internal
coordinates'. The angular momentum quantum numbers can be matched
exactly, but source terms can either be constructed for each $\rho$
value and summed in equation (\ref{sourceint}),
or the separate $\rho$ products can be summed as early as equation (\ref{genq}).
Because the separate $\rho$ values are only used in a summation,
it is most economical to use Gaussian quadrature,
as for a given accuracy this reduces by a half
the number of $\rho_i$ values at which the wave functions of equation
(\ref{tntwf}) need to be calculated and stored.
If the $\rho_i$ are chosen to be the Gaussian quadrature points over
some chosen range, and if $ w_i $ are the corresponding weights,
the equation (\ref{genq}) becomes
\begin{eqnarray}
 {\bf q}^T_ {L , L'} (R,R')  =
   \half \int_ {-1}^{+1}
         r^{-L-1} {r'}^{-L' -1} ~
        [ \sum _ i  w_i {\bf V} u_{12} (r, \rho_i) ~
                                   u'_{12} (r' , \rho_i) ]
              P_T (u) du.
\end{eqnarray}
Equation (\ref{useq}) remains unchanged, and this means that two nucleon
transfers can be calculated efficiently with little more computational work
than that required for single-particle transfers.
%%%
\subsubsection*{Acknowledgements}

I would like to thank Dr. Nagarajan, John Lilley, Ray Mackintosh,
Victoria Andres and a referee for valuable questions and comments at various stages
in this project.
Manchester University, Daresbury Laboratory, the Niels Bohr Institute, and
Bristol University are thanked for their hospitality when these ideas were
being developed, and the Department of Engineering Mathematics at Bristol
is thanked for their assistance in the preparation of this paper.


\begin{thebibliography}{99}
\itemsep=-2pt
\bibitem{TAM65}
T. Tamura, Rev. Mod. Phys. {\bf 37} (1965) 679.
\bibitem{tam74}
T. Tamura, Phys. Reports {\bf 14C} (1974) 59.
\bibitem{aust70}
N. Austern, {\em Direct Nuclear Reaction Theories} (Wiley, New York, 1970).
\bibitem{rbrown80}
M. Rhoades-Brown, M.H. Macfarlane and S.C. Pieper, Phys. Rev. C {\bf 21} (1980) 2417.
\bibitem{PIECAN}
L.D. Tolsma, Phys. Rev. C {\bf 20} (1979) 592, J. Comp. Phys. {\bf 17} (1975) 384.
\bibitem{fesh58}
H. Feshbach, Annals of Phys. (N.Y.) {\bf 5} (1958) 357, {\bf 19} (1962) 287.
\bibitem{uda73}
T. Udagawa, H.H. Wolter and W.R. Coker, Phys. Rev. Letts.
{\bf 31} (1973) 1507.
\bibitem{iman74}
B. Imanishi, M. Ichimura and M. Kawai, Phys. Lett. {\bf 52B} (1974) 267.
\bibitem{lill83}
J.S. Lilley, B.R. Fulton, D.W. Banes, T.M. Cormier, I.J. Thompson, S. Landowne,
and H.H. Wolter, Phys. Letts. {\bf 128B} (1983) 153, and
J.S. Lilley, M.A. Nagarajan, D.W. Banes, B.R. Fulton and I.J. Thompson,
Nucl. Phys. {\bf A463} (1987) 710.
\bibitem{ogle71}
W. Ogle, S. Wahlborn, R. Piepenbring and S. Frederiksson, Rev. Mod. Phys.
{\bf 43} (1971) 424.
\bibitem{vaag79}
J.S. Vaagen, B.S. Nilsson, J. Bang and R.M. Ibarra,
Nucl. Phys. {\bf A319} (1979) 143;
J.M. Bang, F.G. Gareev, W.T. Pinkston and J.S. Vaagen, Phys. Reports
{\bf 125} (1985) 253.
\bibitem{mosh60}
T.A. Brody and M. Moshinsky, {\em Tables of Transformation Brackets}.,
(Monografias del Instituto de Fisica, Mexico, 1960);
D.H. Feng and T. Tamura, Comput. Phys. Commun. {\bf 10} (1975) 87.
\bibitem{bay67}
B. Bayman and A. Kallio, Phys. Rev. {\bf 156} (1967) 1121.
\bibitem{berry66}
M.V. Berry, Proc. Phys. Soc. {\bf 89} (1966) 479,
J. Phys. B: Atom. Molec. Phys. {\bf 2} (1969) 381;
R.C. Fuller, Phys. Rev. C {\bf 12} (1975) 1561.
\bibitem{row78}
N. Rowley, J. Phys. A: Math. Gen. {\bf 11} (1978) 1545.
\bibitem{simon74}
M. Simonius, {\em Lecture Notes in Physics}.,
Ed D. Fick, {\bf 30} (Springer, 1974) 38.
\bibitem{iman69}
B. Imanishi, Nucl. Phys. {\bf A125} (1969) 33.
\bibitem{satch60}
G.R. Satchler, Nucl. Phys. {\bf 21} (1960) 116.
\bibitem{bs68}
D.M. Brink and G.R. Satchler, {\em Angular Momentum}.,
(Clarendon Press, Oxford, 1968).
\bibitem{alder66}
K. Alder and \AA. Winther, {\em Coulomb Excitation}
(Academic Press, New York, 1966)
and {\em Electromagnetic Excitation} (North Holland, Amsterdam, 1975).
\bibitem{aust64}
N. Austern, R.M. Drisko, E.C. Halbert and G.R. Satchler, Phys. Rev.
{\bf 133} (1964) B3.
\bibitem{robson72}
D. Robson and R.D. Koshel, Phys. Rev. {\bf C6} (1972) 1125.
\bibitem{nagel76}
P. Nagel and R.D. Koshel, Phys. Rev. {\bf C13} (1976) 907,
Comput. Phys. Commun. {\bf 15} (1978) 193.
\bibitem{buttle64}
P.J.A. Buttle and L.J.B. Goldfarb,
Proc. Phys. Soc. (London) {\bf 83} (1964) 701;
for a review see G.R. Satchler,  \cite[\S 6.14.1]{SATCH83}.
\bibitem{gold68}
L.J.B. Goldfarb and E. Parry, Nucl. Phys. {\bf A116} (1968) 309.
\bibitem{tam67}
T. Tamura, Oak Ridge National Laboratory Report No. ORNL-4152 (1967).
\bibitem{melk66}
M.A. Melkanoff, T. Sawada and J. Raynal,
{\em Methods in Computational Physics}, Vol. 6 (1966),
Academic Press (New York).
\bibitem{bang83}
J. Bang, J.-J. Benayoun, C. Gignoux, and I.J. Thompson,
Nucl. Phys. {\bf A405} (1983) 126.
\bibitem{ald77}
K. Alder, F. R\"osel and R. Morf, Nucl. Phys. {\bf A284} (1977) 145.
\bibitem{ichi0000}
M. Ichimura, M. Igarashi, S. Landowne, C.H. Dasso, B.S. Nilsson, R.A. Broglia
and \AA. Winter, Phys. Letts. {\bf 67B} 129.
\bibitem{wynn66}
P.Wynn, Numer. Math. {\bf 8} (1966) 264;
see also A. Genz, p 112 in ref.\cite{Pade}
\bibitem{wynn61}
P. Wynn, Math. Comput. {\bf 16} (1961) 23.
\bibitem{SATCH83}
G.R. Satchler, {\em Direct Nuclear Reactions}. (Clarendon, Oxford, 1983).
\bibitem{FRESCO}
I.J. Thompson, FRESCO users' manual and code, available from
the author.
\bibitem{AROSA}
F. R\"osel, J.X. Saladin and K. Alder, Comput. Phys. Commun. {\bf 8}
(1974) 35.
\bibitem{AROSA2}
G.H. Rawitscher and C.H. Rasmussen, Comput. Phys. Commun. {\bf 11}
(1976) 183.
\bibitem{TOLSMA2}
L.D. Tolsma, Phys. Rev. C {\bf 35} (1987) 177.
\bibitem{ECIS2}
J.D. Raynal, Phys. Rev. {\bf C23} (1983) 2571.
\bibitem{CDCC}
G.H. Rawitscher, Phys. Rev. {\bf C9} (1974) 2210,
{\bf C11} (1975) 1152, Nucl. Phys. {\bf A241} (1975) 365.
\\
M. Yahiro and M. Kamimura, Prog. Theor. Phys. {\bf 65} (1981) 2046,
{\bf 65} (1981) 2051.
\bibitem{CDCC2}
Y. Sakuragi, M. Yahiro and M. Kamimura, Prog. Theor. Phys. Suppl.
{\bf 89} (1986).
\bibitem{CV}
S.R. Cotanch and C.M. Vincent, Phys. Rev. {\bf C14} (1976) 1739.
\bibitem{OSAKA}
I.J. Thompson, in: {\em Light Ion Reaction Mechanism}., proc. of RCNP
Symposi\bibitem{OUKID}
P. Nagel and R.D. Koshel, Phys. Rev. {\bf C13} (1976) 907,
Comput. Phys. Commun. {\bf 15} (1978) 193.um, Osaka, 1983, ed. H. Ogata.
\bibitem{COULCC}
I.J. Thompson and A.R. Barnett, Comput. Phys. Commun. {\bf 36}
(1985) 363, J. Comp. Phys. {\bf 64} (1986) 490.
\bibitem{Sturm}
J.M. Bang and  J.S. Vaagen, Z. Physik A {\bf 297} (1980) 223.
\bibitem{OHMURA}
T. Ohmura, B. Imanishi, M. Ichimura and M. Kawai, Prog. Theor. Phys.
{\bf 41} (1969) 391, {\bf 43} (1970) 347,
{\bf 44} (1970) 1242.
\bibitem{TUM}
T. Tamura, T. Udagawa, K.E. Wood and H. Amakawa, Comput. Phys. Commun.
{\bf 18} (1979) 63.
\\
T. Tamura, T. Udagawa and M.C. Mermaz,  Phys. Reports {\bf 65} (1980) 345.
\bibitem{Pade}
P.R. Graves-Morris, {\em Pad\'e Approximants},
(Institute of Physics, London and Bristol, 1973)
\bibitem{PHD}
I.J. Thompson, Ph.D. report (University of Auckland, 1979, unpublished).\bibitem{KAWAI}
M. Kawai, M. Kamimura, Y. Mito, K. Takesako, Prog. Theor. Phys.
{\bf 59} (1978) 674 and 676.
\\
M. Kamimura, Prog. Theor. Phys. Suppl. No. 62 (1977) 236.
\bibitem{ECIS1}
J. Raynal, {\em Computing As a Language of Physics},
(IAEA, Vienna, 1972) 281.
\bibitem{ptolemy}
M. Rhoades-Brown, M.H. Macfarlane and S.C. Pieper,
Phys. Rev. C {\bf 21} (1980) 2436.
\bibitem{BSH}
B. Buck, A.P. Stamp and P.E. Hodgson, Phil. Mag., {\bf 8} (1963)
1805.
\bibitem{TOLV}
L.D. Tolsma and G.W. Veltkamp, Comput. Phys. Commun. {\bf 40} (1986)
233.
\bibitem{DAISY}
P.J.A. Buttle, Comput. Phys. Commun. {\bf 14} (1978) 133.
\bibitem{SM1}
T. Tamura,  Comput. Phys. Commun. {\bf 8} (1974) 349.
\bibitem{YOSH}
S. Yoshida, Proc. Phys. Soc. (London) {\bf A69} (1956) 668.

\bibitem{APS}
A.P. Stamp, Nucl. Phys. B{\bf 3} (1966) 232.

\bibitem{RAWIT}
G.H. Rawitscher, Phys. Rev. {\bf 163} (1967) 1223.
\\
G.H. Rawitscher and S.N. Mukherjee, Phys. Rev. {\bf 181} (1969) 1518.

\bibitem{Bang}
J.M. Bang and S. Wollesen, Phys. Lett. {\bf 33B} (1970) 395.

\bibitem{asc}
R.J. Ascuitto, N.K. Glendinning and B. S{\o}rensen,
Nucl. Phys. {\bf A170} (1971) 65.

\bibitem{RMA}
R.S. Mackintosh, Nucl. Phys. {\bf A209} (1973) 91.

\bibitem{SCHB}
R. Schaeffer and G.F. Bertsch, Phys. Lett. {\bf 38B} (1972) 159.

\bibitem{Toyama}
M. Toyama, Phys. Lett. {\bf 38B} (1972) 147.

\bibitem{CHUCK}
P.D. Kunz, Program CHUCK3 (University of Colorado, 1978, unpublished),

\bibitem{CHUCK3}
Extended version of \cite{CHUCK}
by J.R. Comfort, (University of Pittsburgh, 1980, unpublished).

\bibitem{QUICC}
A.J. Baltz, Phys. Rev. {\bf C25} (1982) 240.

\bibitem{C13}
B. Imanishi and W. von-Oertzen, Phys. Lett. {\bf 87B} (1979) 188.

\bibitem{TWOSTP}
M. Toyama and M. Igarashi, 1972 and 1977, unpublished.

\bibitem{ZAFRA}
N.M. Clarke, J. Phys. G: Nucl. Phys. {\bf 10} (1984) 1219.

\bibitem{MOSH}
M. Moshinsky, Nucl. Phys. {\bf 13} (1959) 104.
\bibitem{LOLA}
R.M. DeVries, Phys. Rev. {\bf C8} (1973) 951.
\end{thebibliography}
%%%
\appendix
\newpage
\section{Notation and Phase Conventions}

\subsubsection*{Spherical Harmonics}
  The phase convention used here is
\begin{eqnarray} \nonumber
 Y_L^m ( \theta ,\phi ) =
  \sqrt {{ 2L+1 \over 4 \pi} {(L-m)! \over (L+m)!} } (-1)^m e^{im\phi}
  P_L^m ( \cos \theta )
\end{eqnarray}
for $ m \geq 0 $, and
$ Y_L^{-m} = (-1)^m Y_L^{m *}$
to give negative $ m$ values.
%%%
\subsubsection*{Angular Momentum Coupling Coefficients}

%\newcommand{\ba}{\begin{array}}
\newcommand{\baa}{\begin{array}{c}}
\newcommand{\bab}{\begin{array}{cc}}
\newcommand{\bac}{\begin{array}{ccc}}
\newcommand{\ea}{\end{array}}

The notation $\langle \ell_1 m_1 \ell_2 m_2 | LM\rangle $
has been used for the Clebsch-Gordon coupling coefficient
for coupling states $\ell_1 m_1 $ and
$\ell_2 m_2 $ together to form $LM.$
The
\begin{eqnarray} \nonumber
%   \left ( a .ab\alpha b .ab beta c above\ gamma \right ) \equiv ~
    \left ( \bac a&b&c\\ \alpha&\beta&\gamma \ea \right ) \equiv ~
%    \left ( \begin{array}{ccc} a&b&c\\ \alpha&\beta&\gamma \end{array} \right ) \equiv ~
        {(-1)^{a - b -\gamma} \over \hat c}
                    \langle a\alpha b \beta | c -\gamma\rangle
\end{eqnarray}
represents the Wigner 3-$j$ symbol, and
\begin{eqnarray} \nonumber
     \hat x \equiv \sqrt{2x+1}  .
\end{eqnarray}
The 9-$j$ coupling coefficient is used in two forms related by
\begin{eqnarray} \nonumber
   \left \{ \bac a&b&c\\d&e&f\\g&h&i \ea \right \}
      \equiv \hat c \hat f \hat g \hat h ~
   \left ( \bac a&b&c\\d&e&f\\g&h&i \ea \right ) .
\end{eqnarray}
The binomial coefficient is
\begin{eqnarray} \nonumber
\left ( \baa x \\ y \ea \right ) = {x! \over y!(x-y)!} .
\end{eqnarray}
%%%
\newpage
\section{Coupled Channels Codes in Nuclear Physics}

There is a natural progression of complexity in the codes being
considered here:
\begin{enumerate}
\item One-step (DWBA) codes
 Inelastic excitations
 Zero-range transfers (ZR)
 No-recoil transfers (NR)
%%%
\item Coupled channels (CC) codes with local form factors
 Inelastic excitations (CC)
 Zero-range transfers (ZR-CC, sometimes included in CCBA)
 No-recoil transfers (NR-CC, sometimes included in CCBA)
%%%
\item One-step (DWBA) codes for exact finite-range transfers (EFR-DWBA)
%%%
\item Coupled channels Born approximation (CCBA): a coupled set of channels
followed by a finite-range transfer (sometimes called EFR-CCBA).
%%%
\item Two-step DWBA
 Zero-range transfers (2-step ZR-DWBA)
 No-recoil transfers (2-step NR-DWBA)
%%%
\item One-step DWBA codes for exact finite-range transfers (EFR-DWBA)
%%%
\item Two-step DWBA for exact finite-range transfers (2-step EFR-DWBA)
%%%
\item Coupled reaction channels (CRC), allowing finite-range transfers.
%%%
\end{enumerate}

The following is a summary of the more widely known coupled channels
codes (codes which can only perform one-step DWBA calculations have
been excluded).
\begin{enumerate}
\itemsep=-2pt
%%.rc 3 on/off
\item Yoshida \cite{YOSH}: Inelastic CC with $\delta$-function interactions
\item Buck, Stamp and Hodgson \cite{BSH} and Satchler (see ref.\cite{BSH}): Inelastic CC
\item Tamura \cite{TAM65}: General purpose inelastic CC
\item Stamp \cite{APS}: ZR-CC
\item Rawitscher \cite{RAWIT}: ZR-CC using iterated Green functions
\item Tamura and Low \cite{SM1} and \cite{TUM}: Saturn-Mars - NR-DWBA and EFR-DWBA
%%% \item Tamura, Low and Udagawa (1973, unpublished): Saturn-Mars II - EFR-CCBA
%%% \item Udagawa, Low and Tamura (1974, unpublished): Saturn-Mars III - EFR-CRC ?
\item Ohmura et al. \cite{OHMURA}: CRC for deuterons
\item Ascuitto et al. \cite{asc}: ZR-CCBA using source terms
\item Mackintosh \cite{RMA}: ZR-CRC for deuterons and protons
\item Bang and Wollesen \cite{Bang}: two-step ZR-DWBA.
\item Toyama \cite{Toyama}: two-step ZR-DWBA
\item Schaeffer and Bertsch \cite{SCHB}: two-step ZR-DWBA
\item R\"osel et al. \cite{AROSA}, extended by Rawitscher \cite{AROSA2}:
AROSA - CC for Coulomb excitations
\item Cotanch and Vincent \cite{CV}: CRC for deuterons
\item Raynal \cite{ECIS1} and \cite{ECIS2}:
General purpose inelastic CC and ZR-CC
\item Kunz \cite{CHUCK}, and later Comfort \cite{CHUCK3}:
CHUCK - General purpose CC (inelastic and ZR)
\item Nagel and Koshel \cite{OUKID}: OUKID - EFR-CCBA for light ions
\item Kawai \cite{KAWAI}: CRC for deuterons
\item Baltz \cite{QUICC}: QUICC - Inelastic CC for heavy ions
\item Imanishi \cite{C13}: CRC for $^{12}$C+$^{13}$C reactions
\item Tolsma \cite{PIECAN} and \cite{TOLSMA2}:
PIECANSOL - Inelastic CC for heavy ions
\item Toyama and Igarashi \cite{TWOSTP}: TWOSTP - 2-step ZR- and EFR-DWBA for light ions, and
\\
Igarashi: TWOFNR/PTFF - 2-step EFR-DWBA for sequential and simultaneous
transfers of two nucleons for light ion reactions.
\item Thompson \cite{PHD}: CRC for deuterons
\item MacFarlane, Pieper and Rhoades-Brown \cite{ptolemy}\\
 PTOLEMY/1 - Inelastic DWBA and EFR-DWBA for heavy and light ions\\
 PTOLEMY/2 - General purpose inelastic CC for heavy ions
\item Kunz  : CHORK - ZR-CC, and NR-CC for heavy ions
\item Thompson \cite{FRESCO}: FRESCO - General purpose CRC for light
and heavy ions
\item Clarke \cite{ZAFRA}: a new `zero-angle' approximation for
CC finite-range transfers (i.e. `ZA-CC')
\end{enumerate}

Note that it is sometimes difficult to put these developments in a definite
chronological order.





\end{document}
